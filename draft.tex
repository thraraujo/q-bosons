\documentclass[a4paper,11pt]{amsart}

\usepackage{/home/thiago/.config/dot-files/latex/globaldef}

% DEFINITIONS

%%%%%%%%%%%%%%%%%%%%%%%%
% ADDITIONAL PACKAGES
%%%%%%%%%%%%%%%%%%%%%%%% 
\usepackage{parskip}
\usepackage{tikz} % Drawing package
\usepackage[backend=biber, style=alphabetic]{biblatex}
\usepackage{listings}
\usepackage{ytableau}

\usetikzlibrary{intersections,decorations.text} % this is to make the cover
\usetikzlibrary{matrix, arrows.meta} % this is to improve diagrams

\setlength\parindent{0pt}

%\setdefaultlanguage[variant=brazilian]{portuguese}
%\setdefaultlanguage{english}

%%%%%%%%%%%%%%%%%%%%%%%%
% METADATA
%%%%%%%%%%%%%%%%%%%%%%%% 

%\usepackage{framed} 


% Official manual defines the options
\hypersetup{ 
	pdftitle={integrable systems, programming and so on},
	pdfsubject={High Energy Physics and so on}, 
	pdfauthor={author},
	pdfkeywords={gauge; susy; strings; fields; cft; python},
	colorlinks=true, % false: color frames ; true: color links
    linkcolor=myPurple, % Color of internal links (sections, pages and so on)
    citecolor=myPurple, % color for bibliographical citations
    urlcolor =myPurple, % color for linked URL
    linktocpage=true % link page to table of contents
}


% Options for listings package I saw the sniipet below here: 
% https://stackoverflow.com/questions/3175105/inserting-code-in-this-latex-document-with-indentation
\lstset{frame=tb,
  language=Python,
  aboveskip=3mm,
  belowskip=3mm,
  showstringspaces=false,
  columns=flexible,
  basicstyle={\small\ttfamily},
  numbers=none,
  numberstyle=\tiny\color{myPurple},
  keywordstyle=\color{myRed},
  commentstyle=\color{myBlue},
  stringstyle=\color{myPurple!80},
  breaklines=true,
  breakatwhitespace=true,
  tabsize=3
}

%%%%%%%%%%%%%%%%%%%%%%%% 
% COLORS
%%%%%%%%%%%%%%%%%%%%%%%% 
% see palette here: https://github.com/enkia/tokyo-night-vscode-theme
\definecolor{myPurple}{RGB}{90, 74, 120}
\definecolor{myBlue}{RGB}{15, 75, 110}
\definecolor{myRed}{RGB}{191,97,106}
\definecolor{myDarkGray}{RGB}{216, 222, 233}
\definecolor{myLightGray}{RGB}{236, 239, 244}

\definecolor{c1}{RGB}{129, 162, 193}%  {15, 75, 110} % myBlue
\definecolor{c2}{RGB}{216, 222, 233} % myDarkGray
\definecolor{c3}{RGB}{236, 239, 244} % myLightGray
\definecolor{c4}{RGB}{59, 66, 82}
\definecolor{c5}{RGB}{76, 86, 106}

% This color is a framed requirement
\definecolor{shadecolor}{RGB}{236, 239, 244}

%%%%%%%%%%%%%%%%%%%%%%%%
% THEOREM
%%%%%%%%%%%%%%%%%%%%%%%% 

\newtheorem{theorem}{Theorem}
\newtheorem{corollary}{Corollary}
\newtheorem{proposition}{Proposition}
\newtheorem{conjecture}{Conjecture}
\newtheorem{lemma}{Lemma}
\newtheorem{example}{Example}
\newtheorem{remark}{Remark}

%%%%%%%%%%%%%%%%%%%%%%%%
% MACROS
%%%%%%%%%%%%%%%%%%%%%%%% 

\DeclarePairedDelimiter{\bra}{\langle}{\rvert}
\DeclarePairedDelimiter{\ket}{\lvert}{\rangle}
\DeclarePairedDelimiter{\bbra}{\langle\!\langle}{\rvert}
\DeclarePairedDelimiter{\kket}{\lvert}{\rangle\!\rangle}
\DeclarePairedDelimiterX{\bracket}[2]{\langle}{\rangle}{#1\vert#2}
\DeclarePairedDelimiterX{\bbracket}[2]{\langle\!\langle}{\rangle\!\rangle}{#1\vert#2}


\bibliography{bib-database.bib}

\usepackage{amsaddr} % affiliation in the first page

\begin{document}

%%%%%%%%%%%%%%%%%%%%%%%%%%%%%%%%%%%%%%%%%%%%%%%%%%
%%%%%%%%%%%%%%%%%%%%%%%%%%%%%%%%%%%%%%%%%%%%%%%%%%

\title[\(Q\)-Bosons and relations with integrable hierarchies]{\(Q\)-Bosons
and their relations with integrable hierarchies}

\author{Thiago Araujo}

\address{\noindent 
Instituto de Física Teórica, UNESP-Universidade Estadual Paulista,
R. Dr. Bento T. Ferraz 271, Bl. II, Sao Paulo 01140-070, SP, Brazil\\
\&
Instituto de Física, Universidade de S\~ao Paulo,
Rua do Matão Travessa 1371, 05508-090 São Paulo, SP. Brazil
}

\email{\texttt{\href{tr.araujo@unesp.br}{tr.araujo@unesp.br}}}

\begin{abstract}
In this work, we study aspects of the \(Q\)-boson model, and 
its relations with integrable hierarchies. 
\end{abstract}

%\date{\today}
\keywords{Integrability, Bethe, Schur}
\subjclass[2020]{82B20, 82B23}
\maketitle

\setcounter{tocdepth}{2}
\tableofcontents

%*%*%*%*%*%*%*%*%*%*%*%*%*%*%*%*%*%*%*%*%*%*%*%*%*
\section{Introduction}
%*%*%*%*%*%*%*%*%*%*%*%*%*%*%*%*%*%*%*%*%*%*%*%*%*


%*%*%*%*%*%*%*%*%*%*%*%*%*%*%*%*%*%*%*%*%*%*%*%*%*
\section{Phase Model and q-Bosons}
%*%*%*%*%*%*%*%*%*%*%*%*%*%*%*%*%*%*%*%*%*%*%*%*%*

This section introduces the \(q\)-boson and phase
models~\cite{Bogoliubov:1992, Bogoliubov:1997soj, Bogoliubov2005,
  Bogoliubov:1997soj, Tsilevich:2006}, and consists of a general
review of the literature. Consequently, there is nothing particularly
original, except its presentation.  In spite of that, I believe that
it has some pedagoginal value, and a good collection of references.
Here we follow some conventions of~\cite{Wheeler:2010vmq}, but with a
presentation that is closer to~\cite{Tsilevich:2006}.

%%%%%%%%%%%%%%%%%%%%%%%%%%%%%%%%%%%%%%%%%%%%%%%%%%
%%%%%%%%%%%%%%%%%%%%%%%%%%%%%%%%%%%%%%%%%%%%%%%%%%

\subsection{Phase model}
Consider the \((M+1)\) set of operators \(\{\phi_i,
\phi^\dagger,\mathcal{N}_i\}_{i=0}^M\) such that
\begin{equation}
 [\mathcal{N}_i, \phi_j] = - \phi_i \delta_{i,j} \quad
 [\mathcal{N}_i, \phi_j^\dagger] =  \phi_i^\dagger \delta_{i,j}  \quad 
 [\phi_i, \phi_j^\dagger] =  \pi_i \delta_{i,j}  
\end{equation}
where \(\pi_i =(\ket{0} \bra{0})_i\) is the vacuum projection operator.
These operators can be written as
\begin{equation}
\phi = \sum_{n\geq 0}\ket{n}\bra{n+1}\quad 
\phi^\dagger = \sum_{n\geq 0}\ket{n+1}\bra{n} \quad 
N = \sum_{n\geq 0}n\ket{n}\bra{n}\; ,
\end{equation}
and it is easy to see that \(\phi^\dagger \phi = \bm{1} - \ket{0}
\ket{0}\) and \(\phi\phi^\dagger = \bm{1}\).

The Hamiltonian is given by
\begin{equation}
  H = - \frac{1}{2} \sum_{n =0}^M \left(\phi_n^\dagger \phi_{n+1}
  + \phi_n \phi_{n+1}^\dagger \right) + \bar{\mathcal{N}}\; .
\end{equation}
where \(\bar{\mathcal{N}} = \sum_{i=0}^M \mathcal{N}_i\), and
periodic boundary conditions \(\phi_{M+1} = \phi_0\) and
\(\phi_{M+1}^\dagger = \phi_0^\dagger\).

These operators appear in the context of quantum optics, and for
this reason it is referred as the \emph{phase model}. It 
corresponds to the strongly correlated \(q\)-bosons
model~\cite{Bogoliubov:1997soj} the we will define soon.

\subsubsection{Representation}
Let us define the i\(^{th}\)-vacuum \(\ket{0}_i\) by \(\phi_i\ket{0}_i
= 0\).  The representation of this Hilbert space, denoted by
\(\mathcal{F}_i\), is given by \(\ket{n_i}_i = \phi_i^\dagger
\ket{0}_i\).

Given the vacuum \(\ket{\bm{0}} = \ket{0}_0\otimes \ket{0}_1
\otimes \cdots \otimes  \ket{0}_M\),
the Fock space is defined as 
\begin{equation}
  \mathcal{F} = \bigotimes_{i=0}^M \mathcal{F}_i = 
  \left\{\ket{\vec{n}} = \ket{n_0}\otimes \ket{n_1} \otimes \cdots
  \otimes \ket{n_M} \ | \ n_i \in \mathbb{N} \right\}\; .
\end{equation}
where the states \(\ket{\vec{n}}\) are defined as 
\begin{equation}
  \ket{\vec{n}} = \ket{n_0}\otimes \ket{n_1} \otimes \cdots \otimes \ket{n_M} 
 : =  (\phi_0^\dagger)^{n_0} (\phi_1^\dagger)^{n_1} \cdots  (\phi_M^\dagger)^{n_M} \ket{\bm{0}} \; ,
\end{equation}
where we write \(\phi_i^\dagger \equiv \bm{1} \otimes  \cdots \otimes
\phi_i^\dagger \otimes \cdots \otimes \bm{1}\).
These states are normalized, that is
\(\bracket{\vec{n}}{\vec{m}}=\delta_{\vec{n}, \vec{m}}\). 

Finally, the actions of the operators \(\mathcal{N}_i\) and \(\pi_i\) are
\begin{equation}
    \pi_i\ket{\vec{n}}  = \delta_{n_i, 0} \ket{\vec{n}} \qquad 
    \mathcal{N}_i\ket{\vec{n}} = n_i \ket{\vec{n}}\; .
\end{equation}
 
Given a state \(\ket{\vec{n}} = \ket{n_0, n_1, \dots, n_M}\), we
associate a partition \( \lambda = (0^{n_0} 1^{n_1} 2^{n_2} \cdots
M^{n_M})\). Observe that this correspondence is not unique, since the
partition \(\lambda\) ignores the number \(n_0\) of particles in the
site \(0\). From the total number of particles, we can, on the other hand,
determine the number of particles \(n_0 = N - \ell(\lambda)\), where
\(\ell(\lambda)\) is the number of rows in the Young diagram defined
by \(\lambda\). This is very important, since in our consideations 
below, the \(N\) particle sector is fixed, thanks to the integrability of 
the model, therefore, the number \(n_0\) is known once we define the 
partition \(\lambda\). 

In fact, from the correspondence above, Wheeler~\cite{Wheeler:2010vmq}
defines a map \(\mathcal{M}_\psi: \mathcal{F}\to
\mathcal{F}^{(0)}_\psi\), where \(\mathcal{F}^{(0)}_\psi\) is the
space Fock space of charged free fermions built upon the neutral
(fermionic) vacuum. We will discuss a map from the bosonic space to
the space of symmetric functions.

\subsubsection{Bethe Ansatz}
The \(L\)-matrix is given by 
\begin{equation}
  L_{an} = 
\begin{pmatrix}
x^{ - 1/2} & \phi_n^\dagger \\ \phi_n & x^{1/2}
\end{pmatrix}_a\; , 
\end{equation}
where \(x \in \mathbb{S}^1\). From where we can build the monodromy
matrix
\begin{equation}
  T_a(x) = L_{aM}(x) \cdots L_{a0}(x) = 
\begin{pmatrix}
A(x) & B(x) \\ C(x) & D(x)
\end{pmatrix}_a\; .
\end{equation}

With these expressions, one can finally build the Bethe states
\begin{equation}
  \ket{\Psi(y_1, \dots, y_N)} = \prod_{j=1}^N \mathbb{B}(y_j) \ket{\bm{0}} \qquad 
  \bra{\Psi(y_1, \dots, y_N)} = \bra{\bm{0}}\prod_{j=1}^N \mathbb{C}(y_j) 
\end{equation}
where \(\mathbb{B}(x) = x^{M/2} B(x)\) and \(\mathbb{C}(x) = x^{M/2} C(1/x)\).
When the set \(\{ y_j \ | \ j
=1, \dots , N\}\) satisfies the Bethe equations
\begin{equation}
\label{eq:bethe_eq}
  y^{N + M}_i = (-1)^{N-1} \prod_{\substack{j = 1 \\ j \neq i}}^N y_j\; , \qquad i = 1, \dots, N\; , 
\end{equation}
we says that the Bethe states are \emph{on-sheel}. In what follows, we will 
only consider \emph{off-shell} states. 

%%%%%%%%%%%%%%%%%%%%%%%%%%%%%%%%%%%%%%%%%%%%%%%%%%
%%%%%%%%%%%%%%%%%%%%%%%%%%%%%%%%%%%%%%%%%%%%%%%%%%

\subsection{q-Bosons}
The Hamiltonian for the q-boson model is
given by
\begin{equation}
  \mathcal{H} = -\frac{1}{2} \sum_{n=0}^M
  \left(b_n^\dagger b_{n+1} + b_n b_{n+1}^\dagger \right) + \bar{\mathcal{N}}\; ,
\end{equation}
where \(\bar{\mathcal{N}} = \sum_{i=0}^M \mathcal{N}_i\), and as in
the case of the phase mode, we impose periodic boundary conditions
\(b_{M+1} = b_0\) and \(b_{M+1}^\dagger = b_0^\dagger\). The set of
operators \(\{b_i, b^\dagger,\mathcal{N}_i\}_{i=0}^M\) form the
\(M+1\) copies of independent q-boson algebras:
\begin{equation}
[\mathcal{N}_i, b_j^\dagger]=\delta_{i,j} b_i^\dagger\; , \quad 
[\mathcal{N}_i, b_j]=-\delta_{i,j}b_i\; , \quad
[b_i, b_j^\dagger]= \delta_{i,j} q^{-2\mathcal{N}_i}  \equiv \delta_{i,j} Q^{\mathcal{N}_i}\; , 
\end{equation}
where we write the deformation parameter as \(Q = q^{-2}\). 
In the limit \(Q\to 1\), the q-bosons are ordinary bosons, and 
the limit \(Q\to 0\) (\(q\to \infty\)) gives the phase model we have discussed above. 

\subsubsection{Representation}
We build the representation of the q-boson algebra with some modifications of the phase 
model we have just discussed. Let us define the i\(^{th}\)-vacuum \(\ket{0}_i\) by
\(b_i\ket{0}_i = 0\). The representation of this Hilbert space, denoted by
\(\mathcal{F}_i^Q\), is given by \(\ket{n_i}_i \propto b_i^\dagger \ket{0}_i\).

Given the vacuum \(\ket{\bm{0}} = \ket{0}_0\otimes \ket{0}_1
\otimes \cdots \otimes  \ket{0}_M\),
the Fock space is defined as 
\begin{equation}
  \mathcal{F}_{Q} = \bigotimes_{i=0}^M \mathcal{F}_i^Q = 
  \left\{\ket{\vec{n}} = \ket{n_0}\otimes \ket{n_1} \otimes \cdots
  \otimes \ket{n_M} \ | \ n_i \in \mathbb{N} \right\}\; .
\end{equation}
where the actions of the operators \(\{b_i, b_i^\dagger\}\) are given by 
the following relations
\begin{subequations}
\begin{alignat}{3}
    & b_0\ket{n_0} =(1 - \delta_{0, n_0}) \frac{(1 - Q^{n_0})}{(1 - Q)^{1/2}}\ket{n_0 - 1}
    &\hspace{0.5cm}& b_0^\dagger \ket{n_0} =  \frac{1}{(1 - Q)^{1/2}}\ket{n_0 + 1}  \\
    & b_i\ket{n_i} = \frac{(1 - \delta_{0, n_i})}{(1 - Q)^{1/2}}\ket{n_i - 1}
    && b_i^\dagger \ket{n_0} =  \frac{(1 - Q^{n_i+1})}{(1 - Q)^{1/2}}\ket{n_0 + 1}\quad i\neq 0\; .
\end{alignat}
\end{subequations}
Observe that the modes \(\{b_0, b_0^\dagger \}\) satisfies a slightly
different relation.

Therefore, the states \(\ket{\vec{n}} \in \mathcal{F}_Q\) are defined as 
\begin{equation}
  \ket{\vec{n}} = \ket{n_0}\otimes \ket{n_1} \otimes \cdots \otimes \ket{n_M} 
 : =  (b_0^\dagger)^{n_0} (b_1^\dagger)^{n_1} \cdots  (b_M^\dagger)^{n_M} \ket{\bm{0}} \; ,
\end{equation}
where we write \(\phi_i^\dagger \equiv \bm{1} \otimes  \cdots \otimes
\phi_i^\dagger \otimes \cdots \otimes \bm{1}\).
Moreover, the norm of these states satisfy 
\begin{equation}
  \bracket{\vec{n}}{\vec{m}} = \frac{[m_0]!}{\prod_{i=1}^M[m_i]!} \delta_{\vec{n}\vec{m}}\qquad 
    [n]! =
    \left\{
    \begin{array}{ll}
    \prod_{i=1}^n(1 - Q^i) & \textrm{if}\ \ n\neq 0 \\
    1 & \textrm{otherwise}
  \end{array}\right.\; .
\end{equation}
As before, we can associate to a given states \(\ket{\vec{n}} =
\ket{n_0, n_1, \dots, n_M}\), a partition -- or Young diagram --
\(\lambda = (1^{n_1} 2^{n_2} \cdots M^{n_M})\), but now we assume that 
there is a proportionaly  factor relating these two objects, 
\begin{equation}
  \ket{\lambda}_Q  = b_\lambda(Q) \ket{\vec{n}}  \ \; , \qquad b_\lambda(Q) = \prod_i [p_i(\lambda)]! 
\end{equation}
where \(p_i(\lambda)\) denotes the number of parts of size \(i\) in
the partition. These partition states satisfy \(
\bracket{\lambda}{\mu}_Q = b_\lambda(Q) \delta_{\lambda,\mu}\). Once
again, we see that the correspondence is not unique since the number
of oscilators at the site \(i=0\) are completely ignored in the
partition notation. And finally, if the number of particles is fixed,
then \(n_0 = N - \ell(\lambda)\).

\subsubsection{Bethe Ansatz}
The \(L\)-operator for the q-boson is given by
\begin{equation}
  L_{an}(x, Q) =
  \begin{pmatrix}
    x^{-1/2} & (1 - Q)^{\frac{1}{2}} b_n^\dagger \\ (1 - Q)^{\frac{1}{2}} b_n & x^{1/2}
  \end{pmatrix}_a\; .
\end{equation}
The monodromy matrix is 
\begin{equation}
  T_a(x,Q) = L_{am}(x, Q)  \dots  L_{a0}(x, Q) = 
  \begin{pmatrix}
    A(x, Q) & B(x, Q) \\ C(x, Q) & D(x, Q)
  \end{pmatrix}_a\; .
\end{equation}

The eigenstates of the Hamiltonian are of the form
\begin{equation}
  \ket{\Psi(y_1, \dots, y_N; Q)} = \prod_{j=1}^N \mathbb{B}(y_j, Q) \ket{\bm{0}}\qquad 
  \bra{\Psi(y_1, \dots, y_N; Q)} = \bra{\bm{0}}\prod_{j=1}^N \mathbb{C}(y_j, Q) \; .
\end{equation}
where \(\mathbb{B}(x, Q) = x^{M/2} B(x, Q)\) and \(\mathbb{C}(x, Q) =
x^{M/2} C(1/x, Q)\).  When the set of parameters \(\{ y_j \ | \ j =1, \dots , N\}\)
satisfies the Bethe equations
\begin{equation}
  y^{N + M}_i =\prod_{\substack{j = 1 \\ j \neq i}}^N\frac{Q y_i - y_j}{y_i - Q y_j}\; , \qquad i = 1, \dots, N\; , 
\end{equation}
we say that the states \(\ket{\Psi(y_1, \dots, y_N; Q)}\) are \emph{on-shell} Bethe states, otherwise, 
we have \emph{off-shell} Bethe states. 

%%%%%%%%%%%%%%%%%%%%%%%%%%%%%%%%%%%%%%%%%%%%%%%%%%
%%%%%%%%%%%%%%%%%%%%%%%%%%%%%%%%%%%%%%%%%%%%%%%%%%

%*%*%*%*%*%*%*%*%*%*%*%*%*%*%*%*%*%*%*%*%*%*%*%*%*
\section{Tau functions in the phase model}
%*%*%*%*%*%*%*%*%*%*%*%*%*%*%*%*%*%*%*%*%*%*%*%*%*

This section discusses the presence of integrable hierarchies in the phase
model.  We argue that it is also a Toda hierachy tau function, and we
discuss some consequences of this fact, in particular, its connection
with a matrix model. Moreover, we also point the presence of
correlation functions in this model that also satisfy the KP hierarchy
equations. We finally argue how these results also imply that these
scalar products are also KP hierarchy tau functions, what is expected from 
the results of Wheeler~\cite{Wheeler:2010vmq}.

Bogoliubov~\cite{Bogoliubov2005}, has shown that the scalar product of
two vectors in the \(N\)-particles sector in a chain of length \(M+1\)
is
\begin{equation}
\begin{split}
\label{eq:scalar}
  \mathcal{I}(N, M|\bm{x}, \bm{y}) & =
  \bra{\bm{0}} \prod_{j=1}^N \mathbb{C}(x_j) \prod_{j=1}^N \mathbb{B}(y_j) \ket{\bm{0}} \\ 
 & = \frac{ \det H(\bm{x},\bm{y})}{ \prod_{i<j}(x_i - x_j)(y_i - y_j)} \; ,
\end{split}
\end{equation}
where \(H\) is an \(N\times N\) matrix with components
\begin{equation}
\label{eq:h-matrix}
  H_{ij} = H(x_i, y_j) 
  =\frac{1 - (x_i y_j)^{ M + N}}{1 - x_i y_j }\; .
\end{equation}

%%%%%%%%%%%%%%%%%%%%%%%%%%%%%%%%%%%%%%%%%%%%%%%%%%
%%%%%%%%%%%%%%%%%%%%%%%%%%%%%%%%%%%%%%%%%%%%%%%%%%

\subsection{Toda tau functions}
In this section, we argue that the scalar product defined above is a tau function 
of the Toda hierarchy. Let us first write the function \(H(z,w)\) as the geometric sum 
\begin{equation}
\label{eq:h-exp}
  H(z,w) = \frac{1 - (zw)^{M+N}}{1 - zw} = \sum_{k=1}^{M+N} (zw)^{k-1} \; , 
\end{equation}
in such a way that 
we can write
\begin{equation}
  \det_{i,j} \left(H(x_i, y_j)\right) = \det_{i,j} \left( \sum_{k=1}^{M+N} x_i^{k-1} y_j^{k-1}\right)
\end{equation}
and we can thing of this expression as the product of a \(N\times (N+M)\) matrix \(\mathcal{X}\)
and another \((M + N)\times N\) matrix \(\mathcal{Y}\), given by
\begin{equation}
  \mathcal{X} = 
  \begin{pmatrix}
  x_1^0 & x_1^1 & \dots & x_1^{M+N-1} \\  
  x_2^0 & x_2^1 & \dots & x_2^{M+N-1} \\  
  \vdots \\
  x_N^0 & x_N^1 & \dots & x_N^{M+N-1} 
  \end{pmatrix}\quad \textrm{and} \quad 
  \mathcal{Y} = 
  \begin{pmatrix}
  y_1^0 & y_2^0 & \dots & y_N^0 \\  
  y_1^1 & y_2^1 & \dots & y_N^1 \\  
  \vdots & \vdots & & \vdots \\
  y_1^{N+M-1} & y_2^{M+N-1} & \dots & y_N^{M+N-1} 
  \end{pmatrix}
\end{equation}
therefore 
\begin{equation}
  \det_{i,j} \left(H(x_i, y_j)\right) =  \det_{i,j} \left( \mathcal{X}\mathcal{Y}\right)
= \sum_{0 \leq \ell_{N} \leq \dots \leq \ell_1\leq N+M }  \det_{ik}(x_i^{\ell_k})  \det_{ik}(y_j^{\ell_k})\; .
\end{equation}
If we now define \(\ell_j = \lambda_k - k + N\), and
using~(\ref{eq:scalar}), we have
\begin{equation}
\label{eq:scalar_exp}
\mathcal{I}(N, M|\bm{x}, \bm{y}) = \sum_{\lambda\subseteq [N,M]}
\frac{\det_{ik}(x_i^{\lambda_k - k + N})}{\Delta(\bm{x})} \frac{ \det_{ik}(y_j^{\lambda_k - k + N})}{\Delta(\bm{y})}
= \sum_{\lambda\subseteq [N,M]} s_\lambda(\bm{x}) s_\lambda(\bm{y}) \; .
\end{equation}
where \(\Delta(\bm{x})\) and \(\Delta(\bm{x})\) are Vandermonde
determinants. This formula agrees with the Schur expansion defined
in~\cite{Bogoliubov2005}.

Let us define two sets of Miwa coordinates \(\bm{t} = (t_1, t_2, \dots)\)
and \(\bm{t}' = (t'_{-1}, t'_{-2}, \dots)\) by
\begin{equation}
  t_p = \frac{1}{p}\sum_{j=1}^N x_j^p\qquad 
  t'_{-p} = \frac{1}{p}\sum_{j=1}^N y_j^p\; ,
\end{equation}
therefore, we have 
\begin{equation}
  \mathcal{I}(N, M|\bm{t}, \bm{t}') = \sum_{\lambda\subseteq [N,M]} s_\lambda(\bm{t}) s_\lambda(\bm{t}') \; .
\end{equation}
This expression is known to be a tau function for \(M, N\to \infty\).
In fact, using the free fermions representation~\cite{Alexandrov:2012tr},
it can be written as
\begin{equation}
\sum_{\lambda } s_{\lambda}(\bm{t}) s_{\lambda}(\bm{t}')
 = \bra{\bm{0}} e^{\bm{J}_+(\bm{t})} e^{-\bm{J}_-(\bm{t}')} \ket{\bm{0}}\; .
\end{equation}
It is a tau function of the Toda hierarchy with trivial element \(1 \in GL(\infty)\).
In fact, it is nothing but the Cauchy's identity
\begin{equation}
  \sum_{\lambda } s_{\lambda}(\bm{t}) s_{\lambda}(\bm{t}')
    = \exp \left( \sum_{m\geq1} m t_m t'_{-m} \right) \; .
\end{equation}
where \(\bm{t}\) and \(\bm{t}'\) are two independent sets of Miwa coordinates.

Putting all these facts together, the truncation for finite \(M\) and
\(N\) are also tau functions of the Toda hierarchy. More specifically,
according to~\cite{Alexandrov:2012tr, Kharchev:1991gd,
  Zabrodin:2010ii}, the truncation of the tau function corresponds to
the inclusion of a projection operator \(\mathrm{P}^+\) in the
expectation value of the tau function written in the fermionic
representation.

As a final remark, we can also write
\begin{equation}
  H(z,w) = \sum_{k=0}^{M+N} h_k(zw)^{k-1} \; , 
\end{equation}
with \(h_k=1\) if \(k\in [0, M+N-1]\) and \(0\) otherwise. In this
case, one can define a diagonal \((N+M)\times (N+M)\) matrix
\(\mathcal{H} = \textrm{diag}(h_0, \dots, h_{M+N-1})\). Therefore, if
we repeat the arguments above, we find
\begin{equation}
\mathcal{I}(N, M|\bm{t}, \bm{t}') = \sum_{\lambda} h_\lambda s_\lambda(\bm{t}) s_\lambda(\bm{t}') \; .
\end{equation}
where \(h_\lambda = 1\) if \(\lambda \subset [N,M]\) and it vanishes
otherwise.  In this case, we have that this tau function is of the
type considered in~\cite{orlov:2001}. Evidently that it is a trivial
example, but we expect that in the case of the \(q\)-Boson model,
where \(q\) is finite, the diagonal terms \(h_\lambda\) are more
interesting. We will discuss this case next section.

\subsubsection{Matrix Models}

It is also interesting to notice that the particular case when \(M\to
\infty\), but with a finite number of particles \(N\). The
partitions \(\lambda \in [N, \infty]\) satisfy the condition
\(\ell(\lambda) \leq N\). Therefore, the scalar
product~(\ref{eq:scalar_exp}) becomes
\begin{equation}
  \mathcal{I}_N(\bm{t}, \bm{t}') = \sum_{\substack{\lambda \\ \ell(\lambda) \leq N}}
  s_\lambda(\bm{t})  s_\lambda(\bm{t}') \; .
\end{equation}
From~\cite{Zabrodin:2010ii}, we know that this expression can be
written as the following integral
\begin{equation}
  \mathcal{I}_N(\bm{t}, \bm{t}') =
  \frac{1}{N!} \prod_{\ell=1}^N \oint_{\Gamma_\ell} \frac{dz_\ell}{2 \pi i z_\ell}
  e^{\xi(\bm{t}, z_\ell) - \xi(\bm{t}', z_{\ell}^{-1})} \Delta(z)\Delta(z^{-1})\; ,
\end{equation}
where
\begin{equation}
\xi(\bm{t}, z) = \sum_{k\geq 0} t_k z^k \qquad
\xi(\bm{t}', 1/z) = \sum_{k\geq 0} t'_{-k} z^{-k} \; . 
\end{equation}
See also~\cite{Kharchev:1991gd} for a detailed proof of this relation,
and~\cite{Orlov:2005} for other details. 

What is very interesting about this result is what it implies when we 
impose the Bethe equations~(\ref{eq:bethe_eq}). Suppose that 
\(x_j = e^{-ip_j} \in \mathbb{S}^1\), and we set \(\bm{t}' = - \bm{t}^\star\).
In this particular case, we have
\begin{equation}
\xi(\bm{t}, z) - \xi(\bm{t}', 1/z)  = 2 \ \textrm{Re}\left(\sum_k t_k z^k\right)\; .
\end{equation}
Then, the phase model is equivalent to an ensemble of \(N\) 2D Coulomb
particles on a circle. In this case, we have that the quantities
\(\lambda_{\ell}\) are eigenvalues of a matrix \(U\).

Moreover, Zabrodin~\cite{Zabrodin:2010ii}, see citations therein, we also know
that under the rescaling \(t_k \to T_k/ \hbar\), \(t'_k \to T_{-k}/ \hbar\)
and \(N = T_0/ \hbar\), we get the dispersion limit tau function 
\begin{equation}
  F_0(\bm{T}) = \log S_{T_0}(\bm{T}, \bm{T}') + \mathcal{O}(\hbar)\; ,
\end{equation}
a free energy, from the viewpoit of the matrix integral partition
function.  It is not clear how one can use this fact to determine
properties of the integrable model, but it might be possible to study
the analytic structure of the free energy \(F_0\) to gain some
understanding on the Bether roots \(\bm{x}\).  This problem is
currently under further investigation, and we hope to report some new
results elsewhere.

%%%%%%%%%%%%%%%%%%%%%%%%%%%%%%%%%%%%%%%%%%%%%%%%%%
%%%%%%%%%%%%%%%%%%%%%%%%%%%%%%%%%%%%%%%%%%%%%%%%%%

\subsubsection{KP tau function}
Finally, one may also observe that if we fix one set of coordinates,
say \(\bm{y}\), then we write
\begin{equation}
 \mathcal{I}(N,M|\bm{t}) 
 = \sum_{\lambda} s_\lambda(\bm{y})  s_\lambda(\bm{t}) 
 \equiv \sum_{\lambda} c_\lambda(\bm{y})  s_\lambda(\bm{t}) \; ,
\end{equation}
with coefficients \(c_\lambda(\bm{y}) = \det (h_{\lambda_i-i +j}(\bm{y}))\).
But now, it is a trivial to notice that these are Plücker
coordinates in the Jacobi-Trudi form, see~\cite{Miwa2000, Alexandrov:2012tr}. 

Therefore, the expression
\begin{equation}
 \mathcal{I}(N,M|\bm{t}) = \sum_{\lambda \subseteq [N,M]} s_\lambda(\bm{y})  s_\lambda(\bm{t}) \; ,
\end{equation}
is also a KP tau function: A fact that we already know
from~\cite{Wheeler:2010vmq}, where the author has proved this
statement using the free fermions formalism.


%%%%%%%%%%%%%%%%%%%%%%%%%%%%%%%%%%%%%%%%%%%%%%%%%%
%%%%%%%%%%%%%%%%%%%%%%%%%%%%%%%%%%%%%%%%%%%%%%%%%%

\subsection{Correlation functions}
In~\cite{Bogoliubov2005}, Bogoliubov has also shown that the
correlation functions
\begin{subequations}
\begin{equation}
\begin{split}
\label{eq:correlation}
  A_m(N, M|\bm{x}, \bm{y}\setminus \{y_N\})
  & = \bra{0} \prod_{j=1}^N \mathbb{C}(x_j)
  \prod_{k=1}^{N-1} \mathbb{B}(y_k) \phi_m^\dagger \ket{0}\\
  & =  \prod_{j=1}^N x_{j}^{M/2} \prod_{k=1}^{N-1} y_{j}^{M/2}
  \bra{0} \prod_{j=1}^N C(1/x_j) \prod_{k=1}^{N-1} B(y_k) \phi_m^\dagger \ket{0}
\end{split}
\end{equation}
can be written as
\begin{equation}
  A_m(N, M|\bm{x}, \bm{y}\setminus \{y_N\}) = 
  \frac{(-1)^{N-1}}{y_N^{(N-1)/2}} \prod_{j=1}^N x_{j}^{M/2}
  \prod_{k=1}^{N-1} y_{j}^{M/2}
  \left( \prod_{t<N} \frac{y_N - y_t}{y_t} \right) \frac{\det Q}{\det H} S(N,M|\bm{x}, \bm{y})
\end{equation}
\end{subequations}
where the \(N\times N\) matrix \(Q\) has components 
\begin{equation}
 Q_{jN} = x_j^{(M + N - 1- 2m)/2} \quad  \textrm{and} \quad 
 Q_{jk} = H_{jk} \; , 
\end{equation}
where \(H_{jk} = H(x_j, y_k)\) are the components of the matrix \(H\) in~(\ref{eq:h-matrix}). 
Since the components \(Q_{jN}\) are independent of the coordinates \(y\), 
we cannot write the above expression as a Toda hierarchy tau
function. 

We already know from~(\ref{eq:scalar}) that 
\begin{equation}
    \frac{\mathcal{I}(N,M|\bm{x}, \bm{y})}{\det H}=  \frac{1}{\Delta(x)\Delta(y)} \; ,
\end{equation}
therefore we have
\begin{equation}
  A_m(N, M|\bm{x}, \bm{y}\setminus{y_N})  \propto
  \frac{\det Q}{\Delta(x)} \equiv \mathcal{A}_m(N,M|\bm{x}, \bm{y})\; ,
\end{equation}
and we are going to treat the coordinates \(\{ \bm{y} \}\) as
a set of \(N\) fixed parameters. Furthermore, we write \(N\)
functions \(\bm{F}(z) = (F_1, \dots, F_N)\) defined as
\begin{equation}
  F_j (z) = H(z, y_j) \quad \textrm{if} \ j \neq N\; , \qquad \textrm{and}\qquad 
   F_N (z)  = z^{(M + N - 1 - 2m)/2} \; .
\end{equation}
As before, we expand \(F_j(z)\) as the geometric sum
\begin{equation}
  F_j(z) = \frac{1 - (y_j z)^{M + N}}{1 - y_j z} = \sum_{n=0}^{N + M - 1} (y_j z)^n
  \equiv \sum_{n=0}^{M + N -1} f_{j, n} z^n\; , 
\end{equation}
where we will also assume the \(M + N - 1 \in 2 \mathbb{Z}\).

With these definitions, we conclude that
\begin{equation}
\mathcal{A}_m(N,M|\bm{x}, \bm{y}) =\frac{\det_{jk} F_j(x_k)}{\Delta(x)}\; .
\end{equation}
From the expansion 
\begin{equation}
  \det_{jk} F_j(x_k) = \det_{jk} \left(  \sum_{n=0}^{M + N -1} f_{j, n} x_k^n \right) \; .
\end{equation}
and with the aid of the Cauchy-Binet formula, we write  
\begin{equation}
    \mathcal{A}_m(N,M|\bm{x}, \bm{y})
  = \sum_{0\leq \ell_N\leq \dots \leq \ell_1 \leq N+M}
  \frac{\det_{jk}(f_{j, \ell_k}) \det_{jk}(x_k^{\ell_j})}{\Delta(x)}\; . 
\end{equation}
Now we basically use the definition of the Schur polynomials and the
Jacobi-Trudi expression of Plücker coordinates, see~\cite{Alexandrov:2012tr}. 
Finally, this last expression becomes
\begin{equation}
\mathcal{A}_m(N,M|\bm{x}, \bm{y}) =
\sum_{\lambda} c_\lambda(\bm{y}) s_\lambda(\bm{x}) \; ,
\end{equation}
where \(c_\lambda \equiv \det_{jk}(y_j^{\ell_k})\),  \(\ell_k = \lambda_k - k +N\). Therefore,
we have that this expression is a KP tau function. 

From the coordinate expansion of the Bethe
vectors~\cite{Bogoliubov2005, Tsilevich:2006}, it is easy to show that
\begin{equation}
\prod_{j=1}^N \mathbb{B}(x_j)\ket{\lambda}  = \sum_{\mu \supset \lambda } s_{\mu/\lambda}(\bm{x})\ket{\mu}\qquad 
 \bra{\lambda} \prod_{j=1}^N \mathbb{C}(x_j) = \sum_{\mu \supset \lambda } s_{\mu/\lambda}(\bm{x})\bra{\mu}
\end{equation}
where \(s_{\mu/\lambda}\) are Schur polynomials~\cite{Macdonald:1998}.
Therefore, we can write the correlation function~(\ref{eq:correlation}) as 
\begin{equation}
\begin{split}
  A_m(N, M|\bm{x}, \bm{y}\setminus \{y_N\})
  & = \sum_\lambda \sum_{\mu \supset (m)} s_{\lambda}(\bm{x}) s_{\mu/(m)}(\bm{y}\setminus\{y_N\})\\
  & = \sum_\mu s_{\mu/(m)}(\bm{y}\setminus\{y_N\}) s_{\mu}(\bm{x})\; .
\end{split}
\end{equation}
where we have used that \(\phi^\dagger \ket{\bm{0}} = \ket{(m)}\), and in the second line
we sum over all partitions, since \(s_{\mu/(m)} = 0\), \(\forall \) \(\mu \not \supset (m)\).
This expression also shows that the expression \(P_m\) can also be written as a tau function. 

\subsubsection{Skew Schur polynomials expansion}
More generally, let us consider the correlation functions 
\begin{equation}
\begin{split}
  A_{\lambda_1 \lambda_2}(N', N, M|\bm{x}, \bm{y}) & =
  \bra{\lambda_1} \prod_{j=1}^N \mathbb{C}(x_j)
  \prod_{k=1}^{N'} \mathbb{B}(y_k)\ket{\lambda_2} \\
  & = \sum_{\mu} s_{\mu/\lambda_1}(\bm{x}) s_{\mu/\lambda_2}(\bm{y}) \; .
\end{split}
\end{equation}
where \(N' + \ell(\lambda_2) + n_0 = N' + \ell(\lambda_2) + n'_0\) and 
we have also used that the expression for any Young diagram
\(\mu\) that does not contain \(\lambda_1\) and \(\lambda_2\)
vanishes. Using the elementary property of skew Schur
polynomials~\cite{Macdonald:1998}
\begin{equation}
  \sum_{\mu} s_{\mu/\lambda_1}(\bm{x}) s_{\mu/\lambda_2}(\bm{y}) = \prod_{i,j}\frac{1}{1 - x_i y_j}
 \sum_\nu s_{\lambda_1/\nu}(\bm{x}) s_{\lambda_2/\nu}(\bm{y})\; ,
\end{equation}
and the Cauchy's identity, we have 
\begin{equation}
\begin{split}
  A_{\lambda_1 \lambda_2}(N', N, M|\bm{x}, \bm{y}) & =
 \sum_\mu s_{\mu}(\bm{x}) s_{\mu}(\bm{y})
 \sum_\nu s_{\lambda_1/\nu}(\bm{x}) s_{\lambda_2/\nu}(\bm{y})\; .
\end{split}
\end{equation}
Therefore, this correlation function is nothing but the off-shell
norm~(\ref{eq:scalar}) times a finite sum over skew Schur functions.
This expression shows that the existence of these tau functions is not
a trivial fact of this model.

%%%%%%%%%%%%%%%%%%%%%%%%%%%%%%%%%%%%%%%%%%%%%%%%%%
%%%%%%%%%%%%%%%%%%%%%%%%%%%%%%%%%%%%%%%%%%%%%%%%%%

\subsection{More tau functions}
Based on the discussion we had so far, and on the general map between 
the Phase model and the free fermions, one can understand the general form 
of tau functions in this phase model context. 

Let us consider the vertex operator construction~\cite{Okounkov2001},
see also~\cite{Alexandrov:2012tr, Wheeler:2010vmq}. Consider the
vacuum \(\ket{\bm{0}}\), a ``Dirac sea'', defined by the conditions \(\psi_m\ket{\bm{0}}
= \psi_n^\star\ket{\bm{0}} =0 \) for \(m<0\) and \(n \geq 0\), where 
\(\psi_n\) are components of a holomorphic free fermionic field. 
In this formalism, the partition states are given by 
\begin{equation}
  \ket{\mu} = \textrm{sign}(\sigma) \prod_{j=1}^d \psi_{a_j} \psi^\star_{-b_j}\ket{\bm{0}}\; ,
\end{equation}
where \(\textrm{sign}(\sigma) = \pm 1\) are defined in a such a way
that the Shur coefficients of the vertex operators, defined below,
have positive coefficients.

The pairs \((a_j|b_j)_{j=1}^d\) define the Frobenius notation of the
partition \(\mu = (\mu_1, \mu_2, \dots, \mu_\ell)\), that is given by
\(a_j = \mu_j - j\) and \(b_j = \mu'_j - j\), where \(d\) is the
number of boxes in the diagonal of the Young diagram, and \(\mu'\) is
its conjugate, or transpose.  From these definitions, we have the
equivalence
\begin{equation}
  \prod_{k\geq 1} (\phi_k^\dagger)^{n_k} \mapsto \textrm{sign}(\sigma) \prod_{j=1}^d
  \psi_{a_j} \psi^\star_{-b_j} \qquad |\mu| = \sum_k k n_k = \sum_j(a_j + b_j) + d\; .
\end{equation}

Finally, the \(\mathfrak{gl}(\infty)\) has generators given by the
bilinears \(X = \sum_{j, j \in \mathbb{Z}} x_{ij} \bm{\colon} \psi_i \psi_j^\star\bm{\colon}
+ c\), where \(c\in \mathbb{C}\), \(x_{ij} =
0\) for large \(|j -i|\), say \(\geq M\), and the colons denote the normal ordering
\begin{equation}
  \bm{\colon} \psi_i \psi_j^\ast \bm{\colon} =  \psi_i \psi_j^\ast
  - \bra{0} \psi_i \psi_j^\ast \ket{0} \; .
\end{equation}
The group \(GL(\infty)\) is defined through the \(\exp:
\mathfrak{gl}(\infty) \to GL(\infty)\) as usual.

Observe that these elements have finitely many non zero elements, and
that these include diagonal terms \(\bm{\colon} \psi_i
\psi_i^\star\bm{\colon}\), upper and lower triangular terms. These are
equivalent to the Number operators \(\mathcal{N}\), annihilation
\(\phi\) and creation \(\phi^\dagger\) , respectivelly.  The central
charge is equivalent to the vaccum projection \(\pi =
\ket{\bm{0}}\bra{\bm{0}}\). Therefore, the elements
\begin{equation}
   GL(\infty) \ni G \mapsto
   \mathcal{G} = \exp \left[\sum_{i=1}^M \left( \sum_{a=1}^3
   c_{i, a} T_i^{a}  + c \pi_i \right)\right]\; ,
\end{equation}
where \(T^{a}_i \in \{ \mathcal{N}_i, \phi_i, \phi_i^\dagger \}_{i=1}^M\). 

The operators \(\mathbb{B}(x)\) and \(\mathbb{C}(x)\), 
for a large enough chain \(M\to \infty\),
are related to the vertex operators
\(\Gamma_-(x)\) and \(\Gamma_+(x)\), respectivelly, as
\begin{equation}
  \begin{split}
    \mathbb{B}(x) \mapsto \Gamma_-(x) & = \exp \left( \sum_{n\geq 1} \frac{1}{n}x^n J_{-n}\right) \\
    \mathbb{C}(x) \mapsto \Gamma_+(x) & = \exp \left( \sum_{n\geq 1} \frac{1}{n}x^n J_{n}\right) \; ,
  \end{split}
\end{equation}
where \(J_n \) is written, in terms of free fermions, as \(J_n =
\sum_{j\in \mathbb{Z}} \bm{\colon} \psi_j \psi_{j+n}^\ast
\bm{\colon}\),
These set of operators generate a Heisenberg subalgebra
\(\widehat{\mathfrak{gl}}(1) \subset \mathfrak{gl}(\infty)\) given by
\begin{equation}
  [J_m, J_n] = m \delta_{n+m,0}\; .
\end{equation}

Putting all these facts together, we have that the tau functions of the Toda 
hierarchies, given by
\begin{equation}
  \tau_s(\bm{x}, \bm{y}) = \bra{s} \prod_{i} \Gamma_+(x_i) G \prod_j \Gamma_-(x_j) \ket{s}\; ,
\end{equation}
are mapped into objects of the form 
\begin{equation}
\begin{split}
  \tau_0(\bm{x}, \bm{y}) & = \bra{\Psi(\bm{x})} \mathcal{G} \ket{\Psi(\bm{y})} \\
  \\ & = \bra{\bm{0}} \prod_{i} \mathbb{C}(x_i)
  \mathcal{G} \prod_j \mathbb{B}(y_j) \ket{\bm{0}}\; ,
\end{split}
\end{equation}
where we necessarily have \(s=0\) in the phase model


\subsubsection{Schur polynomials expansion of some correlation functions}
In fact, it is worth noticing that other quantities, that are not tau
functions, might have interesting Schur polynomials expansion.
Consider the state calculated in~\cite{Bogoliubov2005}
\begin{equation}
  \ket{\mathcal{Y}} =
 \sum_{\vec{n}} \prod_{j=0}^M \ket{n_j} = \sum_\mu \ket{\mu} \qquad \sum_j n_j = N\; .
\end{equation}
Then
\begin{equation}
 \bra{\mathcal{Y}} \prod_{j=1}^N B(\bm{x}) \ket{\nu} = 
  \sum_{\lambda\subseteq (N,N,M)} s_{\lambda/\nu}(\bm{x})\; . 
\end{equation}
And it is just a sum of skew Schur polynomials.

%%%%%%%%%%%%%%%%%%%%%%%%%%%%%%%%%%%%%%%%%%%%%%%%%%
%%%%%%%%%%%%%%%%%%%%%%%%%%%%%%%%%%%%%%%%%%%%%%%%%%

%*%*%*%*%*%*%*%*%*%*%*%*%*%*%*%*%*%*%*%*%*%*%*%*%*
\section{Tau functions in the Q-bosons model}
%*%*%*%*%*%*%*%*%*%*%*%*%*%*%*%*%*%*%*%*%*%*%*%*%*

We now move out attention to the case of the q-bosons. The 
analysis is parallel to what we did before, but the  specific 
details are very different. We start with the study of the norm of  
two off-shell Bethe states. 

It has been shown~\cite{Tsilevich:2006} conf. \cite{Sulkowski:2008mx,
  Wheeler:2010vmq}, that the Bethe states \(\ket{\Psi(\bm{x})}\) have 
coordiante expansion given by
\begin{equation}
  \ket{\psi(\bm{x})} = \sum_{\mu \subset [N,M]} P_\mu(\bm{x}, Q) \ket{\mu}  \qquad 
  \bra{\psi(\bm{x})} = \sum_{\mu \subset [N,M]} P_\mu(\bm{x}, Q) \bra{\mu}\; ,
\end{equation}
where \(P_\lambda\) denote the Hall-Littlewood polynomials.  In this
form, scalar product of two off-shell Bethe states in the q-boson
model is easily calculated to be
\begin{equation}
\label{eq:scalar-hl}
\mathcal{I}_Q(N,M | \bm{x}, \bm{y}) = \bra{\bm{0}} \prod_{j=1}^N C(x_j, Q)
\prod_{k=1}^N B(y_k, Q) \ket{\bm{0}}
= \sum_{\lambda \subseteq
  (N,N,M)} b_\lambda(Q) P_{\lambda}(\bm{x}, Q) P_{\lambda}(\bm{y}, Q)
\end{equation}
where we use that \(\bracket{\lambda}{\mu} = b_\lambda(Q)
\delta_{\lambda, \mu}\).  It turns out that this expansion is also
related to the Cauchy identity for Hall-Littlewood
polynomials~\cite{Macdonald:1998}
\begin{equation}
\label{eq:cauchy_hl}
\sum_{\lambda} b_\lambda(Q) P_{\lambda}(\bm{x}, Q) P_{\lambda}(\bm{y}, Q)
= \prod_{j, k=1}^\infty \frac{1-Q x_j y_k}{1 - x_j y_k}\; ,
\end{equation}

Ourt goal is to consider some elementary properties of Hall-Littlewood polynomials 
to understand aspects of this expansion. 

%%%%%%%%%%%%%%%%%%%%%%%%%%%%%%%%%%%%%%%%%%%%%%%%%%
%%%%%%%%%%%%%%%%%%%%%%%%%%%%%%%%%%%%%%%%%%%%%%%%%%

\subsection{Determinant formula}

We would like to refine the expressions above.  Let us first consider
a determinant expression for the scalar product~(\ref{eq:scalar-hl}).
Let us continue with the case where \(M\) and \(N\) are very large,
but finite. In this case, we write this scalar product as
\begin{equation}
  \mathcal{I}_Q(N,M | \bm{x}, \bm{y})  
= \prod_{j_1, j_2} (1-Q x_{j_1} y_{j_2}) \prod_{k_2, k_2}(1 - x_{k_1} y_{k_2})^{-1}\; .
\end{equation}

Using the Cauchy's identity for Schur polynomials, that is 
\begin{equation}
  \sum_\lambda s_\lambda(\bm{x}) s_\lambda(\bm{y}) = \prod_{i,j} \frac{1}{1 - x_i y_j}\; ,
\end{equation}
and using the results we have derives in the Phase model, we have that 
\begin{equation}
  \prod_{i,j}\frac{1}{1 - x_i y_j}  \equiv \mathcal{I}(N,M|\bm{x}, \bm{y}) = 
  \frac{\det_{j,k}H(\bm{x},\bm{y})}{\Delta(\bm{x}) \Delta(\bm{y})}\; .
\end{equation}
where \(H\) is the matrix~(\ref{eq:h-matrix}). Therefore
\begin{equation}
  \mathcal{I}_Q(N,M | \bm{x}, \bm{y})  
= \frac{\mathcal{I}(N,M|\bm{x}, \bm{y})}{\mathcal{I}(N,M|\bm{x}, Q\bm{y})}\; , 
\end{equation}
and we see that the scalar product of the q-boson model is the quotient of 
scalar products of the phase model. Observe that in the case \(Q = 0\), we have 
\(\mathcal{I}(N, M| \bm{x}, \bm{0}) = 1\), therefore
\(\mathcal{I}_0(N,M | \bm{x}, \bm{y}) = \mathcal{I}(N,M | \bm{x}, \bm{y})\), 
as expected. 

Moreover
\begin{equation}
  \mathcal{I}_Q(N,M | \bm{x}, \bm{y})  
  = \frac{\Delta(Q\bm{y})}{\Delta(\bm{y})}
  \frac{\det H(\bm{x}, \bm{y})}{\det H(\bm{x},Q \bm{y})} \; .
\end{equation}
Using that \(\Delta(Q\bm{y})/ \Delta(\bm{y}) = \prod_{j=1}^{N-1} Q^j\), we have 
\begin{equation}
  \mathcal{I}_Q(N,M | \bm{x}, \bm{y}) =  Q^{N(N-1)/2}\;
  \frac{\det H(\bm{x}, \bm{y})}{\det H(\bm{x}, Q \bm{y})} \; .
\end{equation}
Let us denote \(H(\bm{x}, Q\bm{y}) = H_{Q}\), therefore
\begin{equation}
  \mathcal{I}_Q(N,M | \bm{x}, \bm{y}) = Q^{N(N-1)/2} \det H_{Q}^{-1} \det H
= Q^{N(N-1)/2} \det \mathcal{H}(\bm{x}, \bm{y}) \; ,
\end{equation}
where \(\mathcal{H} = H_{Q}^{-1} H\) is a matrix with components 
\(\mathcal{H}_{i,j} \equiv \mathcal{H}(x_i, y_j)\) where
\begin{equation}
  \mathcal{H}(z, w) = \sum_{k} \frac{(1 - z y_k)}{(1 - Q z y_k)} 
\frac{(1 - Q x_k w)}{(1 - x_k w)}\; .
\end{equation}
Observe that from this expression, we cannot decompose this function
as in~(\ref{eq:h-exp}) since the coefficients in this expansion are 
are dependent on \(\bm{x}\) and \(\bm{y}\).

%%%%%%%%%%%%%%%%%%%%%%%%%%%%%%%%%%%%%%%%%%%%%%%%%%
%%%%%%%%%%%%%%%%%%%%%%%%%%%%%%%%%%%%%%%%%%%%%%%%%%

\subsection{Scalar product: Big Schur functions}

Let us now recover the result originally derived
in~\cite{Foda:2008hn} that shows how the scalar
product~(\ref{eq:scalar-hl}) is a tau function of the KP hierarchy.
From~\cite[see chap.3, sec. 4, eq. (4.7)]{Macdonald:1998}, we can
expand the Cauchy identity~(\ref{eq:cauchy_hl}) as
\begin{equation}
 \prod_{j, k=1}^\infty \frac{1-Q x_j y_k}{1 - x_j y_k} = 
\sum_{\lambda} \mathcal{S}_{\lambda}(\bm{x}, Q) s_{\lambda}(\bm{y}) =
\sum_{\lambda} \mathcal{S}_{\lambda}(\bm{y}, Q) s_{\lambda}(\bm{x}) \; ,
\end{equation}
where the polynomials \(\mathcal{S}_\lambda\), the big-Schur
functions, are given by a Jacobi-Trudi formula
\begin{equation}
  \mathcal{S}_{\lambda} (\bm{y}, Q) = \det(q_{\lambda_i -i + j}(\bm{y}, Q))\; , 
\end{equation}
and the coefficients \(q_m\) can be obtained from the following expression
\begin{equation}
 \sum_{m} q_m(\bm{y}, Q) z^m =
 \prod_{j} \frac{1-Q y_j z}{1 - y_j z}
\end{equation}
where \(z\) is a formal variable. It has been
argued~\cite{Foda:2008hn} in that if we interpret the big-Schur
functions as Plücker coordinates, the expression
\begin{equation}
  \mathcal{I}_Q(N,M | \bm{x}, \bm{y})
  = \sum_{\lambda \subseteq (N,N,M)} \mathcal{S}_{\lambda}(\bm{x}, Q) s_{\lambda}(\bm{y})
  = \sum_{\lambda \subseteq (N,N,M)} \mathcal{S}_{\lambda}(\bm{y}, Q) s_{\lambda}(\bm{x})\; ,
\end{equation}
is a restricted KP tau function with respect to both set of coordinates,
that is \(\bm{x}\) and \(\bm{y}\).

%%%%%%%%%%%%%%%%%%%%%%%%%%%%%%%%%%%%%%%%%%%%%%%%%%
%%%%%%%%%%%%%%%%%%%%%%%%%%%%%%%%%%%%%%%%%%%%%%%%%%

\subsection{Kostka-Foulkes expansion}
Let us expand these polynomials in a Schur polynomials basis, that is
\begin{equation}
P_\lambda(\bm{x}, Q) = \sum_{\mu} K^{-1}_{\lambda \mu}(Q) s_\lambda(\bm{x})\; , 
\end{equation}
where \(K^{-1}_{\mu\nu}(Q) \in \mathbb{Z}[Q]\) are inverse
Kostka-Foulkes polynomials~\cite{Macdonald:1998, Wheeler:2018}.

Expanding now the big-Schur functions as
\begin{equation}
\label{eq:kf-exp}
  \mathcal{S}_\lambda(\bm{x}, Q) = \sum_{\mu} c_{\lambda\mu}(Q) s_\mu(\bm{x})\; ,
\end{equation}
we can fix the coefficients \(c_{\lambda\mu}\) in terms of Kostka-Foulkes
polynomials using the orthogonality relations of these polynomials~\cite{Macdonald:1998}. 
In particular, there is an inner product in the ring of symmetric functions such that
\begin{equation}
  \frac{1}{b_\mu}\langle P_\mu(\bm{x}, Q), P_\mu(\bm{x}, Q)\rangle = \langle
  \mathcal{S}_\mu(\bm{x}, Q), s_\mu(\bm{x})\rangle = \delta_{\mu\nu}\; .
\end{equation}
Therefore,
\begin{equation}
  \begin{split}
\delta_{\mu\nu} & = \sum_\lambda c_{\mu\lambda}  \langle s_\lambda(\bm{x}), s_\mu(\bm{x})\rangle\\
& = \sum_\lambda \sum_{\pi_1, \pi_2} c_{\mu\lambda}  K_{\lambda \pi_1} K_{\mu \pi_2}
\langle P_{\pi_1}(\bm{x}, Q), P_{\pi_2}(\bm{x}, Q)\rangle \\ 
& = \sum_\lambda \sum_{\pi} c_{\mu\lambda} b_{\mu}^{-1}  K_{\lambda \pi} K_{\mu \pi}\; ,
  \end{split}
\end{equation}
and we conclude that 
\begin{equation}
\label{eq:indices}
c_{\mu \nu} = \sum_\pi K_{\mu\pi}^{-1} b_\mu (K_{\nu\pi}^{-1})^T\; .
\end{equation}
\marginpar[left]{\tiny Fix these \\ indices later.}

Hence, we conclude that 
\begin{equation}
\sum_{\mu} S_{\mu}(\bm{x},Q) s_{\nu}(\bm{y}) = 
\sum_{\mu , \nu} c_{\mu\nu} s_{\mu}(\bm{x}) s_{\nu}(\bm{y}) 
\end{equation}
and everything boils down to the analysis of the
indices~(\ref{eq:indices}). In has been shown
in~\cite{Necoechea:2019wbg} that the big Schur functions themselves
are KP-tau functions\footnote{ We give a simple argument for this fact
in the remark 2 below.} therefore, the coefficients \((c_\lambda)_\mu
\equiv c_{\lambda\mu}\) must satisfy Plücker relations for the KP
hierarchy.

\begin{remark}
 This expression show somethign interesting. We can formulate this
problem in terms of partitions defined in the phase
model. Equivalently, we can use the usual vertex operator
construction, and not the q-deformed version of Jing~\emph{\cite{Jing1991,
  Jing1995}}, that is much harder to deal with.
\end{remark}

If these coefficients satisfy the Toda hierarchy Plücker relations, then 
one can find a \(Q\)-dependent group element \(G_Q\in GL(\infty)\) such that 
\begin{equation}
  c_{\mu\nu} = \bra{\mu} G_Q \ket{\nu}\; ,
\end{equation}
where \(\ket{\mu}\) and \(\ket{\nu}\) are phase model
states. Unfortunatelly, we could not find any good reason to explore
this expansion in a consistent way. In what follows, we try to explore
alternative expansions that give a double Schur expansion.

%%%%%%%%%%%%%%%%%%%%%%%%%%%%%%%%%%%%%%%%%%%%%%%%%%
%%%%%%%%%%%%%%%%%%%%%%%%%%%%%%%%%%%%%%%%%%%%%%%%%%

\subsection{Scalar product: Supersymmetric Schur polynomials}

With this result, we now argue that~(\ref{eq:scalar-hl}) is also a tau
function of the Toda hierarchy in disguise. Let us decompose
\begin{equation}
 \sum_{m} q_m(\bm{y}, Q) z^m =
 \prod_j (1 + y_j (-Q z)) \prod_k (1 - y_k z)^{-1} = 
 \sum_{j, k} e_j(\bm{y}) h_k(\bm{y}) (- Q)^j z^{j+k}\; .
\end{equation}
And reorganizing this sum, we conclude that 
\begin{equation}
  q_m(\bm{y}, Q)  = \sum_{j=0}^m e_j(\bm{y}) h_{m-j}(\bm{y}) (- Q)^j =
 \sum_{j=0}^m e_j(-Q\bm{y}) h_{m-j}(\bm{y}) \; ,
\end{equation}
where in the last equality we have used the homogeneity of the elementary symmetric polynomials,
and \( -Q\bm{y} = (-Qy_1, -Qy_2, \dots)\). 

It is easier to consider the following expansions 
\begin{subequations}
\begin{equation}
  \begin{split}
    \prod_{j=1}^N (1 - (Q y_i)z) & = \prod_{j=1}^N e^{ \ln  (1 - (Q y_i)z)} = 
    \prod_{j=1}^N \exp \left( - \sum_{n>0} \frac{Q^n y^n_j}{n} z^n \right) \\ 
    & = \exp \left( - \sum_{n>0} \sum_{j=1}^N \frac{Q^n y^n_j}{n} z^n \right)  =
    \exp \left( - \sum_{n>0} t^{(Q)}_n z^n \right)  
  \end{split}
\end{equation}
where \(t^{(Q)}_n = \frac{Q^n}{n} \sum_j y_j^n = Q^n t'_n\), and 
\begin{equation}
    \prod_{j=1}^N \frac{1}{(1 - y_iz)} = \prod_{j=1}^N e^{- \ln  (1 - y_iz)} = 
    \prod_{j=1}^N \exp \left( \sum_{n>0} \frac{y_j}{n} z^n \right)=
    \exp \left(\sum_{n>0} t_n z^n \right) \; . 
\end{equation}
\end{subequations}

Therefore 
\begin{equation}
 \sum_{m} q_m(\bm{y}, Q) z^m =
 \prod_j \frac{1 + y_j (-Q z)}{1 - y_k z}
 = \exp \left( \sum_{n\geq 1} (t'_n - t_n^{(Q)})z^n \right)
 \equiv \exp \left( \sum_{n\geq 1} T_n z^n \right)\; ,
\end{equation}
with \(T_n = (1 - Q^n) t'_n\). Then, we conclude that \(q_m(\bm{y}, Q)\)
are homogeneous polynomials with respect the Miwa coordinates
\(\bm{T} = (T_1, T_2, \dots)\). Then
\begin{equation}
  \mathcal{S}_\lambda(\bm{t}', Q) = \det \left(h_{\lambda_i - i +j}(\bm{T})\right) = s_\lambda(\bm{T})\; .
\end{equation}
All in all, we conclude that the big Schur functions
\(S_\lambda(\bm{t}', Q)\) are ordinary Schur functions with respect
the coordinates \(\bm{T}\).

In fact, we can do even better. Supersymmetric (or Hook) Schur
functions~\cite{Berele:1983}, see also see~\cite{Macdonald:1998, Moens:2003},
\(s_\lambda(\bm{\alpha}/\bm{\beta})\) are defined as ordinary schur
functions evaluated at Miwa coodinates of the form
\begin{equation}
  T_n = \frac{1}{n} \left( \sum_{i=1}^{\dim(\alpha)} \alpha_i^n
       - \sum_{i=1}^{\dim(\beta)} (-\beta_i)^n\right)\; ,
\end{equation}
and comparing with the results above, we can see that the big Schur functions
\(S_\lambda(\bm{t}', Q)\)\; , is the supersymmetric Schur functions for \(\bm{\alpha} = \bm{y}\) 
and \(\bm{\beta} = - Q \bm{y}\), that is 
\begin{equation}
  S_\lambda(\bm{t}', Q) = s_\lambda[\bm{y}/(- Q\bm{y})]\; .
\end{equation}

Putting these facts together, we immediately conclude that since 
\begin{equation}
  \sum_\lambda s_\lambda(\bm{T}) s_\lambda (\bm{t}) \; ,
\end{equation}
is a Toda hierarchy tau function, we obviously have that 
\begin{equation}
  S_Q(N,M| \bm{x}, \bm{y}) \equiv S_Q(N,M| \bm{t}, \bm{T})
  = \sum_{\lambda \subseteq [N,N,M]} s_\lambda(\bm{T}) s_\lambda(\bm{t})\; , 
\end{equation}
is also a restricted tau function of the Toda hierarchy with respect
to \(\bm{t}\) and \(\bm{T}\).

Now we can consider an expansion that differs from
from~(\ref{eq:kf-exp}). Let us consider the supersymmetric Schur
polynomials in terms of ordinary Schur functions as~\cite[Sec. I.5,
  exerc. 23]{Macdonald:1998}
\begin{equation}
  s_{\lambda}(\bm{x}/\bm{y}) = \sum_\mu s_{(\lambda/\mu)'}(\bm{y}) s_\mu(\bm{x})\; ,
\end{equation}
where the prime denotes the conjugate diagram. Then
\begin{equation}
  \label{eq:tau-hl}
  \sum_{\lambda } s_\lambda(\bm{y}/ (- Q\bm{y})) s_\lambda(\bm{x}) =
  \sum_{\lambda,\mu } s_{(\lambda/\mu)'}(- Q\bm{y}) s_\lambda(\bm{x}) s_\mu(\bm{y})\; .
\end{equation}

From this expression we conclude (and speculate) the following. 

\begin{remark}
Since the skew Schur polynomials have determinant expressions,
therefore we have that the \(\bm{y}\)-dependent coefficients \(c_{\mu
  \lambda} = s_{(\lambda/\mu)'}(-Q\bm{y})\) have Jacobi-Trudi
expressions. It is tempting to think of these objects as
\(\bm{y}\)-dependent Pl\"ucker coordinates. In this sense, we would have a curve in 
the indinite Grassmannian, instead a point.
\end{remark}

\begin{remark}
Remeber that one of the easiest nontrivial solutions of the KP
hierarchies are the Schur functions
themselves~\emph{\cite{Zabrodin2018}}. We have just seen that the
big-Schur polynomials can be written as supersymmetric Shur
polynomials, and these are nothing but ordinary Schur polynomials in a
particular choice of Miwa coordiantes. Then, we also see that 
the big Schur polynomials are also KP tau functions. This direct conclusion 
is an alternative proof of this fact, see~\emph{\cite{Necoechea:2019wbg}}.
\end{remark}

Putting these facts together, we have that the left-hand side
of~(\ref{eq:tau-hl}) is a tau function with respect the coordinates
\(\bm{T}\) and \(\bm{t}\). Furthermore, from the Vertex operator
formalism, we write the Schur polynomials as
\begin{equation}
  s_{\lambda}(\bm{x}) = \bra{\lambda} e^{J_-}\ket{\bm{0}}\qquad
  s_{\mu}(\bm{y}) = \bra{\bm{0}} e^{J_+}\ket{\mu}\; ,
\end{equation}
we have
\begin{equation}
  \sum_{\lambda,\mu } s_\mu(\bm{y})c_{\mu\lambda}(Q,\bm{y}) s_\lambda(\bm{x}) = 
  \sum_{\lambda,\mu } \bra{\bm{0}} e^{J_+}\ket{\mu} c_{\mu\lambda}(Q,\bm{y})
  \bra{\lambda} e^{J_-}\ket{\bm{0}} \; ,
\end{equation}
therefore 
\begin{equation}
  c_{\mu\lambda}(Q, \bm{y}) = \bra{\mu} G_Q(\bm{y}) \ket{\lambda}\; ,
\end{equation}
where \(G_Q(\bm{y}) \in GL(\infty)\). We can find the expression for
this group element as follows. 

Remember that \(T_m = t'_m - Q^m t'_m\), therefore we write
\begin{subequations}
\begin{equation}
  \bra{\bm{0}} e^{J_+(\bm{t})} e^{J_-(\bm{T})} \ket{\bm{0}} = 
\bra{\bm{0}} e^{J_+(\bm{t})} e^{J_-(\bm{t}')  - J_-(\bm{t}^{(Q)})} \ket{\bm{0}} \; .
\end{equation}
From the Heisenberg algebra, we conclude that \([J_-(\bm{t}),
  J_-(\bm{t}')] = 0\), therefore
\begin{equation}
  \bra{\bm{0}} e^{J_+(\bm{t})} e^{J_-(\bm{T})} \ket{\bm{0}} = 
\bra{\bm{0}} e^{J_+(\bm{t})} e^{-J_-(\bm{t}^{(Q)})} e^{J_-(\bm{t}')}  \ket{\bm{0}} \; ,
\end{equation}
\end{subequations}
from where we finally conclude that 
\begin{equation}
  G_{Q}(\bm{y}) = \exp \left(-J_-(\bm{t}^{(Q)})\right) \qquad
  t^{(Q)}_n = \frac{Q^n}{n}\sum_j y_j^n\; . 
\end{equation}

In general, the expectation values with coordinate dependent group
elements \(G_{Q}(\bm{y})\) are not Toda hierarchy tau functions. But
in the above example, the coordinates combine among themselves and
generate a tau function with respect the coordiantes \(\bm{T}\) and
\(\bm{t}\).
 
%%%%%%%%%%%%%%%%%%%%%%%%%%%%%%%%%%%%%%%%%%%%%%%%%%
%%%%%%%%%%%%%%%%%%%%%%%%%%%%%%%%%%%%%%%%%%%%%%%%%%

%*%*%*%*%*%*%*%*%*%*%*%*%*%*%*%*%*%*%*%*%*%*%*%*%*
\section{Conclusions and Perspectives}
%*%*%*%*%*%*%*%*%*%*%*%*%*%*%*%*%*%*%*%*%*%*%*%*%*

Let us now consider the fermionic construction of this tau function. We write 
the limit \(N,M\to \infty\) of the scalar product
\begin{subequations}
\begin{equation}
  S_Q(\bm{T}, \bm{t}') = \sum_{\mu} s_\mu(\bm{T}) s_\mu(\bm{t}')
\end{equation}
that is quite quite trivial from the viewpoint of the free fermion construction, 
\begin{equation}
  S_Q(\bm{T}, \bm{t}') = \bra{\bm{0}} e^{\bm{J}_+(\bm{T})} e^{\bm{J}_-(\bm{t}')} \ket{\bm{0}}\; .
\end{equation}

Moreover, we also know that
\begin{equation}
  S_Q(\bm{t}, \bm{t}') = \sum_{\mu\nu} c_{\mu\nu} s_\mu(\bm{t}) s_\mu(\bm{t}')
\end{equation}
where \(n t = \sum_j x^n_j\) and \(n t' = \sum_j y^n_j\) are Miwa coordinates, then  
\begin{equation}
  S_Q(\bm{t}, \bm{t}') = \bra{\bm{0}} e^{J_+(\bm{t})} G_Q  e^{J_-(\bm{t}')} \ket{\bm{0}}
\end{equation}
and \(G_Q \in GL(\infty)\) that depends on \(Q\). We could also think of \(G_Q\)
as a twist
\(G_Q = e^{Q H} G e^{-Q H}\), with the action of \(H\in \mathfrak{gl}(\infty)\)
on ordinary fermions generating the Q-fermions. 
\marginpar[left]{\tiny Investigate this\\ twists}
\end{subequations}

It seems that we have some different representations for this tau function, 
and it is worth investigating what is happening here. That is what we want 
to do now. 

\printbibliography

\end{document}
