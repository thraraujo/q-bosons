\documentclass[a4paper,11pt]{amsart}

\usepackage{/home/thiago/.config/dot-files/latex/globaldef}

% DEFINITIONS

%%%%%%%%%%%%%%%%%%%%%%%%
% ADDITIONAL PACKAGES
%%%%%%%%%%%%%%%%%%%%%%%% 
\usepackage{parskip}
\usepackage{tikz} % Drawing package
\usepackage[backend=biber, style=alphabetic]{biblatex}
\usepackage{listings}
\usepackage{ytableau}

\usetikzlibrary{intersections,decorations.text} % this is to make the cover
\usetikzlibrary{matrix, arrows.meta} % this is to improve diagrams

\setlength\parindent{0pt}

%\setdefaultlanguage[variant=brazilian]{portuguese}
%\setdefaultlanguage{english}

%%%%%%%%%%%%%%%%%%%%%%%%
% METADATA
%%%%%%%%%%%%%%%%%%%%%%%% 

%\usepackage{framed} 


% Official manual defines the options
\hypersetup{ 
	pdftitle={integrable systems, programming and so on},
	pdfsubject={High Energy Physics and so on}, 
	pdfauthor={author},
	pdfkeywords={gauge; susy; strings; fields; cft; python},
	colorlinks=true, % false: color frames ; true: color links
    linkcolor=myPurple, % Color of internal links (sections, pages and so on)
    citecolor=myPurple, % color for bibliographical citations
    urlcolor =myPurple, % color for linked URL
    linktocpage=true % link page to table of contents
}


% Options for listings package I saw the sniipet below here: 
% https://stackoverflow.com/questions/3175105/inserting-code-in-this-latex-document-with-indentation
\lstset{frame=tb,
  language=Python,
  aboveskip=3mm,
  belowskip=3mm,
  showstringspaces=false,
  columns=flexible,
  basicstyle={\small\ttfamily},
  numbers=none,
  numberstyle=\tiny\color{myPurple},
  keywordstyle=\color{myRed},
  commentstyle=\color{myBlue},
  stringstyle=\color{myPurple!80},
  breaklines=true,
  breakatwhitespace=true,
  tabsize=3
}

%%%%%%%%%%%%%%%%%%%%%%%% 
% COLORS
%%%%%%%%%%%%%%%%%%%%%%%% 
% see palette here: https://github.com/enkia/tokyo-night-vscode-theme
\definecolor{myPurple}{RGB}{90, 74, 120}
\definecolor{myBlue}{RGB}{15, 75, 110}
\definecolor{myRed}{RGB}{191,97,106}
\definecolor{myDarkGray}{RGB}{216, 222, 233}
\definecolor{myLightGray}{RGB}{236, 239, 244}

\definecolor{c1}{RGB}{129, 162, 193}%  {15, 75, 110} % myBlue
\definecolor{c2}{RGB}{216, 222, 233} % myDarkGray
\definecolor{c3}{RGB}{236, 239, 244} % myLightGray
\definecolor{c4}{RGB}{59, 66, 82}
\definecolor{c5}{RGB}{76, 86, 106}

% This color is a framed requirement
\definecolor{shadecolor}{RGB}{236, 239, 244}

%%%%%%%%%%%%%%%%%%%%%%%%
% THEOREM
%%%%%%%%%%%%%%%%%%%%%%%% 

\newtheorem{theorem}{Theorem}
\newtheorem{corollary}{Corollary}
\newtheorem{proposition}{Proposition}
\newtheorem{conjecture}{Conjecture}
\newtheorem{lemma}{Lemma}
\newtheorem{example}{Example}
\newtheorem{remark}{Remark}

%%%%%%%%%%%%%%%%%%%%%%%%
% MACROS
%%%%%%%%%%%%%%%%%%%%%%%% 

\DeclarePairedDelimiter{\bra}{\langle}{\rvert}
\DeclarePairedDelimiter{\ket}{\lvert}{\rangle}
\DeclarePairedDelimiter{\bbra}{\langle\!\langle}{\rvert}
\DeclarePairedDelimiter{\kket}{\lvert}{\rangle\!\rangle}
\DeclarePairedDelimiterX{\bracket}[2]{\langle}{\rangle}{#1\vert#2}
\DeclarePairedDelimiterX{\bbracket}[2]{\langle\!\langle}{\rangle\!\rangle}{#1\vert#2}


\bibliography{bib-database.bib}

\begin{document}

%%%%%%%%%%%%%%%%%%%%%%%%%%%%%%%%%%%%%%%%%%%%%%%%%%
%%%%%%%%%%%%%%%%%%%%%%%%%%%%%%%%%%%%%%%%%%%%%%%%%%

\title{Comments on \(Q\)-Bosons, deformed fermions and their relations
    with integrable hierarchies}

\author{Thiago Araujo}

\address{\noindent 
Instituto de Física Teórica, UNESP-Universidade Estadual Paulista
R. Dr. Bento T. Ferraz 271, Bl. II, Sao Paulo 01140-070, SP, Brazil\\
and
Instituto de Física\\ Universidade de S\~ao Paulo\\ 
Rua do Matão Travessa 1371, 05508-090\\ São Paulo, SP. Brazil
}

\email{\texttt{\href{tr.araujo@unesp.br}{tr.araujo@unesp.br}}}

\keywords{integrability, spin-chains}
\subjclass[2020]{82B20, 82B23}
\date{\today}

\begin{abstract}
  In this work, we study aspects of the \(Q\)-boson model, and 
its relations with integrable hierarchies. 
% \vspace{4pt}
 \noindent \textbf{Keywords:} Integrability, Q-bosons, Schur
\end{abstract}

\maketitle

\setcounter{tocdepth}{1}
\tableofcontents

%*%*%*%*%*%*%*%*%*%*%*%*%*%*%*%*%*%*%*%*%*%*%*%*%*
\section{Phase Model and q-Bosons}
%*%*%*%*%*%*%*%*%*%*%*%*%*%*%*%*%*%*%*%*%*%*%*%*%*

This section introduces the \(q\)-boson and phase
models~\cite{Bogoliubov:1992, Bogoliubov:1997soj, Bogoliubov2005,
  Bogoliubov:1997soj, Tsilevich:2006}, and consists of a general
review of the literature. Consequently, there is nothing particularly
original, except its presentation.  In spite of that, I believe that
it has some pedagoginal value, and a good collection of references.
Here we follow some conventions of~\cite{Wheeler:2010vmq}, but with a
presentation that is closer to~\cite{Tsilevich:2006}.

%%%%%%%%%%%%%%%%%%%%%%%%%%%%%%%%%%%%%%%%%%%%%%%%%%
%%%%%%%%%%%%%%%%%%%%%%%%%%%%%%%%%%%%%%%%%%%%%%%%%%

\subsection{Phase model}
Consider the \((M+1)\) set of operators \(\{\phi_i,
\phi^\dagger,\mathcal{N}_i\}_{i=0}^M\) such that
\begin{equation}
 [\mathcal{N}_i, \phi_j] = - \phi_i \delta_{i,j} \quad
 [\mathcal{N}_i, \phi_j^\dagger] =  \phi_i^\dagger \delta_{i,j}  \quad 
 [\phi_i, \phi_j^\dagger] =  \pi_i \delta_{i,j}  
\end{equation}
where \(\pi_i =(\ket{0} \bra{0})_i\) is the vacuum projection operator.
These operators can be written as
\begin{equation}
\phi = \sum_{n\geq 0}\ket{n}\bra{n+1}\quad 
\phi^\dagger = \sum_{n\geq 0}\ket{n+1}\bra{n} \quad 
N = \sum_{n\geq 0}n\ket{n}\bra{n}\; ,
\end{equation}
and it is easy to see that \(\phi^\dagger \phi = \bm{1} - \ket{0}
\ket{0}\) and \(\phi\phi^\dagger = \bm{1}\).

The Hamiltonian is given by
\begin{equation}
  H = - \frac{1}{2} \sum_{n =0}^M \left(\phi_n^\dagger \phi_{n+1}
  + \phi_n \phi_{n+1}^\dagger \right) + \bar{\mathcal{N}}\; .
\end{equation}
where \(\bar{\mathcal{N}} = \sum_{i=0}^M \mathcal{N}_i\), and
periodic boundary conditions \(\phi_{M+1} = \phi_0\) and
\(\phi_{M+1}^\dagger = \phi_0^\dagger\).

These operators appear in the context of quantum optics, and for
this reason it is referred as the \emph{phase model}. It 
corresponds to the strongly correlated \(q\)-bosons
model~\cite{Bogoliubov:1997soj} the we will define soon.

\subsubsection{Partitions representation}
Let us define the i\(^{th}\)-vacuum \(\ket{0}_i\) by the the conditions 
\(\phi_i\ket{0}_i\). The representation of this Hilbert space, denoted by
\(\mathcal{F}_i\), is given by \(\ket{n_i}_i = \phi_i^\dagger \ket{0}_i\). 

Given the vacuum \(\ket{\bm{0}} = \ket{0}_0\otimes \ket{0}_1
\otimes \cdots \otimes  \ket{0}_M\),
the Fock space is defined as 
\begin{equation}
  \mathcal{F} = \bigotimes_{i=0}^M \mathcal{F}_i = 
  \left\{\ket{\vec{n}} = \ket{n_0}\otimes \ket{n_1} \otimes \cdots
  \otimes \ket{n_M} \ | \ n_i \in \mathbb{N} \right\}\; .
\end{equation}
where the states \(\ket{\vec{n}}\) are defined as 
\begin{equation}
  \ket{\vec{n}} = \ket{n_0}\otimes \ket{n_1} \otimes \cdots \otimes \ket{n_M} 
 : =  (\phi_0^\dagger)^{n_0} (\phi_1^\dagger)^{n_1} \cdots  (\phi_M^\dagger)^{n_M} \ket{\bm{0}} \; ,
\end{equation}
where we write \(\phi_i^\dagger \equiv \bm{1} \otimes  \cdots \otimes
\phi_i^\dagger \otimes \cdots \otimes \bm{1}\).
We will assume that these states are normalized, that is
\(\bracket{\vec{n}}{\vec{m}}=\delta_{\vec{n}, \vec{m}}\). 

Finally, the actions of the operators \(\mathcal{N}_i\) and \(\pi_i\) are
\begin{equation}
    \pi_i\ket{\vec{n}}  = \delta_{n_i, 0} \ket{\vec{n}} \qquad 
    \mathcal{N}_i\ket{\vec{n}} = n_i \ket{\vec{n}}\; .
\end{equation}
 
Given a state \(\ket{\vec{n}} = \ket{n_0, n_1, \dots, n_M}\), we
associate a partition \( \lambda = (0^{n_0} 1^{n_1} 2^{n_2} \cdots
M^{n_M})\). Observe that this correspondence is not unique, since the
partition \(\lambda\) ignores the number \(n_0\) of particles in the
site \(0\). From the total number of particles, we can, on the other hand,
determine the number of particles \(n_0 = N - \ell(\lambda)\), where
\(\ell(\lambda)\) is the number of rows in the Young diagram defined
by \(\lambda\). This is very important, since in our consideations 
below, the \(N\) particle sector is fixed, thanks to the integrability of 
the model, therefore, the number \(n_0\) is known once we define the 
partition \(\lambda\). 

In fact, from the correspondence above, Wheeler~\cite{Wheeler:2010vmq}
defines a map \(\mathcal{M}_\psi: \mathcal{F}\to
\mathcal{F}^{(0)}_\psi\), where \(\mathcal{F}^{(0)}_\psi\) is the
space Fock space of charged free fermions built upon the neutral
(fermionic) vacuum. We will discuss a map from the bosonic space to
the space of symmetric functions.

\subsubsection{Bethe Ansatz}
The \(L\)-matrix is given by 
\begin{equation}
  L_{an} = 
\begin{pmatrix}
x^{ - 1/2} & \phi_n^\dagger \\ \phi_n & x^{1/2}
\end{pmatrix}_a\; , 
\end{equation}
from where we can build the monodromy matrix 
\begin{equation}
  T_a(x) = L_{aM}(x) \cdots L_{a0}(x) = 
\begin{pmatrix}
A(x) & B(x) \\ C(x) & D(x)
\end{pmatrix}_a\; .
\end{equation}

With these expressions, one can finally build the Bethe states and their dual 
\begin{equation}
  \ket{\widehat{\Psi}(y_1, \dots, y_N)} = \prod_{j=1}^N B(y_j) \ket{\bm{0}}\qquad 
  \bra{\widehat{\Psi}(y_1, \dots, y_N)} = \bra{\bm{0}}\prod_{j=1}^N C(y_j) \; .
\end{equation}
When the set \(\{ y_j \ | \ j =1, \dots , N\}\) satisfies the Bethe equations 
\begin{equation}
  y^{N + M}_i = (-1)^{N-1} \prod_{\substack{j = 1 \\ j \neq i}}^N y_j\; , \qquad i = 1, \dots, N\; , 
\end{equation}
we says that the Bethe states are \emph{on-sheel}. In what follows, we will 
only consider \emph{off-shell} states. 

\subsubsection{Coordinate expansion}
From the map defined in~\cite{Wheeler:2010vmq}, we can associate to
each state \(\ket{\lambda}\), a Schur function \(s_\lambda(\vec{x})\),
where the dummy coordinates are \(\vec{x} = (x_1, \dots, x_N)\), with
\(N = \sum_{j=0}^M n_j\). Therefore, we can work in the ring of
symmetric functions \(\Lambda_M\), with basis given by Schur functions
\(s_\lambda\) and Young diagrams such that \(\max(\lambda_1) = M\),
and \(\max(\ell(\lambda)) = N\).

That also means that the operator \(B\) should also be realized in the
space \(\Lambda_M\), see~\cite{Tsilevich:2006}. The operators
\(\phi^\dagger_j\) and \(\phi_j\) are, respectively, \(1\)-particle
creation and annihilation at the site \(j\). In terms of partitions,
we can think of them as insertion and removal of a row of length
\(j\). That is, given the state \(\ket{\lambda} =\ket{1^{n_1} 2^{n_2}\cdots}\), 
we have
\begin{equation}
\begin{split}
  \phi^\dagger_j \ket{\lambda} & = \ket{1^{n_1} 2^{n_2}\cdots j^{j+1}\cdots} \equiv \ket{\lambda \oplus (j)}\\
  \phi_j \ket{\lambda} & = \ket{1^{n_1} 2^{n_2}\cdots j^{j-1}\cdots}
  \equiv \ket{\lambda \ominus (j)}\; , 
\end{split}
\end{equation}
Once again, we see that the case \(j=0\) is meaningless in the
partition construction. Given two partitions \(\lambda\) and \(\mu\),
we say that \(\lambda\) interlaces \(\mu\) if \(\lambda_i \geq \mu_i \geq \lambda_{i+1}\), 
and we write \(\lambda\succ \mu\). Therefore, it is easy to see that the partition 
\(\lambda\oplus (j)\) interlaces \(\lambda\), and \(\lambda\)
interlaces \(\lambda \ominus (j)\). We write this condition as 
\(\lambda \oplus (j) \succ \lambda \succ \lambda \ominus (j)\).

Now, we can use the explicit expression for the \(L\)-matrix to build
the monodromy matrix, and extract the operator \(B(x)\) as
\begin{subequations}
\begin{equation}
  B(x)_a = T_a(x)_{12} = \sum_{j_1, \dots, j_M = 1,2} L_{a j_M}(x) L_{j_M j_{M-1}}(x)\cdots  L_{j_1 2}(x)\; ,
\end{equation}
and it is easy to see that it has the form  
\begin{equation}
  B(x) \equiv x^{-M/2} \mathbb{B}(x) = x^{-M/2} \sum_{k=0}^M x^k \mathcal{B}_k \; .
\end{equation}
The problem now is to find the action of the operators \(\mathcal{B}_k\) on
states \(\ket{\lambda}\).
\end{subequations}
Let us consider some simple cases.

\textbullet \ M=0: We have just one site in the  
chain, and all particles site at the site \(0\), as we have discussed many times, 
does not contribute to the partition representation.

\textbullet \ M=1: Now we have 1 row Young diagrams \(\lambda = (1^{n_1})\), where each row
corresponds to a boson. It is easy to show that 
\begin{equation}
 \mathbb{B}_{M=1}(x) =  x \phi_1^\dagger + \phi_0^\dagger \; ,
\end{equation}
therefore, \(\mathcal{B}_0 = \phi_0^\dagger\) and  \(\mathcal{B}_1 = \phi_1^\dagger\). 
From this expression, we have 
\begin{equation}
  \mathcal{B}_0\ket{(1^{n_1})} = \ket{(1^{n_1})} \qquad  
  \mathcal{B}_1\ket{(1^{n_1})} = \ket{(1^{n_1+1})} \; .
\end{equation}
All in all, we conclude that 
\begin{equation}
 \mathbb{B}_{M=1}(x)\ket{\lambda}  =  \ket{\lambda} + x\ket{\lambda \oplus (1)} \;.
\end{equation}

\textbullet \ M=2: Now we have 2 row Young diagrams \(\lambda = (1^{n_1}2^{n_2})\).
Now, one can show that
\begin{equation}
  \mathbb{B}_{M=2}(x) = x^2 \phi_2^\dagger + x \left( \phi_1^\dagger
  + \phi_2^\dagger \phi_1 \phi_0^\dagger \right) + \phi_0^\dagger \; ,
\end{equation}
therefore,
\begin{equation}
\mathcal{B}_0 = \phi_0^\dagger \qquad 
\mathcal{B}_1 = \phi_1^\dagger + \phi_2^\dagger \phi_1 \phi_0^\dagger \qquad 
\mathcal{B}_2 = \phi_2^\dagger \; .
\end{equation}

Consequently. we find
\begin{subequations}
\begin{equation}
\begin{split}
  \mathcal{B}_0\ket{(1^{n_1}2^{n_2})} & = \ket{(1^{n_1} 2^{n_2})} \qquad  
  \mathcal{B}_1\ket{(1^{n_1}2^{n_2})} = \ket{(1^{n_1+1} 2^{n_2})} + \ket{(1^{n_1-1} 2^{n_2 + 1})} \\
  \mathcal{B}_2\ket{(1^{n_1}2^{n_2})} & = \ket{(1^{n_1}2^{n_2+1})}\; .
\end{split}
\end{equation}
\end{subequations}
Observe that the action of the operator \(\mathbb{B}_k(x)\) on \(\ket{\lambda}\)
generates diagrams \(\ket{\mu}\) such that \(k = |\mu| - |\lambda|\). 
\begin{equation}
  \mathbb{B}_{M=2}(x) \ket{\lambda}  =
  \sum_{\lambda \prec \mu \subseteq [2,\ell(\lambda) + 1]} x^{|\mu|-|\lambda|} \ket{\mu}\; .
\end{equation}

One can keep doing this game, and we reach to the general conclusion that 
\begin{equation}
  \mathbb{B}(x) \ket{\lambda}  =
  \sum_{\lambda \prec \mu \subseteq [M,\ell(\lambda) + 1]} x^{|\mu|-|\lambda|} \ket{\mu}\; .
\end{equation}

Therefore, we conclude that the action of the operators
\(\mathcal{B}_k\) on \(\ket{\lambda}\) generates diagrams
\(\ket{\mu}\) that are obtained from \(\lambda\) by adding \(k\) boxes,
not in different columns. From the viewpoint of symmetric functions, this rule
is exactly the Pieri's rule~\cite{Macdonald:1998}, that states that
we have that
\begin{equation}
  s_{(k)} s_\lambda = \sum_{\mu \succ \lambda} s_\mu\;.
\end{equation}

Therefore, we have the map from the spin chain Hilbert space to the
ring of symmetric functions given by \(\imath : \mathcal{B}_k \mapsto
s_{(k)}\) where \(s_{(k)}\) is the Schur function with 1 row of length
\(k\).




%%%%%%%%%%%%%%%%%%%%%%%%%%%%%%%%%%%%%%%%%%%%%%%%%%
%%%%%%%%%%%%%%%%%%%%%%%%%%%%%%%%%%%%%%%%%%%%%%%%%%

\subsection{q-Bosons}
The Hamiltonian for the q-boson model is
given by
\begin{equation}
  \mathcal{H} = -\frac{1}{2} \sum_{n=0}^M
  \left(b_n^\dagger b_{n+1} + b_n b_{n+1}^\dagger \right) + \bar{\mathcal{N}}\; ,
\end{equation}
where \(\bar{\mathcal{N}} = \sum_{i=0}^M \mathcal{N}_i\), and as in
the case of the phase mode, we impose periodic boundary conditions
\(b_{M+1} = b_0\) and \(b_{M+1}^\dagger = b_0^\dagger\). The set of
operators \(\{b_i, b^\dagger,\mathcal{N}_i\}_{i=0}^M\) form the
\(M+1\) copies of independent q-boson algebras:
\begin{equation}
[\mathcal{N}_i, b_j^\dagger]=\delta_{i,j} b_i^\dagger\; , \quad 
[\mathcal{N}_i, b_j]=-\delta_{i,j}b_i\; , \quad
[B_i, B_j^\dagger]= \delta_{i,j} q^{-2\mathcal{N}_i}  \equiv \delta_{i,j} Q^{\mathcal{N}_i}\; , 
\end{equation}
where we write the deformation parameter as \(Q = q^{-2}\). 
In the limit \(Q\to 1\), the q-bosons are ordinary bosons, and 
the limit \(Q\to 0\) (\(q\to \infty\)) gives the phase model we have discussed above. 

\subsubsection{Representation}
We build the representation of the q-boson algebra with some modifications of the phase 
model we have just discussed. We suppose that the action of the operators \(\{b_i, b_i^\dagger\}\) 
satisfies 
\begin{subequations}
\begin{alignat}{3}
    & b_0\ket{n_0} =(1 - \delta_{0, n_0}) \frac{1 - Q^{n_0}}{(1 - Q)^{1/2}}\ket{n_0 - 1}
    &\hspace{0.5cm}& b_0^\dagger \ket{n_0} =  \frac{1}{(1 - Q)^{1/2}}\ket{n_0 + 1}  \\
    & b_i\ket{n_i} = \frac{(1 - \delta_{0, n_i})}{(1 - Q)^{1/2}}\ket{n_i - 1}
    && b_i^\dagger \ket{n_0} =  \frac{(1 - Q^{n_i+1})}{(1 - Q)^{1/2}}\ket{n_0 + 1}\quad i\neq 0
\end{alignat}
\end{subequations}
where the modes \(\{b_0, b_0^\dagger \}\) satisfies a slightly different relation.

\subsubsection{Bethe Ansatz}
The \(L\)-operator for the q-boson is given by
\begin{equation}
  L_n(x, Q) =
  \begin{pmatrix}
    x^{-1/2} & (1 - Q)^{1/2} b_n^\dagger \\ (1 - Q)^{1/2} b_n & x^{1/2}
  \end{pmatrix}\; .
\end{equation}
The monodromy matrix is 
\begin{equation}
  T_a(x,Q) = L_{am}(x, q)  \dots  L_{a0}(x, q) = 
  \begin{pmatrix}
    A(x) & B(x) \\ C(x) & D(x)
  \end{pmatrix}\; .
\end{equation}

The eigenstates of the Hamiltonian are of the form
\begin{equation}
  \ket{\Psi(y_1, \dots, y_N; Q)} = \prod_{j=1}^N B(y_j, Q) \ket{\bm{0}}\qquad 
  \bra{\Psi(y_1, \dots, y_N; Q)} = \bra{\bm{0}}\prod_{j=1}^N C(y_j, Q) \; .
\end{equation}
When the set \(\{ y_j \ | \ j =1, \dots , N\}\) satisfies the Bethe equations 
\begin{equation}
  y^{N + M}_i =\prod_{\substack{j = 1 \\ j \neq i}}^N\frac{Q y_i - y_j}{y_i - Q y_j}\; , \qquad i = 1, \dots, N\; , 
\end{equation}
Once again, we will consider \emph{off-shell} Bethe states. 

%%%%%%%%%%%%%%%%%%%%%%%%%%%%%%%%%%%%%%%%%%%%%%%%%%
%%%%%%%%%%%%%%%%%%%%%%%%%%%%%%%%%%%%%%%%%%%%%%%%%%



%*%*%*%*%*%*%*%*%*%*%*%*%*%*%*%*%*%*%*%*%*%*%*%*%*
\section{Toda tau functions and the phase model}
%*%*%*%*%*%*%*%*%*%*%*%*%*%*%*%*%*%*%*%*%*%*%*%*%*

Bogoliubov~\cite{Bogoliubov2005}, in the notation~\cite{Wheeler:2010vmq},
has shown that the the scalar product of two vectors in the
\(N\)-particles sector in a chain of length \(M + 1\) is
\begin{equation}
\begin{split}
\label{eq:scalar}
  S(N, M|\bm{x}, \bm{y}) & =
  \bra{\bm{0}} \prod_{j=1}^N \mathbb{C}(x_j) \prod_{j=1}^N \mathbb{B}(y_j) \ket{\bm{0}} \\ 
 & = \frac{ \det H(\bm{x},\bm{y})}{ \prod_{i<j}(x_i - x_j)(y_i - y_j)} \; ,
\end{split}
\end{equation}
where \(H\) is an \(N\times N\) matrix with components
\begin{equation}
  H_{ij} = H(x_i, y_j) 
  =\frac{1 - (x_i y_j)^{ M + N}}{1 - x_i y_j }\; .
\end{equation}

This scalar product is related to the counting of plane
partitions, with
\begin{equation}
 S(N, M| \bm{x}, \bm{y}) 
 = \sum_{\lambda \subseteq (N, N, M)} s_{\lambda}(\bm{x}) s_{\lambda} (\bm{y})\; .
\end{equation}
In the limit \(M, N \to \infty\), one can use the Cauchy identity
\begin{equation}
  \sum_{\lambda } s_{\lambda}(\bm{x}) s_{\lambda}(\bm{y})
  = \prod_{i,j>0} \frac{1}{1- x_iy_j}\; ,
\end{equation}
and specialize this expression to \( x_i = y_i = q^{i-1/2} \) to find
the MacMahon function
\begin{equation}
  \sum_{\lambda } s_{\lambda}(\bm{q}_\star) s_{\lambda}(\bm{q}_\star) = \prod_{i,j>0} \frac{1}{1- q^{i +j -1}}\; ,
\end{equation}
where \(\bm{q}_{\star} = (q^{1/2}, q^{3/2}, \dots)\). 

%%%%%%%%%%%%%%%%%%%%%%%%%%%%%%%%%%%%%%%%%%%%%%%%%%
%%%%%%%%%%%%%%%%%%%%%%%%%%%%%%%%%%%%%%%%%%%%%%%%%%
%%%%%%%%%%%%%%%%%%%%%%%%%%%%%%%%%%%%%%%%%%%%%%%%%%

\subsection{KP and Toda tau functions}

Wheeler has shown in~\cite{Wheeler:2010vmq} that the the scalar
product~(\ref{eq:scalar}) is a KP tau function with respect to the
Miwa coordinates \(t_m = \frac{1}{m}\sum_j x_j^m\) (\(t'_m =
\frac{1}{m}\sum_j y_j^m\)), where the Schur functions
\(s_{\lambda}(\bm{y}) = c_\lambda\) ( \(s_{\lambda}(\bm{x}) =
c'_\lambda\)) become Plücker coordinates, and vice-versa.
The proof for this statement is relies on the map \(\mathcal{M}_\psi\)
between the space \(\mathcal{V}\) os Bethe states and Fock
space \(\mathcal{F}_\psi\) of free fermions. 

In fact, there is an easier way to see this fact. First one may see that 
\begin{equation}
 S(N,M,\bm{y}|\{\bm{t}\}) 
 = \sum_{\lambda} s_\lambda(\bm{y})  s_\lambda(\bm{t}) 
 \equiv \sum_{\lambda} c_\lambda(\bm{y})  s_\lambda(\bm{t}) 
\end{equation}
is a tau function of the KP hierarchy, since
\(c_\lambda(\bm{y}) = \det (h_{\lambda_i-i +j}(\bm{y}))\) are Plücker
coordinates~\cite{Miwa2000, Foda:2009zz}. Therefore, the truncation
\begin{equation}
 S(N,M,\bm{y}|\{\bm{t}\}) 
 = \sum_{\lambda \subseteq [N, \infty)} s_\lambda(\bm{y})  s_\lambda(\bm{t}) 
\end{equation}
is also a tau function. According to~\cite{Alexandrov:2012tr,
  Kharchev:1991gd, Zabrodin:2010ii}, the truncation of the tau
function corresponds to the inclusion of a projection operator
\(\mathrm{P}^+\) in the expectation value of the tau function.
% See section 3.3.2 of~\cite{Alexandrov:2012tr}, 

\subsubsection{Scalar product}
We also argue that this is also a tau function of the 2D toda hierarchy. 
We can write the Cauchy's identity as
\begin{subequations}
\begin{equation}
  \sum_{\lambda } s_{\lambda}(\bm{t}) s_{\lambda}(\bm{t}')
    = \exp \left( \sum_{m\geq1} m t_m t'_m \right) \; .
\end{equation}
where \(\bm{t}\) and \(\bm{t}'\) are two independent sets of Miwa coordinates.
It is a trivial tau function of the 2D Toda hierarchy, since one can write it as
\begin{equation}
\sum_{\lambda } s_{\lambda}(\bm{t}) s_{\lambda}(\bm{t}')
 = \bra{\bm{0}} e^{\bm{J}_+(\bm{t})} e^{-\bm{J}_-(\bm{t}')} \ket{\bm{0}}
\end{equation}
and from the free-fermions representation of the tau functions,
see~\cite{Alexandrov:2012tr}, we have that it is a tau function of the
Toda hierarchy with trivial element \(1 \in GL(\infty)\).
\end{subequations}

Therefore, the truncation 
\begin{equation}
 S(N,M|\{\bm{t}\},\{\bm{t}'\}) = 
  \sum_{\lambda \subseteq [N,N,M]} s_{\lambda}(\bm{t}) s_{\lambda}(\bm{t}')
\end{equation}
is a also a restricted tau function. These proofs follow from simple
properties of tau functions, and it should be contrated with the rich
formalism in~\cite{Wheeler:2010vmq}. Moreover, from~\cite{Zabrodin:2010ii}
we have that the this tau function can be written as the matrix integral
\begin{equation}
  S(N,M|\{\bm{t}\}, \{\bm{t}'\}) =
  \frac{1}{N!} \prod_{\ell=1}^N \oint_{\Gamma_\ell} \frac{dz_\ell}{2 \pi i z_\ell}
  e^{\xi_+(\bm{t}_+, z_\ell) - \xi_-(\bm{t}_-, z_{\ell}^{-1})} \Delta(z)\Delta(z^{-1})
\end{equation}

When the Bethe equations are satisfied, we have that \(x_j = y_j = e^{ip_j}\),
therefore \(t_{-m} = - t_m^\star\), and the the phase model is equivalent to
an ensemble of \(N\) 2D Coulomb particles on a circle. 

%%%%%%%%%%%%%%%%%%%%%%%%%%%%%%%%%%%%%%%%%%%%%%%%%%
%%%%%%%%%%%%%%%%%%%%%%%%%%%%%%%%%%%%%%%%%%%%%%%%%%
%%%%%%%%%%%%%%%%%%%%%%%%%%%%%%%%%%%%%%%%%%%%%%%%%%


\subsubsection{Correlation functions}
Bogoliubov has also shown that the correlation functions
\begin{subequations}
\begin{equation}
  P(m|\bm{x}, \bm{y}) = \bra{0} \prod_{j=1}^N C(x) \prod_{j=1}^{N-1} B(y) \phi_m^\dagger \ket{0}
\end{equation}
can be written as
\begin{equation}
  P(m|\bm{x}, \bm{y}) = \frac{(-1)^{N-1}}{(\sqrt{y_N}^{N-1}}
  \left( \prod_{t<N} \frac{x_N - y_t}{y_t} \right) \frac{\det Q}{\det H} S(N,M|\bm{x}, \bm{y})
\end{equation}
\end{subequations}
where the \(N\times N\) matrix \(Q\) has components 
\begin{equation}
 Q_{jN} = x_j^{(M + N - 1- 2m)/2} \quad  \textrm{and} \quad 
 Q_{jk} = H_{jk} \; , 
\end{equation}
where \(H_{jk} = H(x_j, y_k)\) are the components of the matrix \(H\). 
Since the components \(Q_{jN}\) are independent of the coordinates \(y\), 
we cannot write the above expression in terms of a Toda hierarchy tau
function. 

We already know that 
\begin{equation}
   S(N,M|\bm{x}, \bm{y}) =  \Delta(x) \Delta(y) \det H \; ,
\end{equation}
therefore we define
\begin{equation}
  \mathbb{P}(m|\bm{x}, \bm{y}) =\frac{\det Q}{\Delta(x)}\; ,
\end{equation}
and we are going to treat the coordinates \(\{ \bm{y} \}\) as
a set of \(N\) fixed constants, in a such a way that we write \(N\)
functions \(\bm{F}(z) = (F_1, \dots, F_N)\) defined as
\begin{equation}
  F_j (z) = H(z, y_j) \quad \textrm{if} \ j \neq N\; , \qquad \textrm{and}\qquad 
   F_N (z)  = z^{(M + N - 1 - 2m)/2} 
\end{equation}
Moreover, we have that
\begin{equation}
  H(z, y_j) = \frac{1 - (y_j z)^{M + N}}{1 - y_j z} = \sum_{n=0}^{N + M - 1} (y_j z)^n\; , 
\end{equation}
therefore, we write
\begin{equation}
  F_j(z) =  \sum_{n=0}^{M + N -1} f_{j, n} z^n
\end{equation}
where we will also assume the \(M + N \in 2 \mathbb{Z}\). With these definitions,
we conclude that
\begin{equation}
  \mathbb{P}(m|\bm{x}, \bm{y}) =\frac{\det_{jk} F_j(x_k)}{\Delta(x)}\; ,
\end{equation}
and we write~\cite{Takasaki:2010qm}, 
\begin{equation}
  \det_{jk} F_j(x_k) = \det_{jk} \left(  \sum_{n=0}^{M + N -1} f_{j, n} x_k^n \right) \; .
\end{equation}
With the aid of the Cauchy-Binet formula, we write  
\begin{equation}
  \mathbb{P}(m|\bm{x}, \bm{y}) = \sum_{0\leq \ell_N\leq \dots \leq \ell_1 \leq N+M}
  \frac{\det_{jk}(f_{j, \ell_k}) \det_{jk}(x_k^{\ell_j})}{\Delta(x)}\; ,
\end{equation}
and from the definition of the Schur polynomials and the
Jacobi-Trudi expression of Plücker coordinates, we can write this last
expression as 
\begin{equation}
  \mathbb{P}(m|\bm{x}, \bm{y}) = \sum_{\lambda} c_\lambda(\bm{y}) s_\lambda(\bm{x}) \; ,
\end{equation}
where \(c_\lambda(f_{j, \ell_k})\),  \(\ell_k = \lambda_k - k +N\), define point
in the finite dimensional Grassmannian manifold \(\mathrm{Gr}(N, N+M)\). 

%%%%%%%%%%%%%%%%%%%%%%%%%%%%%%%%%%%%%%%%%%%%%%%%%%
%%%%%%%%%%%%%%%%%%%%%%%%%%%%%%%%%%%%%%%%%%%%%%%%%%
%%%%%%%%%%%%%%%%%%%%%%%%%%%%%%%%%%%%%%%%%%%%%%%%%%


\subsubsection{Not everything is a tau-function}
It is also worth noticing that not everything is a tau function. Consider the state
\begin{equation}
  \ket{\mathcal{Y}} =
 \sum_{\vec{n}} \prod_{j=0}^M \ket{n_j}\quad \sum_j n_j = N\; .
\end{equation}
We have~\cite{Bogoliubov2005} the correlation function 
\begin{equation}
  \prod_{j=1}^N x^{M/2} \bra{\mathcal{Y}} \prod_{j=1}^N B(\bm{x}) \ket{\bm{0}} = 
  \sum_{\lambda\subseteq (N,N,M)} s_\lambda(\bm{x})\; . 
\end{equation}
This sum of Schur polynomials are not tau functions of the KP hierarchy.

We can easily see this fact, if we consider the the Fermionic
construction. Since
\begin{equation}
\tau(\bm{t}) = \bra{\bm{0}} e^{\bm{J}_+(\bm{t})} G \ket{\bm{0}}
\end{equation}
where \(G\in GL(\infty)\), and 
\begin{equation}
  \bra{\bm{0}} e^{\bm{J}_+(\bm{t})} = \sum_\lambda \bra{\mu} s_\lambda(\bm{t})\; , 
\end{equation}
therefore
\begin{equation}
\tau(\bm{t}) = \sum_{\lambda} \bra{\lambda} G \ket{\bm{0}} s_\lambda(\bm{t})
\end{equation}
and we need \(1 = \bra{\lambda} G \ket{\bm{0}} \). Then, 
\begin{equation}
  \ket{\Psi_0} = G\ket{\bm{0}} = \sum_\lambda \ket{\lambda}\; ,
\end{equation}
is a coherent state, and we conclude that
\begin{equation}
G = 1 +  \sum_{d\geq 1} \prod_{j=1}^d \psi^\star_{a_j} \psi_{-b_j}\ \notin \ GL(\infty) \;. 
\end{equation}
It is interesting to notice that this state appears as the ground
state of the crystal melting problem in~\cite{Dijkgraaf:2008ua}.

Therefore, the restriction for diagrams with \(\ell(\lambda_1) \leq N\) is
not a tau function either.

%%%%%%%%%%%%%%%%%%%%%%%%%%%%%%%%%%%%%%%%%%%%%%%%%%
%%%%%%%%%%%%%%%%%%%%%%%%%%%%%%%%%%%%%%%%%%%%%%%%%%
%%%%%%%%%%%%%%%%%%%%%%%%%%%%%%%%%%%%%%%%%%%%%%%%%%

\section{Scalar products for the Q-bosons}

Here I would like to start considering the q-boson problem. It has
been shown~\cite{Wheeler:2010vmq, Sulkowski:2008mx, Tsilevich:2006}
that the scalar product of two off-shell Bethe states in the q-boson
model is given by
\begin{equation}
S_Q(N,M | \bm{x}, \bm{y}) = \bra{\bm{0}} \prod_{j=1}^N C(x_j, Q)
\prod_{k=1}^N B(y_k, Q) \ket{\bm{0}}
= \sum_{\lambda \subseteq
  (N,N,M)} b_\lambda(Q) P_{\lambda}(\bm{x}, Q) P_{\lambda}(\bm{y}, Q)
\end{equation}
where \(P_\lambda\) denote the Hall-Littlewood polynomials.

It turns out that this expansion is  also related to the Cauchy identity for 
Hall-Littlewood polynomials~\cite{Macdonald:1998}
\begin{equation}
\sum_{\lambda} b_\lambda(Q) P_{\lambda}(\bm{x}, Q) P_{\lambda}(\bm{y}, Q)
= \prod_{j, k=1}^\infty \frac{1-Q x_j y_k}{1 - x_j y_k}\; ,
\end{equation}
and using the results in~\cite[see chap.3, sec. 4, eq. (4.7)]{Macdonald:1998}, 
we can expand this identity as 
\begin{equation}
\label{eq:cachy-hl}
 \prod_{j, k=1}^\infty \frac{1-Q x_j y_k}{1 - x_j y_k} = 
\sum_{\lambda} S_{\lambda}(\bm{x}, Q) s_{\lambda}(\bm{y}) =
\sum_{\lambda} S_{\lambda}(\bm{y}, Q) s_{\lambda}(\bm{x}) \; ,
\end{equation}
where the polynomials, the big-Schur~\cite{Wheeler:2018}, are given by 
a Jacobi-Trudi formula
\begin{equation}
  S_{\lambda} (\bm{y}, Q) = \det(q_{\lambda_i -i + j}(\bm{y}, Q))\; , 
\end{equation}
and the coefficients \(q_m\) can be obtained from the following expression
\begin{equation}
 \sum_{m} q_m(\bm{y}, Q) z^m =
 \prod_{j} \frac{1-Q y_j z}{1 - y_j z}
\end{equation}
where \(z\) is a formal variable. It has been argued
in~\cite{Foda:2008hn} that if we interpret the \(Q\)-Schur functions
as Plücker coordiantes, the identity~(\ref{eq:cachy-hl}) is a KP-tau
function.

Therefore, we see that the scalar product 
\begin{equation}
  S_Q(N,M | \bm{x}, \bm{y})
  = \sum_{\lambda \subseteq (N,N,M)} S_{\lambda}(\bm{x}, Q) s_{\lambda}(\bm{y})
  = \sum_{\lambda \subseteq (N,N,M)} S_{\lambda}(\bm{y}, Q) s_{\lambda}(\bm{x})\; ,
\end{equation}
is a restricted KP tau function with respect to both set of coordinates,
that is \(\bm{x}\) and \(\bm{y}\).

%%%%%%%%%%%%%%%%%%%%%%%%%%%%%%%%%%%%%%%%%%%%%%%%%%
%%%%%%%%%%%%%%%%%%%%%%%%%%%%%%%%%%%%%%%%%%%%%%%%%%
%%%%%%%%%%%%%%%%%%%%%%%%%%%%%%%%%%%%%%%%%%%%%%%%%%

\subsection{Supersymmetric Schur polynomials}

Let us also decompose 
\begin{equation}
 \sum_{m} q_m(\bm{y}, Q) z^m =
 \prod_j (1 + y_j (-Q z)) \prod_k (1 - y_k z)^{-1} = 
 \sum_{j, k} e_j(\bm{y}) h_k(\bm{y}) (- Q)^j z^{j+k}\; .
\end{equation}
And reorganizing this sum, we conclude that 
\begin{equation}
  q_m(\bm{y}, Q)  = \sum_{j=0}^m e_j(\bm{y}) h_{m-j}(\bm{y}) (- Q)^j =
 \sum_{j=0}^m e_j(-Q\bm{y}) h_{m-j}(\bm{y}) \; ,
\end{equation}
where in the last equality we have used the homogeneity of the elementary symmetric polynomials,
and \( -Q\bm{y} = (-Qy_1, -Qy_2, \dots)\). 

%One can also define the set of coordinates \(\bm{\xi} =
%\{\xi_i(\bm{x}, Q)\}\), in such a way that
%\begin{equation}
% \sum_{m} q_m(\bm{x}, Q) z^m := \sum_{m} h_m(\bm{\xi}) z^m =
% \prod_k \frac{1}{(1 - \xi_k z)} \;.
%\end{equation}
%In this sense, the \(Q\)-Schur function is an ordinary Schur function
%with respect the variables \(\bm{\xi}\).

It is easier to consider the following expansions 
\begin{subequations}
\begin{equation}
  \begin{split}
    \prod_{j=1}^N (1 - (Q y_i)z) & = \prod_{j=1}^N e^{ \ln  (1 - (Q y_i)z)} = 
    \prod_{j=1}^N \exp \left( - \sum_{n>0} \frac{Q^n y^n_j}{n} z^n \right) \\ 
    & = \exp \left( - \sum_{n>0} \sum_{j=1}^N \frac{Q^n y^n_j}{n} z^n \right)  =
    \exp \left( - \sum_{n>0} t^{(Q)}_n z^n \right)  
  \end{split}
\end{equation}
where \(t^{(Q)}_n = \frac{Q^n}{n} \sum_j y_j^n = Q^n t'_n\), and 
\begin{equation}
    \prod_{j=1}^N \frac{1}{(1 - y_iz)} = \prod_{j=1}^N e^{- \ln  (1 - y_iz)} = 
    \prod_{j=1}^N \exp \left( \sum_{n>0} \frac{y_j}{n} z^n \right)=
    \exp \left(\sum_{n>0} t_n z^n \right) \; . 
\end{equation}
\end{subequations}

Therefore 
\begin{equation}
 \sum_{m} q_m(\bm{y}, Q) z^m =
 \prod_j \frac{1 + y_j (-Q z)}{1 - y_k z}
 = \exp \left( \sum_{n\geq 1} (t'_n - t_n^{(Q)})z^n \right)
 \equiv \exp \left( \sum_{n\geq 1} T_n z^n \right)\; ,
\end{equation}
with \(T_n = (1 - Q^n) t'_n\). Then, we conclude that \(q_m(\bm{y}, Q)\)
are homogeneous polynomials with respect the Miwa coordinates
\(\bm{T} = (T_1, T_2, \dots)\). All in all, we conclude that 
\begin{equation}
  S_\lambda(\bm{t}', Q) = \det \left(h_{\lambda_i - i +j}(\bm{T})\right) = s_\lambda(\bm{T})\; .
\end{equation}

Supersymmetric Schur functions, see~\cite{Zinnjustin2009}, \(s_\lambda(\bm{\alpha}/\bm{\beta})\)
are defined as ordinary schur functions evaluated at Miwa coodinates
of the form
\begin{equation}
  T_n = \frac{1}{n} \left( \sum_{i=1}^{\dim(\alpha)} \alpha_i^n
       - \sum_{i=1}^{\dim(\beta)} (-\beta_i)^n\right)\; ,
\end{equation}
and comparing with the results above, we can see that the big Schur functions
\(S_\lambda(\bm{t}', Q)\)\; , is the supersymmetric Schur functions for \(\bm{\alpha} = \bm{y}\) 
and \(\bm{\beta} = - Q \bm{y}\), that is 
\begin{equation}
  S_\lambda(\bm{t}', Q) = s_\lambda[\bm{y}/(- Q\bm{y})]\; .
\end{equation}

Putting these facts together, we immediately conclude that since 
\begin{equation}
  \sum_\lambda s_\lambda(\bm{T}) s_\lambda (\bm{t}) \; ,
\end{equation}
is a Toda hierarchy tau function, we obviously have that 
\begin{equation}
  S_Q(N,M| \bm{x}, \bm{y}) \equiv S_Q(N,M| \bm{t}, \bm{T})
  = \sum_{\lambda \subseteq [N,N,M]} s_\lambda(\bm{T}) s_\lambda(\bm{t})\; , 
\end{equation}
is also a restricted tau function of the Toda hierarchy.

%%%%%%%%%%%%%%%%%%%%%%%%%%%%%%%%%%%%%%%%%%%%%%%%%%
%%%%%%%%%%%%%%%%%%%%%%%%%%%%%%%%%%%%%%%%%%%%%%%%%%

\subsection{Toda hierarchy}

One of the greatest advantages of the expressions above is that we do not 
need to consider the t-Schur fermions.

It would be very interesting to have some basic understanding of what
is happenning in the fermionic representation of this tau function.
Let us expand these polynomials in a Schur polynomials basis, that is
\begin{equation}
P_\lambda(\bm{x}, Q) = \sum_{\mu} K^{-1}_{\lambda \mu}(Q) s_\lambda(\bm{x})\; , 
\end{equation}
where \(K^{-1}_{\mu\nu}(Q) \in \mathbb{Z}[Q]\) are inverse
Kostka–Foulkes coefficients, and \(K^{-1}_{\lambda\mu} = 0 \) if
\(|\lambda|\neq |\mu|\)~\cite{Macdonald:1998, Wheeler:2018}.

Expanding now the \(Q\)-Schur functions as
\begin{equation}
  S_\lambda(\bm{x}, Q) = \sum_{\mu} c_{\lambda\mu}(Q) s_\mu(\bm{x})\; ,
\end{equation}
we can fix the coefficients \(c_{\lambda\mu}\) in terms of Kostka-Foulkes
polynomials as follows. Since~\cite{Macdonald:1998} 
\begin{equation}
  \frac{1}{b_\mu}\langle P_\mu(\bm{x}, Q), P_\mu(\bm{x}, Q)\rangle = \langle
  S_\mu(\bm{x}, Q), s_\mu(\bm{x})\rangle = \delta_{\mu\nu}\; ,  
\end{equation}
we have
\begin{equation}
  \begin{split}
\delta_{\mu\nu} & = \sum_\lambda c_{\mu\lambda}  \langle s_\lambda(\bm{x}), s_\mu(\bm{x})\rangle\\
& = \sum_\lambda \sum_{\pi_1, \pi_2} c_{\mu\lambda}  K_{\lambda \pi_1} K_{\mu \pi_2}
\langle P_{\pi_1}(\bm{x}, Q), P_{\pi_2}(\bm{x}, Q)\rangle \\ 
& = \sum_\lambda \sum_{\pi} c_{\mu\lambda} b_{\mu}^{-1}  K_{\lambda \pi} K_{\mu \pi}\; ,
  \end{split}
\end{equation}
and we conclude that 
\begin{equation}
\label{eq:indices}
c_{\mu \nu} = \sum_\pi K_{\mu\pi}^{-1} b_\mu (K_{\nu\pi}^{-1})^T\; .
\end{equation}
\marginpar[left]{\tiny Fix these \\ indices later.}

Hence, we conclude that 
\begin{equation}
\sum_{\mu} S_{\mu}(\bm{x},Q) s_{\nu}(\bm{y}) = 
\sum_{\mu , \nu} c_{\mu\nu} s_{\mu}(\bm{x}) s_{\nu}(\bm{y}) 
\end{equation}
and everything boils down to the analysis of the
indices~(\ref{eq:indices}).

Do they satisfy Plücker relations? We already know that the
big Schur functions themselves are KP-tau
functions, see also~\cite{Necoechea:2019wbg}, therefore, the coefficients
\((c_\lambda)_\mu \equiv c_{\lambda\mu}\) must satisfy Plücker relations
for the KP hierarchy.

From the above considerations, we can define 2 set of functions \(F_i(x)\) and
\(G_i(y)\) such that
\begin{equation}
 F_i(x) = \sum_{n} F_{i,n} x^n \qquad 
 G_i(y) = \sum_{n} G_{i,n} y^n\; ,
\end{equation}
in such a way that 
\begin{equation}
  c_{\mu\nu} = \det(F_{i, m_j} | G_{i, n_k})\; ,
\end{equation}
and we conclude that the scalar product is, indeed, a Toda tau function. 
\marginpar[left]{\tiny Improve this\\ discussion}
 
%%%%%%%%%%%%%%%%%%%%%%%%%%%%%%%%%%%%%%%%%%%%%%%%%%
%%%%%%%%%%%%%%%%%%%%%%%%%%%%%%%%%%%%%%%%%%%%%%%%%%

\subsection{!! Fermionic construction}

Let us now consider the fermionic construction of this tau function. We write 
the limit \(N,M\to \infty\) of the scalar product
\begin{subequations}
\begin{equation}
  S_Q(\bm{T}, \bm{t}') = \sum_{\mu} s_\mu(\bm{T}) s_\mu(\bm{t}')
\end{equation}
that is quite quite trivial from the viewpoint of the free fermion construction, 
\begin{equation}
  S_Q(\bm{T}, \bm{t}') = \bra{\bm{0}} e^{\bm{J}_+(\bm{T})} e^{\bm{J}_-(\bm{t}')} \ket{\bm{0}}\; .
\end{equation}

Moreover, we also know that
\begin{equation}
  S_Q(\bm{t}, \bm{t}') = \sum_{\mu\nu} c_{\mu\nu} s_\mu(\bm{t}) s_\mu(\bm{t}')
\end{equation}
where \(n t = \sum_j x^n_j\) and \(n t' = \sum_j y^n_j\) are Miwa coordinates, then  
\begin{equation}
  S_Q(\bm{t}, \bm{t}') = \bra{\bm{0}} e^{J_+(\bm{t})} G_Q  e^{J_-(\bm{t}')} \ket{\bm{0}}
\end{equation}
and \(G_Q \in GL(\infty)\) that depends on \(Q\). We could also think of \(G_Q\)
as a twist
\(G_Q = e^{Q H} G e^{-Q H}\), with the action of \(H\in \mathfrak{gl}(\infty)\)
on ordinary fermions generating the Q-fermions. 
\marginpar[left]{\tiny Investigate this\\ twists}
\end{subequations}

It seems that we have some different representations for this tau function, 
and it is worth investigating what is happening here. That is what we want 
to do now. 


%\section{Charged fermions and the Fermion-Boson Correspondence}

\subsection{Representation}

Here I use the notation of~\cite{Dijkgraaf:2008ua, Justin2008} with
some small changes in~\cite{Alexandrov:2012tr, Marino:2005sj}. Other useful references
are \cite{Jimbo:1983if, Babelon2003, Borot2015}. The idea we want to
explore is the relation between Young diagrams and free fermions. We
start with the definition of the generating functions
\begin{equation}
 \psi(z) = \sum_{r \in \mathbb{Z} + 1/2} \psi_{-r} z^{r - 1/2}\qquad 
 \psi^\star(z) = \sum_{r \in \mathbb{Z} + 1/2} \psi_{r}^\star z^{r - 1/2}\; . 
\end{equation}
where the operators \(\psi_r\) and \(\psi_s^\ast\), satisfy the canonical
anticommutation relations
\begin{subequations}
\begin{equation}
\label{anticom}
\{\psi_r, \psi_s\} = \{\psi_r^\star, \psi_s^\star\} = 0
\qquad \{\psi_r, \psi_s^\star\} =\delta_{rs} \quad r, s \in \mathbb{Z}+1/2\; . 
\end{equation}
We can also define the occupancy and vacancy operators at the position
\(r\) 
\begin{equation}
\eta_r = \psi_r^\star \psi_r \, \quad
\zeta_r = 1- \psi_r^\star \psi_r\; .
\end{equation}
\end{subequations}
This is a simple model of charged free fermions, where the particles
are created by the operator \(\psi^\star_r\) and annihilated by
\(\psi_r\) at the \(r^{th}\) site. Equivalently, holes at the site
\(r\) are created by \(\psi_r\) and are annihilated by
\(\psi_r^\star\).

The vacuum configuration \(\ket{\bm{0}}\) is defined by the dirac Sea
filled up to the location \(r=0\), that is
\begin{center}
    \begin{tikzpicture}
    \node at (-4,0) { \(|\bm{0}\rangle=\)};
    \node at (-3,0) {\(\cdots\)};
    \draw (-2.5,0) -- (-2,0);
    \node at (-2,0) {\(\bullet\)};
    \draw (-2,0) -- (-1,0);
    \node at (-2,-.5) {\(-\tfrac{3}{2}\)};
    \node at (-1,0) {\(\bullet\)};
    \draw (-1,0) -- (-.06,0);
    \node at (-1,-.5) {\(-\tfrac{1}{2}\)};
    \node at (0,0) {\(\circ\)};
    \draw (.06,0) -- (.95,0);
    \node at (0,-.5) {\(\tfrac{1}{2}\)};
    \node at (1,0) {\(\circ\)};
    \draw (1.05,0) -- (1.5,0);
    \node at (1,-.5) {\(\tfrac{3}{2}\)};
    \node at (2,0) {\(\cdots\)};
  \end{tikzpicture}
\end{center}	
We assume that this state exists. Alternatively, one might define the empty state
\(\ket{\bm{\infty}}\) as \(\psi_r\ket{\bm{\infty}} = 0 \), \(\forall \ r \), therefore
\(\ket{\bm{0}} = \prod_{r < 0} \psi^\star_r \ket{\bm{\infty}}\). Regardless the case,
these definitions give equivalent results. 

We then have
\begin{equation}
\psi_{r}^\star\ket{\bm{0}}=0\ \ \ r< 0\; , \qquad  
\psi_{r} \ket{\bm{0}} = 0 \ \ \ r> 0
\end{equation}
The dual vacuum is defined as
\begin{equation}
\bra{\bm{0}} \psi_{r}=0\ \ \ r< 0\; , \qquad  
\bra{\bm{0}} \psi_{r}^\star = 0 \ \ \ r> 0\; ,
\end{equation}
therefore, we have the relation \((\psi_r)^\dagger\equiv \psi_r^\star\).

We can also build the shifted vacuum \(\ket{\ell}\) with \(\ell \in
\mathbb{Z}\) as
\begin{equation}
  \label{eq:charged-states}
  \ket{\ell} = 
  \left\{
  \begin{array}{ll}
    \psi^\star_{\ell - 1/2} \psi^\star_{\ell - 3/2} \cdots \psi^\star_{3/2} \psi^\star_{1/2} \ket{\bm{0}} & \ell > 0 \\
    \psi_{\ell + 1/2} \psi_{\ell + 3/2} \cdots \psi^\star_{-3/2} \psi^\star_{-1/2} \ket{\bm{0}} & \ell < 0
  \end{array}
  \right.
\end{equation}
These states satisfy
\begin{equation}
\psi_r\ket{\ell} = 0 \quad r > \ell \quad , \quad  
\psi_r^\star\ket{\ell} = 0  \quad r < \ell \; .
\end{equation}
Let us denote the Fock spaces labeled by these vacua as
\(\mathcal{F}_\ell\), for a complete discussion on the
structure of these spaces, see~\cite{Wheeler:2010vmq}. 
In particular, we have
\begin{equation}
\ket{\ell + 1} = \psi^\ast_{\ell + 1/2} \ket{\ell} \qquad
\ket{\ell} = \psi_{\ell + 1/2} \ket{\ell + 1} \; .
\end{equation}

Many of the results and identities we use have been explained in the
review~~\cite{Alexandrov:2012tr}. In order to pass from their notation
to ours, we can simply consider the transformation
\begin{equation}
\psi_k \equiv (\psi_{AZ})^\star_{k - 1/2} \qquad
\psi_k^\star \equiv (\psi_{AZ})_{k - 1/2} \; .
\end{equation}
where \textrm{AZ} stands for \emph{Alexandrov-Zabrodin.}

%%%%%%%%%%%%%%%%%%%%%%%%%%%%%%%%%%%%%%%%%%%%%%%%%%
%%%%%%%%%%%%%%%%%%%%%%%%%%%%%%%%%%%%%%%%%%%%%%%%%%

\subsection{States and Partitions}

A partition is a sequence of weakly decreasing non-negative integers,
\(\mu_i \geq \mu_{i+1}\) \(i = 1, \dots, n\). We can think of
partition as Young diagrams, where the \(i^{th}\) line has \(\mu_i\)
boxes. Given the partition \(\mu = (\mu_1, \mu_2, \dots, \mu_r)\),
the Frobenius coordinates \(\mu = (\alpha_j , \beta_j)\) are defined as
\begin{subequations}
\begin{equation}
\alpha_j=\mu_j - j \; , \quad 
\beta_j=\mu^T_j - j \; .
\end{equation}
It is convenient to write 
\begin{equation}
a_j=\alpha_j +\frac{1}{2}	\; , \quad 
b_j=\beta_j+\frac{1}{2}\; 
\end{equation}
and hereafter, these are the Frobenius coordinates we work with. For
further details, see~\cite{Alexandrov:2012tr}.
\end{subequations}
Moreover, we denote the number of
boxes in a partition \(\mu\) as
\begin{equation}
\#\texttt{Boxes}\equiv |\mu| = \sum_{j=1}^d (a_j+b_j)
\end{equation}

Given a partition \(\mu = (\mu_1, \mu_2, \cdots, \mu_n)\), one
can associate a fermionic state \cite{Dijkgraaf:2008ua, Okounkov:2003sp}
\begin{equation}
\label{eq:part-states}
\ket{\mu} = (-1)^{b(\lambda)} \prod_{j=1}^d \psi_{a_j}^\star \psi_{-b_j}\ket{\bm{0}}\; , \qquad
\langle \mu | = (-1)^{b(\lambda)}  \bra{\bm{0}}\prod_{j=1}^d  \psi_{-b_j}^\star \psi_{a_j}\;
,\quad a_j, b_j>0\ \ \forall j\; ,
\end{equation}
where we have added the factor \(b(\lambda) = \sum_{j = 1}^d (b_j -
1/2)\) for convenience. In~\cite{Alexandrov:2012tr}, the authors
define the same factor in the Schur polynomials expansion. We have
moved this factor to the states, as in~\cite{Marino:2005sj}, so that
the Schur polynomials expansio does not have any negative sign moving
around. It is worth noticing that if we have different partitions,
\(\ket{\lambda} \neq \ket{\nu}\), then there is a mismatch in the
number of pairs \(\psi^\star \psi\), and the states are orthonormal,
\(\langle \mu |\nu \rangle=\delta_{\mu\nu}\).

It is just a combinatorial exercice to show that the partition
states~\ref{eq:part-states}, and more generally the charged states,
can be equivalently written as
\begin{equation}
  \label{eq:part-states:2}
  \ket{\mu, \ell} = \psi^\star_{\ell + \mu_1 - 1/2} \dots
  \psi^\star_{\ell + \mu_n - n + 1/2} \ket{\ell - n}\; .
\end{equation}
See~\cite{Alexandrov:2012tr} for a discussion on both representations.

The normal ordering (with respect to the vacuum \(\ket{\bm{0}}\) is
defined as
\begin{equation}
:\psi_s \psi_r^\star: = \psi_s \psi_r^\star - \bra{\bm{0}} \psi_s
  \psi_r^\star \ket{\bm{0}} \; .
\end{equation}
That means that we need to move all creation operators to the left, and all
annihilation operators to the left, taking into account the negative factors
comming from the anticommutation relations. That is
\begin{equation}
:\psi_s \psi_r^\star:   = - :\psi_r^\star \psi_s: = 
\left\{
\begin{array}{rll}
-\psi_r^\star \psi_s && r>0 \\
\psi_s \psi_r^\star && r<0 
\end{array}	
\right.
\end{equation}

Using these relations, one can see that the number of boxes in
a generic partition \(\ket{\mu}\) is the eigenvalue of the following
operator \cite{Marino:2005sj}
\begin{equation}
L_0 = \sum_{r \in \mathbb{Z} + 1/2} r :\psi_{-r} \psi_r^\star:
\end{equation}
Therefore, we have
\begin{equation}
q^{L_0}|\mu \rangle = q^{|\mu|}|\mu \rangle \; .
\end{equation}	

% CHECK THESE THINGS LATER
% One interpretation to this system is given in terms of the free fermion
% \begin{subequations}
% \begin{equation}
% H= \sum_r r  \psi_r^\star \psi_r \; .
% \end{equation}
% It is easy to see that the state \(\ket{\bm{0}}\) is, indeed,
% a vacuum of the free Hamiltonian
% \begin{equation}
% H=\sum_r r :\psi^\star_r \psi_r: \; .
% \end{equation} 
% Explicitly, we have
% \begin{equation}
% \begin{split} 
%   H \ket{\bm{0}} & = \left(  \sum_{r<0} r  :\psi^\star_r \psi_r:  + \sum_{r>0} r 
%    :\psi^\star_r \psi_r:  \right) \ket{\bm{0}}  \\
%   & = \left(  \sum_{r<0} r  :\psi^\star_r \psi_r:  + \sum_{r>0} r  :\psi^\star_r 
%    \psi_r:  \right) \ket{\bm{0}}  \\
%   & = \left( - \sum_{r<0} r \psi_r \psi^\star_r  +
%        \sum_{r>0} r \psi^\star_r \psi_r \right) \ket{\bm{0}} = 0	
% \end{split}
% \end{equation}
% The Hamiltonian above is equivalent to the continuous system
% \begin{equation}
% S\sim \int d z\ \psi^\star(z) \partial \psi(z)\ \ \Rightarrow \ \  \sum_r r  \psi_r^\star \psi_r  
% \end{equation}
% \end{subequations}

%%%%%%%%%%%%%%%%%%%%%%%%%%%%%%%%%%%%%%%%%%%%%%%%%%
%%%%%%%%%%%%%%%%%%%%%%%%%%%%%%%%%%%%%%%%%%%%%%%%%%

\subsection{\(\mathfrak{gl}(\infty)\) and \(\widehat{\mathfrak{gl}(1)}\) Algebras}

The free fermion formulation allows us to rewrite the partition
problem above in terms of algebraic problems. Bilinears
\(\psi_r\psi_s^\star\) are closed under commutation
\begin{equation}
[\psi_r\psi_s^\star, \psi_p\psi_q^\star]= \delta_{sp} \psi_r\psi_q^\star -  \delta_{sq} \psi_p\psi_s^\star\; .
\end{equation}
We can define the objects
\begin{equation}
X = \sum_{r,s} M_{rs}   \psi_r \psi^\star_{s}   \; ,
\end{equation}
and it is easy to see that the normal ordering introduces a central
extension in the commutation relation above.  All in all, we define
the algebra \(\widehat{\mathfrak{gl}}(\infty)\) \cite{Babelon2003}
\begin{equation}
\mathfrak{gl}(\infty): = \{X=M_{rs} \psi_r \psi^\star_{s} \ |
\ M_{rs}=0\ \forall\ |r+s|\geq N>> 0 \}\oplus \mathbb{C}\; .
\end{equation}

The Heisenberg algebra \(\hat{\mathfrak{u}}(1) \equiv
\widehat{\mathfrak{gl}}(1) \subset \widehat{\mathfrak{gl}}(\infty) \)
is defined as
\begin{equation}
[J_m , J_n]  =m \delta_{m+n,0}\; ,
\end{equation} 
where the generators are given by
\begin{equation}
J_m := \sum_r  :\psi_{r-m}^\star \psi_r :\; , 
\qquad 
J_m^\dagger := \sum_r :\psi_r^\star \psi_{r-m}:\;. 
\end{equation} 
It is easy to verify that \(J^\dagger_m = J_{-m}\). Moreover, one may easily check
that the normal ordering is unecessary for the cases where \(m\neq 0\), since
\begin{equation}
\begin{split}
J_m := &\sum_r :\psi_{r-m}^\star \psi_r: \\
= & - \sum_{r<m} \psi_r \psi_{r-m}^\star + \sum_{r>m} \psi_{r-m}^\star \psi_r  \qquad m\geq 0\; .
\end{split}
\end{equation} 	
Finally, it is easy to check that
\begin{equation}
J_m \ket{\bm{0}}= \bra{\bm{0}} J_{-m}=0 \qquad \forall \ m\geq 0\; .
\end{equation}

Since the positive and negative modes commute among themselves, we can
define the following operators
\begin{equation}
\begin{split}
\bm{J}_+({\bf t}) & :=\sum_{m>0}t_m  J_m\; ,\quad  \bm{t}=(t_1, t_2, \dots )\\
\bm{J}_-({\bf t}) & :=\sum_{m<0}t_m  J_m\; ,\quad  \bm{t}=(t_{-1}, t_{-2}, \dots )\; ,
\end{split}
\end{equation}
where one mat think of these parameters as an infinite set of
times. Finally, these two objects satisfy the identity
\begin{equation}
\label{v-ope}
\boxed{
  e^{\bm{J}_{+}(\bm{t})}e^{\bm{J}_{-}(\bm{t}')}
  = \exp\left( \sum_{n>0} n t_n t'_{-n} \right) e^{\bm{J}_{-}(\bm{t}')} e^{\bm{J}_{+}(\bm{t})}}
\end{equation}

Observe that \(\bm{J}_+(\bm{t}) \ket{\bm{0}}=0 \) and \(
\bra{\bm{0}} \bm{J}_-(\bm{t})=0 \) so that
\begin{equation}
e^{\bm{J}_+(\bm{t})} \ket{\bm{0}}= \ket{\bm{0}} \; , \qquad
 \bra{\bm{0}} e^{\bm{J}_-(\bm{t})} = \bra{\bm{0}}\; .
\end{equation}
Furthermore, the states \(e^{\bm{J}_-(\bm{t})}\ket{\bm{0}}\) can be
expanded in terms of basis states with the Schur polynomials as their
coefficients~\cite{Alexandrov:2012tr}. In fact, here we assume that
the definitions of the Schur polynomials are given in terms of the
free fermions we have been working so far. 

Let us define the bosonic states
\begin{equation}
|\vec{k}\rangle \equiv |k_1, k_2, \dots \rangle = \prod_{j=1}^{\infty}
(J_{-j})^{k_j}|0\rangle\; .
\end{equation}

The two bases \(|\vec{k}\rangle\) and
\(|\lambda\rangle\) are related by~\cite{Marino:2005sj}
\begin{subequations}
\begin{equation}
|\lambda\rangle = \sum_{\vec{k}}
\frac{\chi_\lambda[C(\vec{k})]}{z_{\vec{k}}} |\vec{k}\rangle\; ,
\end{equation}
and inverse 
\begin{equation}
|\vec{k}\rangle = \sum_{\lambda} \chi_\lambda[C(\vec{k})]
|\lambda\rangle \; ,
\end{equation}
\end{subequations}
where the coefficients \(z_{\vec{k}} = \prod_{j=1}^\infty k_j!
j^{k_j}\), and \(\chi_\lambda[C(\vec{k})]\) is the character of the
symmetric group \(\mathfrak{S}_n\) evaluated in the
\(\lambda\)-representation and conjugacy class
\(C(\vec{k})\). Moreover, the sum is performed over the partitions
\(\lambda\) with number of boxes the level of the state
\(|\vec{k}\rangle\), that is
\begin{equation}
|\lambda|= \sum_{j\geq 1} j k_j\; . 
\end{equation}
In fact, one can also write the bosonic states themselves in a
partition-like notation. First, we observe that \(J_m J_n = J_n J_m\),
for \(m\) and \(n\) negative. Therefore, we organize the state
in decreasing order, that is \(J_m^{k_m} J_n^{k_n}\), \(m<n\)
\begin{equation}
  |\vec{k}\rangle = \prod_{j>0}(J_{-n_j})^{k_j}\ket{\bm{0}} =
J_{-1}^{k_1} J_{-2}^{k_2} \cdots J_{-n}^{k_n}\cdots \ket{\bm{0}} 
\end{equation}
Now, each operator \(J_{-n}\) corresponds to a new row, and its index
\(n\) gives the length of the corresponding row. One useful feature is
that the bosonic and fermionic spaces are decomposed into subspaces
with fixed number os boxes.
Given the state \(\vec{k}\), one can write
the corresponding Young diagram with the Python's codes described in
the appendix~\ref{app:snippets}.

\subsubsection{Frobenius character formula}
The explicit formulas for the characters can be obtained from the
following expression~\cite{Fulton:2004uyc} give the charactes
\begin{subequations}
\begin{equation}
\label{eq:frobenius}
\chi_{\lambda}[C(\vec{k})] = \left[ 
\Delta(\vec{x}) \prod_{j=1}^{r} P_j(\vec{x})^{k_j} 
\right]_{(\ell_1, \dots, \ell_m)}\; ,
\end{equation}
with
\begin{equation}
\Delta(\vec{x}) = \prod_{i<j}(x_i-x_j)\; , \quad P_j(\vec{x})=\sum_{i=1}^mx_i^j\; .
\end{equation}
\end{subequations}
The vector \(\vec{k}\) have a finite number of nonzero
components, \(\vec{k} = (k_1, \dots, k_r)\).Additionally, we have
the auxiliary coordinates \(\vec{x} = (x_1, \dots, x_m)\) and \(m\) is the
number rows of the Young diagram \(\lambda = (\lambda_1, \dots ,
\lambda_m)\). Furthermore, we also need the following coefficients
\begin{equation}
\ell_j = \lambda_j + m -j\; ,\ \ j=1,\dots, m\; .
\end{equation}
Given a generic polynomial \(f(\vec{x})\), we have used the following notation
 \begin{equation}
\left. f(\vec{x})\right|_{(\ell_1, \dots, \ell_m)}:=\text{coeff. of}
\ \ x_1^{\ell_1}\cdots x_m^{\ell_m}\; .
\end{equation}

The advantage of this construction is that now we have
all ingredients to translate bosonic to fermionic partition
states. For example, we have the following characters
\[
\begin{array}{cr} 
    |\lambda|  = 1 &
    (1,0,\dots)\\
    \hline \hline
    \chi_{(1)} & 1
\end{array}
\]
\[
\begin{array}{crr} 
    |\lambda| = 2 & (0,1,0,\dots) & (2,0,\dots) \\
    \hline\hline
    \chi_{(1,1)} & -1 & 1 \\ 
    \chi_{(2)} & 1 & 1 \\ 
\end{array}
\]
\[
\begin{array}{crrr} 
   |\lambda| = 3 & (0,0,1,\dots) &  (1,1,0,\dots) & (3,0,0,\dots) \\
  \hline \hline
    \chi_{(1,1,1)} & 1 & -1 & 1 \\ 
    \chi_{(2,1)} & -1 & 0 & 2 \\ 
    \chi_{(3)} & 1 & 1 & 1 \\ 
\end{array}
\]
For the calculation of any character, one can use the code in 
the appendix~\ref{app:snippets}.

%%%%%%%%%%%%%%%%%%%%%%%%%%%%%%%%%%%%%%%%%%%%%%%%%%
%%%%%%%%%%%%%%%%%%%%%%%%%%%%%%%%%%%%%%%%%%%%%%%%%%

\subsection{Schur Functions}

We define the Schur functions to be the coefficients in the expansions
\begin{equation}
\label{evol}
\begin{split}
    e^{\bm{J}_-(\bm{t})} \ket{\bm{0}}
    & = \sum_\lambda s_{\lambda}(\bm{t}) \ket{\lambda} \\
    \bra{\bm{0}} e^{\bm{J}_+(\bm{t})} 
    &= \sum_\lambda s_{\lambda}(\bm{t})\bra{\lambda}  
\end{split}
\end{equation}
Using the orthogonality of the vector states \(|\mu\rangle\), we have
\begin{equation}
\boxed{
s_{\lambda}(\bm{t}) = 
\bra{\lambda} e^{\bm{J}_-(\bm{t})} \ket{\bm{0}}}
\end{equation}

More generally, one can consider two Young diagrams
\(\mu \subseteq \lambda\), then
\begin{equation}
  \label{eq:skschur}
\begin{split}
e^{\bm{J}_-(\bm{t})} \ket{\mu}
& = \sum_\lambda s_{\lambda/\mu}(\bm{t})\ket{\lambda} \\
\bra{\mu} e^{\bm{J}_+(\mathbf{t})} 
&= \sum_\lambda s_{\lambda/\mu}(\bm{t}) \bra{\lambda}\; .
\end{split}
\end{equation}
Consequently
\begin{equation}
\boxed{
s_{\lambda/\mu}(\bm{t}) = 
 \bra{\lambda}e^{\bm{J}_-(\bm{t})}|\mu\rangle}\;. 
\end{equation}
These functions are called \emph{Skew Schur functions}. 
In the particular case where \(\mu =\bm{0}\), we recover the
expressions for the ordinary Schur functions.

Let us consider the first coefficients in the expansion
\begin{equation}
e^{\bm{J}_-(\bm{t})} \ket{\bm{0}}=\left(
1 + t_{-1}J_{-1} + t_{-2}J_{-2} + \cdots 
\frac{1}{2}t_{-1}t_{-1}J_{-1} J_{-1} + \cdots\; .
\right)
\ket{\bm{0}}
\end{equation}

Diagrammatically, we have
\begin{center}
    \begin{tikzpicture}
    \node at (-2.3,.5) {\(e^{\bm{J}_-(\bm{t})} \ket{\bm{0}} =\)};
    \draw[line width=0.5mm][rotate=45]   (0,0) -- (0,1.7);
    \draw[line width=0.5mm][rotate=45]   (0,0) -- (1.7,0);
    \end{tikzpicture}
    \begin{tikzpicture}
    \node at (-1.4,.5) {\(+ s_{(1)}\)};
    \draw [line width=0.3mm][rotate=45][fill=lightgray]  (0,0) rectangle (.5,.5);
    \draw[line width=0.5mm][rotate=45]   (0,0) -- (0,1.7);
    \draw[line width=0.5mm][rotate=45]   (0,0) -- (1.7,0);
    \end{tikzpicture}		
    \begin{tikzpicture}
    \node at (-1.4,.5) {\(+ s_{(2)}\)};
    \draw [line width=0.3mm][rotate=45][fill=lightgray] (0,0) rectangle (1,.5);
    \draw[line width=0.3mm][rotate=45]   (0.5,0) -- (0.5,.5);
    \draw[line width=0.5mm][rotate=45]   (0,0) -- (0,1.7);
    \draw[line width=0.5mm][rotate=45]   (0,0) -- (1.7,0);
    \end{tikzpicture}		
    \begin{tikzpicture}
    \node at (-1.5,.5) {\(- s_{(1,1)}\)};
    \draw [line width=0.3mm][rotate=45][fill=lightgray] (0,0) rectangle (.5,1);
    \draw[line width=0.3mm][rotate=45]   (0,.5) -- (.5,.5);
    \draw[line width=0.5mm][rotate=45]   (0,0) -- (0,1.7);
    \draw[line width=0.5mm][rotate=45]   (0,0) -- (1.7,0);
    \end{tikzpicture}
    \begin{tikzpicture}
    \node at (-1.4,.5) {\(+ s_{(3)}\)};
    \draw [line width=0.3mm][rotate=45][fill=lightgray] (0,0) rectangle (1.5,0.5);
    \draw[line width=0.3mm][rotate=45]   (.5,0) -- (.5,.5);
    \draw[line width=0.3mm][rotate=45]   (1,0) -- (1,.5);
    \draw[line width=0.5mm][rotate=45]   (0,0) -- (0,1.7);
    \draw[line width=0.5mm][rotate=45]   (0,0) -- (1.7,0);
    \end{tikzpicture}
    \begin{tikzpicture}
    \node at (-1.6,.5) {\(+ s_{(1,1,1)}\)};
    \draw [line width=0.3mm][rotate=45][fill=lightgray] (0,0) rectangle (.5,1.5);
    \draw[line width=0.3mm][rotate=45]   (0,.5) -- (.5,.5);
    \draw[line width=0.3mm][rotate=45]   (0,1) -- (.5,1);
    \draw[line width=0.5mm][rotate=45]   (0,0) -- (0,1.7);
    \draw[line width=0.5mm][rotate=45]   (0,0) -- (1.7,0);
    \end{tikzpicture}
    \begin{tikzpicture}
    \node at (-1.5,.5) {\(+ s_{(2,1)}\)};
    \draw [line width=0.3mm][rotate=45][fill=lightgray] (0,0) rectangle (.5,1);
    \draw [line width=0.3mm][rotate=45][fill=lightgray] (0,0) rectangle (1,.5);	
    \draw[line width=0.3mm][rotate=45]   (.5,0) -- (.5,.5);		
    \draw[line width=0.5mm][rotate=45]   (0,0) -- (0,1.7);
    \draw[line width=0.5mm][rotate=45]   (0,0) -- (1.7,0);
    \node at (1.4,.5) {\(+ \cdots\)};
    \end{tikzpicture}							
\end{center}
where the coefficients are
\begin{equation}
s_{(1)}=t_{-1}\; , \quad s_{(2)}=\frac{1}{2}t_{-1}^2 + t_{-2} \; ,\quad s_{(1,1)}
=\frac{1}{2}t_{-1}^2- t_{-2}\; ,\quad  \dots
\end{equation}

The Miwa coordinates are given by \(t_j = \frac{1}{j}\sum_{a = 1}^j
x_a^m\) where \(\bm{x} = (x_1, x_2, \dots, x_m)\) is a set of \(m\)
variables. If we write these Schur functions in terms of \(\bm{x}\),
these functions become symmetric polynomials, and we call them
\emph{Schur polynomials}. In terms of these coordinates, the Schur
polynomials are characters of the \(\lambda\) representations
evaluated at the diagonal matrix \(\mathrm{diag}(x_1, x_2, \dots,
x_m)\). In any case, in the text we use both terminologies as
synonyms.

%%%%%%%%%%%%%%%%%%%%%%%
%%%%%%%%%%%%%%%%%%%%%%%

\subsection{Fermion-Boson correspondence}

The fermion boson correspondence is a relation between basis of
Hilbert space~\cite{Cordes:1994fc, Marino:2005sj}. Let start this
definition with some basic facts. The complete homogeneous symmetric
polynomials are defined by
\begin{equation}
  \label{eq:com-hom-sym-pol}
\exp\left(\sum_{k\geq 1} T_k z^k\right): = \sum_{k\geq 0} h_k(\bm{T}) z^k \; ,
\end{equation}
with \(h_0=1\) and we also define \(h_{-k}=0\  \forall\ k>0\). Explicitly we have 
\begin{equation}
  h_k(\bm{T}) = \sum_{k_1+2k_2 +\dots = k}\frac{T_1^{k_1}}{k_1!} \frac{T_2^{k_2}}{k_2!}\cdots
  \; .
\end{equation}	

We can use this polynomials to express the \emph{coherent state}
\begin{equation}
\begin{split}
  \ket{\mathcal{T}}& = \exp\left( \sum_{n\geq 1} t_{n} J_{-n}\right) \ket{\bm{0}} = 
  \exp\left.\left(\sum_{n\geq 1} T_n z^n\right)\right|_{z=1}  \ket{\bm{0}} \\
& = \sum_{k\geq 0} \sum_{k_1+2k_2+\cdots = k}\frac{t_{1}^{k_1}}{k_1!}
\frac{t_{2}^{k_2}}{k_2!}\cdots \ket{k_1, k_2, k_3\dots}
\end{split}
\end{equation}
where we have used that \(T_k \equiv t_k J_{-k}\), \(\bm{t}_- = (t_1,
t_2 \dots)\) , \(\vec{k} = (k_1, \dots, k_r)\) and \(\sum_j j k_j =
|\lambda|\).  One should observe that in contrast with our previous
definition, the parameters \(\bm{t}_-\) have been changed to
\(\bm{t}_+\equiv \bm{t}\). Since there is no intrinsic meaning in those objects,
this transformation is just a relabeling.

Using the expansion of \(\ket{\vec{k}}\) in terms of characters
coefficients, we can write the explicit expression for the Schur as
\begin{equation}
\begin{split}
  \ket{\mathcal{T}}& = \sum_{\lambda} s_\lambda(\bm{t})\ket{\lambda} \\
    & = \sum_\lambda\left( \sum_{k\geq 0} \sum_{k_1+2k_2+\cdots = k}\frac{t_{1}^{k_1}}{k_1!}
    \frac{t_{2}^{k_2}}{k_2!}\cdots \chi_\lambda [C(\vec{k})] \right) \ket{\lambda}
\end{split}
\end{equation}
This formula is very hard to calculate explicily, but it is useful to know that
if we change the coefficients \(\chi_\lambda\), we have a new type of fermion boson
correspondence. 

If we now identify \(t_{j} \equiv \frac{1}{j}P_j(x)\), we can
write all expressions in terms of the coordinates \(\vec{x}\). And we
conclude that
\begin{equation}
\langle \mathcal{T}| \vec{k} \rangle= \langle \vec{k} |\mathcal{T}\rangle = P_{\vec{k}}(x)\; .
\end{equation}

Moreover, we have seen that Schur polynomials are defined as
\(s_{\lambda}(t)=\langle \mathcal{T}| \lambda \rangle\), and since
Schur and Newton polynomials are related by the Frobenius formula, it
is easy to see that there is a relation between the fermionic and
bosonic states. This relation is the Fermion-Boson correspondence.

%
%\section{Schur Polynomials}

%%%%%%%%%%%%%%%%%%%%%%%%%%%%%%
%%%%%%%%%%%%%%%%%%%%%%%%%%%%%%

\subsection{Complete symmetric functions.} 

Using the definition of the homogeneous complete symmetric
polynomial~(\ref{eq:com-hom-sym-pol}), the Miwa coordinatees \(t_j =
j^{-1}\sum_i x_i^j\), it is easy to show that \(h_i(\vec{x})\) are
given by
\begin{equation}
\sum_{k=0}^\infty h_k(\vec{x}) z^k := \prod_{i=1}^m\frac{1}{1-x_i z}\; ,
\end{equation} 
for \(\vec{x}=(x_1, x_2, \dots, x_m)\). Explicitly, we have~\cite{Prasad2018}
\begin{equation}
\begin{split}
    h_k(\vec{x}) & = \sum_{\vec{j}_k^{\ \leq}} x_{j_1}\cdots x_{j_k}\; , \quad 
    \vec{j}_k^{\ \leq}:=\{ (j_1, \dots, j_k)\ |\ 1 \leq j_1\leq \cdots \leq j_k\leq m \}\; .
\end{split}
\end{equation}
Consider, for simplicity, the case \(m=3\), then
\begin{subequations} 
\begin{equation}
\begin{split} 
    h_1&  = x_1 + x_2 + x_3\\
    h_2&  = x^2_1 + x^2_2 + x^2_3+x_1 x_2 + x_1 x_3 + x_2 x_3\\
    h_3&  = x^3_1 + x^3_2 + x^3_3+x_1^2 x_2 + +x_1^2 x_3 + +x_2^2 x_1 + x_2^2 x_3 
    +x_3^2 x_1 +x_3^2 x_1 + x_1 x_2 x_3\; .
\end{split}			
\end{equation}
\end{subequations} 

These functions satisfy some important
properties~\cite{Foda:2009zz}. For example
\begin{subequations}
\begin{equation}
\label{hiden01}
h_k(\vec{x}) = h_k(\hat{x}_i) + x_{i} h_{k-1}(\vec{x}) \; ,
\end{equation}
where \(\hat{x}_i\) is the set \(\vec{x}\) without the coordinate
\(x_i\). The proof of this identity is straightforward, for example,
we can pick an arbitrary coordinate \(x_i\) and split the functions
\(h_k(\vec{x})\) in two terms, the first term does not have the chosen
coordinate \(x_i\) and it is simply \(h_k(\hat{x}_i)\), whilst the
second term, \(\tilde{h}_k(\vec{x})\) contains all dependence on the
coordinate \(x_m\). That is
\begin{equation}
\begin{split}
h_k(\vec{x}) & = \sum_{1\leq j_1 \leq \cdots \hat{i}\leq j_k\leq m}
x_{j_1}\cdots \hat{x}_i \cdots x_{j_k} + \tilde{h}_k(\vec{x}) \\ & =
h_k(\hat{x}_i) + \tilde{h}_k(\vec{x})
\end{split}
\end{equation}
where \(\hat{x}_i\) denotes omission of this coordinate.  By
construction, the function \(\tilde{h}_k(\vec{x}) \) has at least one
coordinate \(x_i\), then we can collect this term and write it as
\begin{equation}
\tilde{h}_k(\vec{x}) = x_i \sum_{1\leq j_1 \leq \cdots \hat{i}\leq j_{k-1}\leq m} x_{j_1}
\cdots x_{j_{k-1}}= x_m  \tilde{h}_{k-1}(\vec{x})\; .
\end{equation}
Putting all these facts together, we see that the
identity~(\ref{hiden01}) is satisfied.
\end{subequations}

Using~(\ref{hiden01}) we obtain a second useful identity
\begin{equation}
\begin{split}
    h_k(\hat{x}_i) - h_k(\hat{x}_l) & = h_k(\hat{x}_i) - x_i
    h_{k-1}(\vec{x}) -( {h_k(\hat{x}_l) - x_l h_{k-1}(\vec{x}) }
    )\\ & = (x_l- x_i )h_{k-1}(\vec{x})
\end{split}
\end{equation}

\subsection{Schur polynomials}

Starting with the definitions of \cite{Alexandrov:2012tr}, we consider
a generic partition \(\lambda = (\lambda_1, \lambda_2, \dots, \lambda_m)\) and
the infinite vector \(\mathbf{t} = \{ t_1, t_2, t_3, \dots \}\). One
defines the Schur polynomials \(s_\lambda(\mathbf{t})\) as
\begin{equation}
\label{jti01}
s_\lambda(\mathbf{t}) = \det \mathbf{H}\ \qquad \mathbf{H}_{ij} =
h_{\lambda_i -i+j}\; ,\ \ \ i,j=1,\dots, m\; ,
\end{equation}
where \(h_k\) is the \emph{complete symmetric function of degree k} defined by  
\begin{subequations} 
\begin{equation}
\exp\left(  \sum_{k\geq 1} t_k z^k\right): = \sum_{k\geq 0} h_k(\mathbf{t}) z^k \; ,
\end{equation}
with \(h_0=1\) and we also define \(h_{-k}=0\  \forall\ k>0\). Explicitly we have 
\begin{equation}
h_k(\mathbf{t}) = \sum_{k_1+2k_2 +\dots = k}\frac{t_1^{k_1}}{k_1!} \frac{t_2^{k_2}}{k_2!}\cdots
\end{equation}	
Some examples are 
\begin{equation}
\begin{split}
  & h_1(\mathbf{t})=t_1\qquad	h_2(\mathbf{t})=\frac{1}{2}t_1^2 + t_2 \\
  & h_3(\mathbf{t})=\frac{1}{6}t_1^3 + t_1 t_2 + t_3\qquad
  h_4(\mathbf{t})=\frac{1}{24}t_1^4 + \frac{1}{2}t_2^2 
  + \frac{1}{2} t_1^2t_2 +t_1t_3 + t_4 \\
  & h_5=\frac{t_1^5}{120}+\frac{1}{6} t_2 t_1^3+\frac{1}{2} t_3 t_1^2+\frac{1}{2} t_2^2 t_1+t_4 t_1+t_2 t_3+t_5\\
  & h_6= \frac{t_1^6}{720}+\frac{1}{24} t_2 t_1^4+\frac{1}{6} t_3 t_1^3+\frac{1}{4} t_2^2 t_1^2
  +\frac{1}{2} t_4 t_1^2+t_2 t_3 t_1+t_5 t_1+\frac{t_2^3}{6}+\frac{t_3^2}{2}+t_2 t_4+t_6
\end{split}	
\end{equation}
\end{subequations}
For 1-row diagrams with \(n\) boxes \(\lambda = (n)\), we have 
\begin{equation} 
s_{(n)}(\mathbf{t}) = h_n ( \mathbf{t} )\; . 
\end{equation}

In an equivalent manner, one can define the Schur polynomials as 
\begin{equation}
\label{jti02}
s_\lambda(\mathbf{t}) = \det \mathbf{E}\ \qquad \mathbf{E}_{ij} = e_{\lambda_i^t -i+j}\; ,\ \ \  
i,j=1,\dots, \lambda_1\; ,
\end{equation}
where \(\lambda^t\) is the transpose Young diagram, and
\(e_k\) is the \emph{elementary symmetric function of degree k} defined by  
\begin{subequations}
\begin{equation}
\exp\left(  \sum_{k\geq 1} (-1)^{k-1}  t_k z^k\right): = \sum_{k\geq 0} e_k(\mathbf{t}) z^k \; .
\end{equation}
and these functions satisfy
\begin{equation}
e_j(\mathbf{t}) = (-1)^j h_j(-\mathbf{t})	\; .
\end{equation}
For example, the first four terms are
\begin{equation}
\begin{split}
  & e_1(\mathbf{t})=t_1 \qquad	e_2(\mathbf{t})=\frac{1}{2}t_1^2 - t_2 \\
  & e_3(\mathbf{t})=\frac{1}{6}t_1^3 - t_1 t_2 + t_3\qquad
  e_4(\mathbf{t})=\frac{1}{24}t_1^4 + \frac{1}{2}t_2^2 - \frac{1}{2} t_1^2t_2 +t_1t_3 - t_4\\
  & e_5=\frac{t_1^5}{120}-\frac{1}{6} t_2 t_1^3+\frac{1}{2} t_3 t_1^2+\frac{1}{2} t_2^2 t_1-t_4 t_1-t_2 t_3+t_5\\
  & e_6= \frac{t_1^6}{720}-\frac{1}{24} t_2 t_1^4+\frac{1}{6} t_3 t_1^3+\frac{1}{4}
  t_2^2 t_1^2-\frac{1}{2} t_4 t_1^2-t_2 t_3 t_1
  +t_5 t_1+\frac{t_3^2}{2}+t_2 t_4-t_6-\frac{t_2^3}{6}
\end{split}	
\end{equation}
\end{subequations}
For 1-column diagrams with \(n\) boxes, \(\lambda = (1^n)\), we have 
\begin{equation} 
s_{(1^n)}(\mathbf{t}) = e_n ( \mathbf{t} )\; . 
\end{equation}
Equations (\ref{jti01}) and (\ref{jti02}) are known as
\emph{Jacobi-Trudi identities}.

More generally, we have the relation between the 
Young diagram \(\lambda\) and its transpose \(\lambda'\)
\begin{equation}
s_{\lambda}(\mathbf{t})	= (-1)^{|\lambda|} s_{\lambda'}(-\mathbf{t})	\; .
\end{equation}

%%%%%%%%%%%%%%%%%%%
%%%%%%%%%%%%%%%%%%%

\subsection{Skew Schur polynomials:} Consider now two Young diagrams,
\(\lambda = (\lambda_1, \dots, \lambda_m)\) and \(\mu = (\mu_1, \dots,
\mu_n)\), with \(m\geq n\) and \(\mu \subseteq \lambda\) as a Young
diagram, that is \(\{\mu_i \leq \lambda_i \ | \ i = 1\dots, m\}\).
The \emph{Skew Schur polynomials} \(s_{\lambda/\mu}(\mathbf{t}) \) are
defined by the generalized Jacobi-Trudi identity
\begin{equation}
s_{\lambda/\mu}(\mathbf{t}) = \left. \det\left( h_{\lambda_i-\mu_j
  -i+j} \right)\right|_{i,j=1,\dots, m} = \left. \det\left(
e_{\lambda^t_i-\mu^t_j -i+j} \right)\right|_{i,j=1,\dots, \lambda_1} \; .
\end{equation}
We can also write these functions in terms of the ordinary Schur
functions as
\begin{equation}
s_{\lambda/\mu}(\mathbf{t}) = \sum_{\nu} c_{\mu \nu}^\lambda s_{\nu}(\mathbf{t}) \; ,
\end{equation}
where \(c_{\mu \nu}^\lambda\) are known as \emph{Littlewood-Richardson
coefficients} and are determined by
\begin{equation}
s_{\mu}(\mathbf{t}) s_{\nu}(\mathbf{t})= \sum_{\lambda} c_{\mu
  \nu}^\lambda s_{\lambda}(\mathbf{t})\; .
\end{equation}


\subsection{Special cases} Let us consider some special cases:

\subsubsection{Case 1:} Let us first specialize to \(t_k=z^k/k\). It is easy to see that
\begin{equation}
\label{symm}
\begin{split}
  & h_1=z \quad h_2 = z^2 \quad  h_3 = z^3 \quad h_4 = z^4 \quad 
  h_5 = z^5 \quad h_6 = z^6\quad \dots\quad  h_p = z^p  \dots \\ 
  & e_1=z \quad e_2 = 0 \quad e_3 = 0 \quad e_4 = 0 \quad
  e_5 = 0 \quad e_6 = 0 \quad \dots \quad e_p = 0 \dots
\end{split}
\end{equation}
One can easily see that this result is consistent using the coordinate \(\vec{x}\). 
\begin{equation}
\label{sympol}
t_k = \frac{1}{k}\sum_{i=1}^m x_i^k \; , \quad \vec{x}=(x_1, \dots, x_m)\; ,
\end{equation}
We have just seen that the Schur polynomials form a basis for the space of
symmetric functions. In these variables, we have~\cite{Prasad2018}
\begin{equation}
\begin{split}
    h_k(\vec{x}) & = \sum_{\vec{j}_k^{\ \leq}} x_{j_1}\cdots x_{j_m}\; , \quad 
    \vec{j}_i^{\ \leq}:=\{ (j_1, \dots, j_k)\ |\ 1 \leq j_1\leq \cdots \leq j_k\leq m \}\\
    e_k(\vec{x}) & = \sum_{\vec{j}_k^{\ <}} x_{j_1}\cdots x_{j_m}\; , \quad 
    \vec{j}_i^{\ <}:=\{ (j_1, \dots, j_k)\ |\ 1 \leq j_1< \cdots < j_k\leq m \}
\end{split}
\end{equation}
Consider, for simplicity, the case \(m=3\), then
\begin{subequations} 
\begin{equation}
\begin{split} 
  h_1&  = x_1 + x_2 + x_3\\
  h_2&  = x^2_1 + x^2_2 + x^2_3+x_1 x_2 + x_1 x_3 + x_2 x_3\\
  h_3&  = x^3_1 + x^3_2 + x^3_3+x_1^2 x_2 + +x_1^2 x_3 + +x_2^2 x_1 + x_2^2 x_3 
  +x_3^2 x_1 +x_3^2 x_1 + x_1 x_2 x_3
\end{split}			
\end{equation}
and 
\begin{equation}
\begin{split} 
  e_1&  = x_1 + x_2 + x_3\\
  e_2&  = x_1 x_2 + x_1 x_3 + x_2 x_3\\
  e_3&  = x_1 x_2 x_3
\end{split}			
\end{equation}		
\end{subequations}
Assuming the expansion (\ref{sympol}), the values \(t_k=z^k/k\)
correspond to \(\vec{x} =(z, 0, \dots, 0)\). In this case, it is easy
to see that the symmetric polynomials reduce to (\ref{symm}).

Therefore, if \(t_k = z^k/k\), we have \(h_k=z^k\ \forall \ k\in
\mathbb{N}\) and \(e_l=0\ \forall \ l\geq 2 \). Consequently,
the nontrivial Schur polynomials are row diagrams,
and they are given by
\begin{equation}
  s_{(n)}  (\{z^k / k\}) = z^n\; . 
\end{equation}
Any other diagram vanishes.

We can use the Skew Schur polynomials. We say that two Young diagrams are
interlaced \(\mu \prec \lambda\) if \(\{ \lambda_i \geq \mu_i \geq \lambda_{i+1}
\ | \ i = 1, \dots, m-1\}\). Therefore, one can see that~\cite{Okounkov2001}
\begin{equation}
  \label{eq:schur:case1}
s_{\lambda/\mu}(\{z^k / k\}) =
\left\{ 
\begin{array}{ll}
z^{|\lambda| - |\mu|} & \lambda \succ \mu \\
0 & \lambda \nsucc \mu \\
\end{array}
\right.
\end{equation}
Here is a list of a few Skew Schur polynomials at the values \(t_k = z^k /k\): 
\[
\begin{array}{lllll}
  \hline
    s_{(1)/\emptyset} = z   & s_{(1)/(1)} = 1 &&&	\\
    s_{(2)/\emptyset} = z^2 & s_{(2)/(1)} = z & s_{(2)/(2)} = 1 &\\  
    s_{(2,1)/\emptyset} = 0 & s_{(2,1)/(1)} = z^2 & s_{(2,1)/(1,1)}=z  & s_{(2,1)/(2)}=z & s_{(2,1)/(2,1)}=1  \\ 
    s_{(3,2,1)/\emptyset} = 0 & s_{(3,2,1)/(2,1,0)} = z^3 & s_{(3,2,1)/(3,2,0)}=z & s_{(3,2,1)/(2,2,1)}= z &
    s_{(3,2,1)/(1,0,0)}=0 \\ 
  \hline
\end{array}
\]


\subsubsection{Case 2:} We now consider the case
\(\mathbf{t}_0=(t, 0, \dots)\). In this case we have
\begin{equation}
h_1=e_1=t\; , \ \ h_2=e_2= \frac{1}{2} t^2 \; , \ h_3=e_3 =
\frac{1}{3!} t^3 \; , \ \ h_4 =e_4 = \frac{1}{4!} t^4 \; \ \dots
\ \ h_p = e_p = \frac{1}{p!} t^p \; \dots
\end{equation}
where \(t\) is an arbitrary parameter, but next section we set
\(t=\sqrt{q}\). In this particular case, some Skew Schur polynomials are
\footnote{These results do not seem to agree with the results in
section 2.2 of~\cite{okounkov2003symmetric} and  
equations (2.13) and (2.14) of~\cite{Maeda:2004is}}
\[
\begin{array}{lllll}
  \hline
    s_{(1)/\emptyset} = z   & s_{(1)/(1)} = 1 &&&	\\
    s_{(2)/\emptyset} = z^2/2 & s_{(2)/(1)} = z & s_{(2)/(2)} = 1 &\\  
    {\color{red} s_{(2,1)/\emptyset} = z^3/3} &
      s_{(2,1)/(1)} = z^2 & s_{(2,1)/(1,1)}=z  & s_{(2,1)/(2)}=z & s_{(2,1)/(2,1)}=1  \\ 
    {\color{red} s_{(3,2,1)/\emptyset} = z^6/45} & s_{(3,2,1)/(2,1,0)} = z^3
      & s_{(3,2,1)/(3,2,0)}=z & s_{(3,2,1)/(2,2,1)}= z &
    {\color{red} s_{(3,2,1)/(1,0,0)}= 2 z^5 / 15} \\ 
  \hline
\end{array}
\]
Where we highlight some differences with the previous case. 

% \begin{equation}
% \begin{array}{llllll}
%   \hline
%     s_{(1)} = t     &&&&&	\\ 
%     s_{(2)} = \frac{t^2}{2} & s_{(1,1)}=\frac{t^2}{2}  &&&& \\ 
%     s_{(3)} = \frac{t^3}{6} &
%     s_{(2,1)} = \frac{t^3}{3} & s_{(1,1,1)}=\frac{t^3}{6} &&& \\  
%     s_{(4)} = \frac{t^4}{24} &
%     s_{(3,1)} = \frac{t^4}{8} & s_{(2,1,1)}=\frac{t^4}{8} & s_{(2,2)}=\frac{t^4}{12} &
%     s_{(1^4)}=\frac{t^4}{24} & \\  
%     s_{(5)} = \frac{t^5}{120} &	s_{(4,1)} = \frac{t^5}{30} & s_{(3,2)}=\frac{t^5}{24} &
%     s_{(311)}=\frac{t^5}{20} & s_{(2,2,1)}=\frac{t^5}{24} 
%     & s_{(2,1,1,1)}=\frac{t^5}{30}\\  
%     s_{(1^5)}=\frac{t^5}{120} &&&&& \\ 
%     s_{(6)} = \frac{t^6}{720} & s_{(5,1)} = \frac{t^6}{144} & s_{(4,2)} = \frac{t^6}{80}
%     & s_{(4,1,1)} = \frac{t^6}{72} & s_{(3,3)} = \frac{t^6}{144} & s_{(3,2,1)} = \frac{t^6}{144} \\ 
%     s_{(3,1,1,1)} = \frac{t^6}{72} & s_{(2,2,2)} = \frac{t^6}{72} & s_{(2,2,1,1)} = \frac{t^6}{80}
%     & s_{(2,1,1,1,1)} = \frac{t^6}{144} & s_{(1^6)} = \frac{t^6}{720} &\\ 
%   \hline
% \end{array}
% \end{equation}

One can write these results generically as:
\begin{equation}
\boxed{s_{\lambda/\mu} = C_{\lambda/\mu}\ t^{|\lambda| - |\mu|}}	
\end{equation}
where 
\begin{equation}
C_{\lambda/\mu} = \left.\det h_{\lambda_i - \mu_j -i +j}\right|_{\mathbf{t}=(1, 0, \dots )}\; .
\end{equation}

%%%%%%%%%%%%%%%%%%%%%%%%%%%%%%
%%%%%%%%%%%%%%%%%%%%%%%%%%%%%%


%
%\section{Transfer matrices}

This section is a review of the transfer matrix
formalism~\cite{Alexandrov:2012tr, Okounkov:2003sp, Justin2008}.  Let
us now define the following operators
\begin{subequations}
\begin{equation}
\Gamma_\pm(z):=\exp \phi_\pm(z)\; ,
\end{equation}		
where 
\begin{equation}
\phi_\pm(z):= \sum_{\pm n\geq 1}\frac{z^{n}}{n} J_{n}\; .
\end{equation}
These are the creation and annihilation components of the bosonic
vertex operator \(\mathcal(z) = \exp \phi(z)\).  By analogy with
statistical mechanics systems, we may consider that the time evolves
in a series of small steps, and for this reason we say that
\(\Gamma_\pm(x_i)\) is a transfer operator.
\end{subequations}

Moreover, observe that 
\begin{subequations}
\begin{equation}
\begin{split}
\prod_{a=1}^m \Gamma_\pm(x_a) & = \exp \sum_{a = 1}^m \phi_\pm(x_a) \\
	& = \exp \sum_{a = 1}^m  \sum_{\pm n\geq 1}\frac{x_a^{n}}{n} J_{n}\\
	& = \exp  \sum_{n\geq 1} \left(\sum_{a = 1}^m  \frac{x_a^{n}}{n} \right) J_{n}
\end{split}
\end{equation}
and using the Miwa coordinates (\ref{sympol}), we finally obtain
\begin{equation}
\label{oper}
\boxed{\prod_{i\geq 1} \Gamma_\pm(x_a) = e^{J_\pm (\mathbf{t})}}\; .
\end{equation}	
\end{subequations}

Using the expansion of skew Schur polynomials~(\ref{eq:skschur}), we have 
\begin{subequations}
\begin{equation}
   e^{\bm{J}_-(\bm{t}}\ket{\mu} \equiv 
   \prod_{a=1}^m \Gamma_-(x_a) \ket{\mu}
   = \sum_\lambda s_{\lambda/\mu}(\mathbf{t})\ket{\lambda}\; .
\end{equation}

One now can use the special case 1 we discussed above where \(\vec{x}
= (z,0,0, \dots) \), that as we alrealdy seen, corresponds to the
choice \(t_n = \tfrac{x^n}{n}\), \(h_k = z^k\) and \(e_l = 0\) for \(l > 1\).
Moreover, using~\ref{eq:schur:case1}, we can see that
\begin{equation}
  \begin{split}
    e^{\bm{J}_-(\{z^k/k\})} \ket{\mu} & \equiv \Gamma_-(z) \ket{\mu}
    = \sum_\lambda s_{\lambda/\mu}(\{z^k/k\})\ket{\lambda}\\
    & = \sum_{\lambda \succ \mu}z^{|\lambda| - |\mu|}\ket{\lambda}\; .
  \end{split}
\end{equation}
Therefore, the particular case where \(z=1\) gives the beautiful expression
\begin{equation}
    \Gamma_-(1) \ket{\mu} = \sum_{\lambda \succ \mu}\ket{\lambda}\; .
\end{equation}
It is also easy to show that
\begin{equation}
    \Gamma_+(1) \ket{\mu} = \sum_{\lambda \prec \mu}\ket{\lambda}\; .
\end{equation}
\end{subequations}
For example, the case \(\mu = \bm{\emptyset}\) gives an expansion of
one row diagrams. An in fact, \(\Gamma_+(1)^n \ket{\bm{0}}\) gives an
expansion of interlaced Young diagrams with a maximum of \(n\) rows. 
For more information on this topic, see~\cite{Okounkov:2003sp, Okounkov2001}

%%%%%%%%%%%%%%%%%%%%
%%%%%%%%%%%%%%%%%%%%


\subsection{Sum over partitions}

The tau-function is a function defined on the orbit of the vacuum
\(\ket{\bm{0}}\) under the action of the group element \(g\in
\widehat{GL}(\infty)\). It is defined as the Vacuum Expectation Value (VEV)
\begin{equation}
  \tau(\mathbf{t},g):= \bra{\bm{0}} e^{\mathbf{J}_+(\mathbf{t}) } g
  \ket{\bm{0}}\; , \qquad g\in \widehat{GL}(\infty)\; .
\end{equation}
This object is intrumental in the study of several integrable systems.
Additionally, we have seen that the coherent state
\(e^{\mathbf{J}_-(\mathbf{t})} \ket{\bm{0}}\) can be written in terms
of Schur polynomials. Therefore, we want to study some aspects of
these expectation values.

Let us first define the following fermionic state
\begin{equation}
\begin{split}
   \ket{\widetilde{\Psi}_0} & = \lim_{d\to \infty} \left[ 
   \sum_{\bm{a}^{<}_d;\bm{b}^{<}_d}
   q^{\frac{1}{2}\sum_{i=1}^d (a_i+b_i)} \prod_{i=1}^d \psi^\star_{a_i}\psi_{-b_i}
   \right] |\mathtt{half}\rangle \\ 
   & = q^{L_0/2} \left( 1 + \sum_{a_1; b_1}\psi^\star_{a_1} \psi_{-b_1} + 
   \sum_{\bm{a}^{<}_2 ; \bm{b}^{<}_2} \psi^\star_{a_2} \psi_{-b_2} \psi^\star_{a_1}\psi_{-b_1} + \cdots 
   \right) |\mathtt{half}\rangle 
\end{split}
\end{equation}
where \(\bm{a}^{<}_d = \{ (a_1, \dots, a_d)\ | \ 0<a_1<\cdots < a_d\}\)
and \(\bm{b}^{<}_d = \{ (b_1, \dots, b_d)\ | \ 0<b_1<\cdots < b_d\}\).
We can immediately write this state as
\begin{equation}
\ket{\Psi_0}_f = \sum_{\lambda} q^{\frac{1}{2}|\lambda|} \ket{\lambda} \; ,
\quad \textrm{where}\quad 
\ket{\lambda} = \prod_{j=1}^{d}\psi^\star_{a_j} \psi_{-b_j} \ket{\bm{0}}\; .
\end{equation}
Diagrammatically, this state is the sum
\begin{equation}
\ket{\widetilde{\Psi}_0} =  \ket{\bm{0}} + 
\ytableausetup{centertableaux, smalltableaux}
q^{1/2} |\ydiagram{1} \rangle +
q  \left|\ydiagram{1,1} \right\rangle +
q |\ydiagram{2} \rangle+ 
q^{3/2} \left| \ydiagram{3} \right\rangle+ q^{3/2} \left| \ydiagram{1,2} \right\rangle+
q^{3/2} \left| \ydiagram{1,1,1} \right\rangle+\cdots
\end{equation}
From the fermion-boson correspondence we have discussed earlier, one
can write this state in terms of bosonic states \(\ket{\vec{k}}\) as
\begin{equation}
   \ket{\widetilde{\Psi}_0}  = q^{L_0/2} \sum_{\lambda} \ket{\lambda} = 
q^{L_0/2} \sum_{\vec{k}} \sum_{\lambda} \frac{\chi_\lambda[C(\vec{k})]}{z_{\vec{k}}} \ket{\vec{k}}
\end{equation}
where \(|\lambda| = \sum_j j k_j\).

Using that under the fermion-boson correspondence, the bosonic states might also be
written in terms of conjugacy classes of the symmetric group
\(\mathfrak{S}_n\) (or the partitions of \(n\)), let us write 
\begin{equation}
  \label{eq:bos:states}
  \begin{split}
    \ket{\vec{k}}  & = \ket{(k_1, k_2, \dots, k_m)}  
    = \ket{(1^{k_1} 2^{k_2} 3^{k_3}\dots m^{k_m})} \\
    & \equiv | (\underbrace{m, \dots, m}_{k_m},m-1, \dots ,3, \underbrace{2, \dots, 2}_{k_2},
  \underbrace{1, \dots, 1}_{k_1}, 0\dots)\rangle \equiv
  \kket{\lambda}\; ,
  \end{split}
\end{equation}
where we use the double notation \(\kket{\lambda}\) to denote the bosonic partition. 
These states are not normalized, and in fact, they satisfy
\begin{equation}
\bracket{\vec{k}'}{\vec{k}} = z_{\vec{k}} \delta_{\vec{k}', \vec{k}}\; .
\end{equation}
Therefore, one can define the state
\begin{equation}
\ket{\Psi_0} = q^{L_0/2} \sum_{k\geq 0} \sum_{\vec{k}} \frac{1}{\sqrt{z_{\vec{k}}}} \ket{\vec{k}}\; . 
\end{equation}
Using the bosonization formula, we can write this state as
\begin{equation}
  \ket{\Psi_0} = q^{L_0/2} \sum_{\lambda} \sum_{\vec{k}}
  \frac{1}{\sqrt{z_{\vec{k}}}} \chi_\mu[C(\vec{k})] \ket{\lambda}\; ,
\end{equation}
and the number of boxes in the partition is given by \(|\lambda| = \sum_j j k_j\).
Using the explicit expression for the characters, we have that the
first terms in this expansion are
\begin{equation}
  \begin{split}
\ket{\Psi_0} & = \ket{\bm{0}} + 
\ytableausetup{centertableaux, smalltableaux}
q^{1/2} |\ydiagram{1} \rangle +
q \sqrt{2} |\ydiagram{2} \rangle+ 
q^{3/2} \left[ \frac{1}{\sqrt{6}} (1 + \sqrt{2} + \sqrt{3})\left| \ydiagram{3} \right\rangle + \right. \\
& + \left. \frac{1}{\sqrt{6}} (2 - \sqrt{2}) \left| \ydiagram{1,2} \right\rangle+
 \frac{1}{\sqrt{6}} (1 + \sqrt{2} - \sqrt{3})\left| \ydiagram{1,1,1} \right\rangle\right] + \\
& + q^2 \left[ \frac{1}{12}\left( \sqrt{6} + 3 \sqrt{2} + 4 \sqrt{3}
  + 12\right) \left| \ydiagram{4} \right\rangle 
  + \frac{1}{4}\left( \sqrt{6} - \sqrt{2} \right)\left| \ydiagram{1,3} \right\rangle \right.\\
  & + \left. \frac{1}{2}\left( \sqrt{2} - \frac{2\sqrt{3}}{3} + \frac{\sqrt{6}}{3} \right)
  \left| \ydiagram{2,2} \right\rangle + 
  \frac{1}{4}\left( \sqrt{6} - \sqrt{2} \right) \left| \ydiagram{1,1,2} \right\rangle + 
  \frac{1}{12}\left( \sqrt{6} + 3 \sqrt{2} + 4 \sqrt{3}
  - 12\right) \left| \ydiagram{1,1,1,1} \right\rangle + \dots
  \right]
  \end{split}
\end{equation}

\subsection{Vacuum Expectation Values and Euler Formula}

From the orthonormality of the fermionic plane partition states, we have
\begin{equation}
\begin{split}
  Z & = \bracket{\widetilde{\Psi}_0}{\widetilde{\Psi}_0}\\
  & = 1 + q + 2 q^2 + 3 q^3 + 5 q^4 + \dots = \sum_{n\geq 0} p(n) q^n \\
  & = \prod_{n=1}^\infty \frac{1}{1-q^n}\; .
\end{split}
\end{equation}
that is the Euler formula for integer partitions. It is also easy to see that 
\begin{equation}
\begin{split}
  Z & = \bracket{\Psi_0}{\Psi_0}\\
  & = 1 + q + 2 q^2 + 3 q^3 + 5 q^4 + \dots = \sum_{n\geq 0} p(n) q^n \\
  & = \prod_{n=1}^\infty \frac{1}{1-q^n}\; .
\end{split}
\end{equation}

%\(\clubsuit\) It would be very interesting to see what is the subgroup of
%\(\widehat{\mathfrak{gl}}(\infty)\) that keeps this ground state invariant. 
% See Macmahon function here~\cite{Okounkov:2003sp, Okounkov2001, Dijkgraaf:2008ua}.


%
%\section{Deformed fermions and Hall-Littlewood polynomials}

The deformed fermions and their relations with the Hall-Littlewood
polynomials were defined in~\cite{Jing1991, Jing1995}.
See also~\cite{Foda:2008hn, Wheeler:2010vmq, Sulkowski:2008mx}.

We start with the deformed anticommutation relations.
\begin{equation}
\begin{split} 
\{\psi_m, \psi_n\} &= Q\left( \psi_{m + 1} \psi_{n-1} + \psi_{n+1} \psi_{m-1} \right)\\
\{\psi_m^\star, \psi_n^\star\} &= Q\left( \psi_{m-1}^\star \psi_{n+1}^\star
+ \psi_{n-1}^\star \psi_{m+1}^\star \right)\\
\{\psi_m, \psi_n^\star\} &= Q\left( \psi_{m-1} \psi_{n-1}^\star
+ \psi_{n+1}^\star \psi_{m+1} \right) + (1-Q)^2\delta_{m,n}
\end{split}	
\end{equation}
where \(Q\in \mathbb{C}\). Observe that the indices \(m, n\) are now
integers. The limit \(Q=0\) gives the usual charged free fermions we
have just considered.

The Fock space generated by these operators is similar to the
undeformed one. In particular, we have the vacuum \(\ket{\bm{0}}\)
which is annihilated as
\begin{equation}
\psi_m \ket{\bm{0}} = \psi_{n}^\star \ket{\bm{0}} = 0 \qquad  m < 0 \quad n \geq 0\; .
\end{equation}
Similarly, one can define the charged vacua \(\ket{-\ell}\), with \(\ell > 0\), as
\begin{equation}
\begin{split}
\ket{-\ell} = \psi^\ast_{-\ell}\cdots \psi^\ast_{-1}\ket{\bm{0}}\; .
\end{split}
\end{equation}
These states satisfy the relations
\begin{equation}
  \psi_m\ket{-\ell} = \left\{
\begin{array}{ll}
 0 & \quad m < -\ell\\
 \ket{-\ell + 1} & \quad m = -\ell\\
\end{array}
\right.
\end{equation}
The partition states \(\ket{\lambda}\) with \(\lambda = (\lambda_1,
\dots, \lambda_\ell)\) are defined via
\begin{equation}
\ket{\lambda} = \psi_{\lambda_1 - 1} \cdots \psi_{\lambda_\ell - \ell}\ket{-\ell}\; .
\end{equation}

%%%%%%%%%%%%%%%%%%%%%%%%%%%%%%%%%%%%%%%%%%%%%%%%%%
%%%%%%%%%%%%%%%%%%%%%%%%%%%%%%%%%%%%%%%%%%%%%%%%%%

\subsection{Deformed Heisenberg Algebra}

We also have the deformed Heisenberg algebra
\begin{equation}
[H_m, H_n] = \frac{m}{1-Q^{|m|}}\delta_{m+n}
\end{equation}
where
\begin{equation}
\begin{split}
  H_m = & \frac{1}{1-Q}\sum_{j\in \mathbb{Z}} :\psi_j\psi^\star_{j+m}:\qquad \forall\ m>0\\
  H_m = & \frac{1}{(1-Q)(1-Q^{|m|})}\sum_{j\in \mathbb{Z}} :\psi_j\psi^\star_{j+m}:\qquad \forall\ m<0\; .
\end{split}
\end{equation}
And in the limit \(Q=0\) we recover the components of the current
\(J(z)\).  As in the undeformed case, the normal ordering is not very
important in the case \(m\neq n\). It can also be shown that
\begin{equation}
\label{eq:hpsi}
[\psi_n, H_m] = - \psi_{n -m}\qquad 
[\psi^\star_n, H_m] = \psi^\star_{n + m}\; .
\end{equation}

From these anticommutation relations, it can be easily shown that
\begin{equation}
  \label{eq:modulus}
   \bra{\bm{0}} \prod_{j\geq 1} H_{j}^{k_j} H_{-j}^{k_j}\ket{\bm{0}}
    = \prod_{j\geq 1} \frac{ k_j! j^{k_j}}{(1 - Q^j)^{k_j}}
\end{equation}
therefore
\begin{equation}
    \bracket{\vec{k}}{\vec{k}'}
    = \prod_{j\geq 1} \frac{ k_j! j^{k_j}}{(1 - Q^j)^{k_j}} \delta_{\vec{k}\vec{k}'}
    = z_{\vec{k}} \delta_{\vec{k}\vec{k}'}\; .
\end{equation}
where \(z_{\lambda} = z_{\vec{k}} = \prod_j k_j! j^{k_j}/(1 -
Q^j)^{k_j}\) and the partition \(\lambda = (\lambda_1, \lambda_2,
\dots, \lambda_m)\) can be represented as \(\lambda = (1^{k_1}
2^{k_2} 3^{k_3}\dots)\). Finally, we also have
\begin{equation}
\ket{\vec{k}} = \prod_{j\geq 1} H_{-j}^{k_j}\ket{\bm{0}}\; . 
\end{equation}
In other words, the expression~(\ref{eq:modulus}) describes an inner product. 

%%%%%%%%%%%%%%%%%%%%%%%%%%%%%%%%%%%%%%%%%%%%%%%%%%
%%%%%%%%%%%%%%%%%%%%%%%%%%%%%%%%%%%%%%%%%%%%%%%%%%

\subsection{Vertex Operators}

And from these operators, the vertex operators (or transfer matrices)~\cite{Jing1991, Jing1995}
are \(\Gamma_\pm(w,Q)=\exp (\phi_\pm(w,Q))\), where
\begin{equation}
\phi_\pm(w,Q) = \sum_{m>0} \frac{1-Q^m}{m} w^{\mp m} H_{\pm n}\; .
\end{equation}
We can also write
\begin{equation}
    \Gamma_-(w,Q)  = \exp\left( \sum_{m>0} \frac{1-Q^m}{m} w^{\mp m} H_{\pm n} \right)
    = \sum_{n\geq 0} \Gamma_n w^n  \; ,
\end{equation}
where
\begin{equation}
  \Gamma_n = \sum_{k_1 + 2k_2 + \dots = n} \frac{1}{z_{\vec{k}}} H_{-1}^{k_1} H_{-2}^{k_2}\cdots
    \equiv \sum_{\lambda \vdash n} \frac{1}{z_{\lambda}} \bm{H}_{-\lambda} \; , \qquad 
 \bm{H}_{\lambda} = H_{-\lambda_1} H_{-\lambda_2}\cdots H_{-\lambda_m}\; , 
\end{equation}
and the sum is taken over all partitions of \(n\).

As in the underformed case, we have
\begin{equation}
 \begin{split}
    \prod_{j=1}^N \Gamma_- (x_j, Q) &
    = \exp \left( \sum_{m>0} \sum_{j=1}^N \frac{(1 - Q^m)}{m} z_j^m H_{-m}
    \right) \\ 
    & = \exp \left( \sum_{m>0} (1 - Q^m) t_m H_{-m} \right) \equiv e^{\bm{J}_-(\bm{t}, Q)}\; .
 \end{split}
\end{equation}
where \(t_m = \frac{1}{m} \sum_{j=1}^N x_j\) are the usual Miwa
coordinates.


%%%%%%%%%%%%%%%%%%%%%%%%%%%%%%%%%%%%%%%%%%%%%%%%%%
%%%%%%%%%%%%%%%%%%%%%%%%%%%%%%%%%%%%%%%%%%%%%%%%%%

\subsection{Hall-Littlewood functions}

Now that we have seen that the construction of the deformed fermion is
completely analogous to the undeformed case, we can define the The
Hall-Littlewood polynomials as
\begin{equation}
e^{\bm{J}_-(\bm{t}, Q)} \ket{\bm{0}} = 
  \prod_{j=1}^N \Gamma_-(x_j, Q)  \ket{\bm{0}}
  = \sum_{\lambda}P_{\lambda}(x_1, x_2, \dots, x_N;Q) \ket{\lambda}\; .
\end{equation}
where the \(P_\lambda\) are known as the Hall-Littlewood polynomials.
One can define \(T_m = (1- Q^m) t_m H_m\), and using the definition of
the complete homogeneous polynomials, we have
\begin{equation}
  \sum_{\lambda}P_{\lambda}(x_1, x_2, \dots, x_N;Q) \ket{\lambda} =
  \sum_{n\geq 0}\sum_{k_1 + 2k_2 + \dots =n} \left( \prod_{i\geq 1}
  \frac{(1-Q^i)^{k_i} t_i^{k_i}}{k_i!} H_{-i}^{k_i}\right)  \ket{\bm{0}}\; .
\end{equation}
As in the undeformed case, the Fock space is decomposed in terms of
its levels, or number of boxes in the Young diagram.

It may be interesting to calculate the first terms of this expansion
to see that everything works accordingly. Let us write the the first
few terms.
\begin{equation}
  \begin{split}
    e^{\bm{J}_-(\bm{t}, Q)} \ket{\bm{0}} & =\left( 1 + (1 -
    Q)t_{1}H_{-1} + (1 - Q^2)t_{2}H_{-2} + \frac{(1 -
      Q)^2}{2}t_{1}^2H_{-1}^2 + \cdots \right) \ket{\bm{0}} \\ & =
    P_{(0)}(\bm{t}, Q) \ket{\bm{0}} +
    \ytableausetup{centertableaux, smalltableaux} P_{(1)}(\bm{t}, Q) |\ydiagram{1} \rangle +
    P_{(2)}(\bm{t}, Q) |\ydiagram{2} \rangle+ P_{(1,1)}(\bm{t}, Q)
    \left|\ydiagram{1,1} \right\rangle+ \cdots
    \end{split}
\end{equation}

\emph{n=0:} We obviously have \(P_{(0)} =1\). 

\emph{n=1:} Now we have 
\begin{equation}
    P_{(1)}(\bm{t}, Q) |\ydiagram{1} \rangle = (1 - Q)t_{1}H_{-1} \ket{\bm{0}} \; .
\end{equation}
Using now that 
\begin{equation}
  \begin{split}
  H_{-1} \ket{\bm{0}} & = \frac{1}{(1-Q)^2} \left(\sum_{j < 1} +
  \sum_{j \geq 1} \right) \psi_j\psi^\star_{j-1} \ket{\bm{0}} =
  \frac{1}{(1-Q)^2} \sum_{j < 1} \psi_j\psi^\star_{j-1} \ket{\bm{0}}
  \\ & = \frac{1}{(1-Q)^2} \left ( \sum_{j < -1}
  \psi_j\psi^\star_{j-1} + \psi_0\psi^\star_{-1} +
  \psi_{-1}\psi^\star_{-2}\right)\ket{\bm{0}}
  \end{split}
\end{equation}
where we have used that \(\psi_m^\star \ket{\bm{0}} = 0 \) for \(m\geq
0\). Moreover, let us identify the one box state as \(|\ydiagram{1}
\rangle = \psi_0 \psi_{-1}^\ast \ket{\bm{0}}\).  From the
anticommutation relations, it is easy to see that \(\psi_j
\psi_{j-1}^\ast \ket{\bm{0}} = 0\) for \(j < -1\), as a result of
\(\psi_j\ket{\bm{0}} = 0\). Therefore
\begin{equation}
  H_{-1} \ket{\bm{0}} = \frac{1}{(1-Q)^2} \left(|\ydiagram{1}\rangle
  + \psi_{-1}\psi^\star_{-2}\ket{\bm{0}} \right)\; .
\end{equation}
From the anticommutation relations, we have
\begin{equation}
  \psi_{-1}\psi^\star_{-2}\ket{\bm{0}} = Q\psi_{-1}^\star\psi_{0}\ket{\bm{0}}\; ,
\end{equation}
then
\begin{equation}
  H_{-1} \ket{\bm{0}} = \frac{1}{(1-Q)^2} \left(|\ydiagram{1}\rangle
  + Q \psi_{-1}^\star\psi_{0}\ket{\bm{0}} \right)\; .
\end{equation}
From the anticommutation relations, we have
\begin{equation}
  \begin{split}
    \psi_{-1}^\star\psi_{0}\ket{\bm{0}} & = \left( - \psi_0
    \psi_{-1}^\star + Q \psi_{-1} \psi_{-2}^\star + Q \psi_0^\star
    \psi_1 \right) \ket{\bm{0}} \\ & = - |\ydiagram{1}\rangle +
    Q^2\psi_{-1}^\star\psi_{0}\ket{\bm{0}} + Q \psi_0^\star \psi_1
    \ket{\bm{0}} \; .
  \end{split}
\end{equation}
that is
\begin{equation}
 (1 - Q^2)\psi_{-1}^\star\psi_{0}\ket{\bm{0}} = - |\ydiagram{1}\rangle
  + Q \psi_0^\star \psi_1 \ket{\bm{0}}
\end{equation}

Now, we have 
\begin{equation}
  \psi_0^\star \psi_1 \ket{\bm{0}} = Q \psi_0 \psi_{-1}^\star \ket{\bm{0}} +
  Q \psi^\star_1 \psi_2 \ket{\bm{0}}
\end{equation}
The second term vanishes since its commutation generates terms with
positive indices \(\psi_n^\star\) acting on the vacuum. We can see
this fact using the identity (4.2.7) of~\cite{Wheeler:2010vmq},
\begin{equation}
  \psi_m^\star \psi_n = Q \psi_{n-1} \psi_{m-1} + (Q^2 - 1)
  \sum_{i\geq 0} Q^i \psi_{n+i} \psi_{m+i}^\star + (1-Q)\delta_{m,n}\;
  .
\end{equation}
Therefore
\begin{equation}
  \psi_0^\star \psi_1 \ket{\bm{0}} = Q |\ydiagram{1}\rangle\; , 
\end{equation}
and we conclude that
\begin{equation}
  \begin{split}
    (1 - Q^2)\psi_{-1}^\star\psi_{0}\ket{\bm{0}} & = -
    |\ydiagram{1}\rangle + Q \psi_0^\star \psi_1 \ket{\bm{0}} \\ & = -
    (1-Q^2) |\ydiagram{1}\rangle\quad \Rightarrow \quad
    \psi_{-1}^\star\psi_{0}\ket{\bm{0}} = - |\ydiagram{1}\rangle\; .
  \end{split}
\end{equation}
And we finally conclude that 
\begin{equation}
H_{-1} \ket{\bm{0}} = \frac{1}{(1- Q)}|\ydiagram{1}\rangle \qquad
\Rightarrow \qquad P_{(1)} = t_1\;,
\end{equation}
that is the expected result. 

\emph{n=2:} The calculation of higher state partitions can be
daunting. So, it might be usefult to consider some identities
originally proved in~\cite{Foda:2008hn}. Moreover, it is easier
to calculate the elements
\begin{align}
  \ytableausetup{centertableaux, smalltableaux} 
  P_{(2)}(\bm{t}, Q) & = \frac{1}{\langle \ydiagram{2}  | \ydiagram{2} \rangle} \langle \ydiagram{2}|
  \left((1 - Q^2) t_2 H_{-2} + \frac{(1-Q^2)}{2} t_1^2 H_{-1}^2\right) \ket{\bm{0}} \\
  P_{(1,1)}(\bm{t}, Q) & = \frac{1}{\left\langle \ydiagram{1,1}  | \ydiagram{1,1} \right\rangle}
  \left\langle \ydiagram{1,1}\right|
  \left((1 - Q^2) t_2 H_{-2} + \frac{(1-Q^2)}{2} t_1^2 H_{-1}^2\right) \ket{\bm{0}} \; .
\end{align}

First of all, it has been shown that given two partition states \(\ket{\lambda}\) and \(\ket{\mu}\),
they satisfy
\begin{equation}
  \bracket{\lambda}{\mu} = b_{\lambda} \delta_{\mu\nu}\qquad
  b_\lambda = \prod_{i=1}^\infty \prod_{j=1}^{p_i(\lambda)} (1 - Q^j)\; ,
\end{equation}
where \(p_i(\lambda)\) is the number of rows of size \(i\). Therefore, the normalization factors
are
\begin{equation}
  \langle \ydiagram{2}  | \ydiagram{2} \rangle = 1 - Q\qquad
    \left\langle \ydiagram{1,1}  | \ydiagram{1,1} \right\rangle= (1 - Q)(1- Q^2)\; .
\end{equation}
We now have four different terms to calculate.

\paragraph{1.} We first have
\begin{equation}
  \begin{split}
\langle \ydiagram{2}| H_{-2} \ket{\bm{0}} & = \bra{-1} \psi_1^\ast H_{-2} \ket{\bm{0}} = 
\bra{-1} [\psi_1^\ast, H_{-2}] \ket{\bm{0}} \\
& = \bra{-1} \psi_{-1}^\ast\ket{\bm{0}} = \bra{0} \psi_{-1} \psi_{-1}^\ast \ket{\bm{0}} = 1\; , 
  \end{split}
\end{equation}
where we have used that
\begin{equation}
 \bra{-\ell} H_{-m} = 0 \quad \forall m \geq 1\; ,
\end{equation}
and the commutation relations~(\ref{eq:hpsi}). The proof that
\(\bra{0} \psi_{-1} \psi_{-1}^\ast \ket{\bm{0}} = 1\)  is not very difficult, but a proof can be found in
equations~(4.2.17) and (4.2.18) of~\cite{Wheeler:2010vmq}.

\paragraph{2.} Now
\begin{equation}
  \begin{split}
    \langle \ydiagram{2}| H_{-1}^2 \ket{\bm{0}} & = \bra{-1} \psi_1^\ast H_{-1}^2 \ket{\bm{0}} = 
    \bra{-1} [\psi_1^\ast, H_{-1}^2] + H_{-1}^2  \psi_1^\ast\ket{\bm{0}} \\
    & = \bra{-1} H_{-1}\underbrace{[\psi_1^\ast, H_{-1}]}_{\psi^\ast_0} +
    \underbrace{[\psi_1^\ast, H_{-1}]}_{\psi_0^\ast} H_{-1}  \ket{\bm{0}}
     = \bra{-1} [\psi_0^\ast, H_{-1}]  \ket{\bm{0}} =  \\
     &= \bra{-1} \psi_{-1}^\ast  \ket{\bm{0}} = \bra{\bm{0}} \psi_{-1} \psi_{-1}^\ast  \ket{\bm{0}} =  1\; .
  \end{split}
\end{equation}


\paragraph{3.} Moving to the next diagram
\begin{equation}
  \begin{split}
    \left\langle \ydiagram{1,1}\right| H_{-2} \ket{\bm{0}}
    & = \bra{-2} \psi_{-1}^\ast \psi_0^\ast H_{-2} \ket{\bm{0}} =
    \bra{-2} \psi_{-1}^\ast [\psi_0^\ast, H_{-2}] \ket{\bm{0}} =
    \bra{-2} \psi_{-1}^\ast \psi_{-2}^\ast\ket{\bm{0}}\\
    & = \bra{-2} \left[-\psi_{-2}^\ast \psi_{-1}^\ast
      + Q\left( \psi_{-3}^\ast \psi_0^\ast +
      \psi_{-2}^\ast \psi_{-1}^\ast \right)  \right]\ket{\bm{0}}
    = - (1 - Q) \bra{-2} \psi_{-2}^\ast \psi_{-1}^\ast \ket{\bm{0}}\\
  \end{split}
\end{equation}
where we have used the anticommutation relations. Finally, using the relation
\(\bra{-\ell} \psi_{-\ell}^\ast = \bra{-\ell + 1}\), we find
\begin{equation}
    \left\langle \ydiagram{1,1}\right| H_{-2} \ket{\bm{0}}
    = - (1 - Q) \bra{-1} \psi_{-1}^\ast \ket{\bm{0}} = - (1 - Q)\; .
\end{equation}


\paragraph{4.} Finally
\begin{equation}
\begin{split}
\left\langle \ydiagram{1,1}\right| H_{-1}^2 \ket{\bm{0}}
& = \bra{-2} \psi_{-1}^\ast \psi_0^\ast H_{-1}^2 \ket{\bm{0}} =
\bra{-2} \psi_{-1}^\ast [\psi_0^\ast, H_{-1}^2] \ket{\bm{0}} \\
& = \bra{-2} \psi_{-1}^\ast \left( H_{-1}[\psi_0^\ast, H_{-1}] +
    [\psi_0^\ast, H_{-1}] H_{-1}\right) \ket{\bm{0}} \\
& = \bra{-2} \psi_{-1}^\ast \left( H_{-1}\psi_{-1}^\ast + \psi_{-1}^\ast H_{-1}\right) \ket{\bm{0}} \\
& = \bra{-2} \left([\psi_{-1}^\ast, H_{-1}]\psi_{-1}^\ast + \psi_{-1}^\ast \psi_{-1}^\ast H_{-1}\right) \ket{\bm{0}} 
 = \bra{-2} \left(\psi_{-2}^\ast\psi_{-1}^\ast + Q \psi_{-2}^\ast \psi_{0}^\ast H_{-1}\right) \ket{\bm{0}} \\
& = \bra{-2} \left(\psi_{-2}^\ast \psi_{-1}^\ast + Q \psi_{-2}^\ast [\psi_{0}^\ast, H_{-1}]
\right) \ket{\bm{0}}
= \bra{-2} \left(\psi_{-2}^\ast \psi_{-1}^\ast + Q \psi_{-2}^\ast \psi_{-1}^\ast
\right) \ket{\bm{0}} = 1 + Q
\\
\end{split}
\end{equation}

Putting all these facts together, we find that
\begin{subequations}
\begin{equation}
\begin{split}
P_{(2)} & = \frac{1}{(1 - Q)}\left( (1 - Q^2) t_2 + \frac{(1 - Q)^2}{2}t_1^2 \right)\\
 & =(1 + Q) t_2 + \frac{(1 - Q)}{2}t_1^2 
  \end{split}
\end{equation}
and 
\begin{equation}
\begin{split}
P_{(1,1)} & = \frac{1}{(1 - Q)(1 - Q^2)}\left( - (1 - Q)(1 - Q^2) t_2 + \frac{(1 + Q)(1 - Q)^2}{2}t_1^2 \right)\\
 & = \frac{1}{2}t_1^2  - t_2 = e_2\; .
  \end{split}
\end{equation}
\end{subequations}


%
%\section{Topological strings}

Moreover, we also have 
\begin{equation}
 S(N, q, \{\bm{t}\}, \{\bar{\bm{t}}\}) 
 = \sum_{\lambda \subseteq [N, \infty)} s_\lambda(\bm{t})  s_\lambda(\bar{\bm{t}}) 
\end{equation}
that is a trivial case of a hypergeometric tau function~\cite{orlov:2001}. 

The partition function of D-branes on \(\mathbb{C}^3\)
is similar, see\cite{Saulina:2004da}. 
\begin{equation}
  Z_D(q, a) = Z(q) \prod_{n=1}^\infty \frac{1}{1- e^a q^{n - 1/2}}\; .
\end{equation}
We  can also write this extra factor in terms of Schur polynomials as
with \(x= (e^a, 0, 0, \dots)\) and \(y = (q^{1/2}, q^{3/2}, \dots)\),
therefore
\begin{equation}
\prod_{n=1}^\infty \frac{1}{1- e^a q^{n - 1/2}} = \sum_\lambda s_{\lambda}(e^a, 0, 0, \dots)
s_{\lambda}(q^{1/2}, q^{3/2}, \dots)\; .
\end{equation}
And using the homogeneity of the Schur polynomials, we have
\begin{equation}
s_{\lambda}(e^a, 0, 0, \dots) = e^{|\lambda|a} s_{\lambda}(1, 0, 0, \dots)\; . 
\end{equation}
One can also observe that this equation is the homogeneous polynomial at the point
\(z = e^a\) and \(x=(q^{1/2}, q^{3/2}, \dots)\).

In fact, the partition function for \(M\) D-branes in this geometry is given by 
\begin{equation}
Z_{MD} = M(q) \prod_{i<j}(1 - q^{N_i - N_j}) \prod_{i=1}^M \prod_{n_i=1}^{N_i} (1 - q^{n_i})^{-1}\; .
\end{equation}
For just one brane, we recover something similar to Bogoliubov. So, it seems
that \(M\) counts the types of bosons, while \(N_0\) is the soliton sector. 



%\addcontentsline{toc}{section}{References}
\printbibliography

\end{document}
 
