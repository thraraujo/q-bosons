\documentclass[a4paper,11pt]{amsart}

\usepackage{/home/thiago/.config/dot-files/latex/globaldef}

% DEFINITIONS

%%%%%%%%%%%%%%%%%%%%%%%%
% ADDITIONAL PACKAGES
%%%%%%%%%%%%%%%%%%%%%%%% 
\usepackage{parskip}
\usepackage{tikz} % Drawing package
\usepackage[backend=biber, style=alphabetic]{biblatex}
\usepackage{listings}
\usepackage{ytableau}

\usetikzlibrary{intersections,decorations.text} % this is to make the cover
\usetikzlibrary{matrix, arrows.meta} % this is to improve diagrams

\setlength\parindent{0pt}

%\setdefaultlanguage[variant=brazilian]{portuguese}
%\setdefaultlanguage{english}

%%%%%%%%%%%%%%%%%%%%%%%%
% METADATA
%%%%%%%%%%%%%%%%%%%%%%%% 

%\usepackage{framed} 


% Official manual defines the options
\hypersetup{ 
	pdftitle={integrable systems, programming and so on},
	pdfsubject={High Energy Physics and so on}, 
	pdfauthor={author},
	pdfkeywords={gauge; susy; strings; fields; cft; python},
	colorlinks=true, % false: color frames ; true: color links
    linkcolor=myPurple, % Color of internal links (sections, pages and so on)
    citecolor=myPurple, % color for bibliographical citations
    urlcolor =myPurple, % color for linked URL
    linktocpage=true % link page to table of contents
}


% Options for listings package I saw the sniipet below here: 
% https://stackoverflow.com/questions/3175105/inserting-code-in-this-latex-document-with-indentation
\lstset{frame=tb,
  language=Python,
  aboveskip=3mm,
  belowskip=3mm,
  showstringspaces=false,
  columns=flexible,
  basicstyle={\small\ttfamily},
  numbers=none,
  numberstyle=\tiny\color{myPurple},
  keywordstyle=\color{myRed},
  commentstyle=\color{myBlue},
  stringstyle=\color{myPurple!80},
  breaklines=true,
  breakatwhitespace=true,
  tabsize=3
}

%%%%%%%%%%%%%%%%%%%%%%%% 
% COLORS
%%%%%%%%%%%%%%%%%%%%%%%% 
% see palette here: https://github.com/enkia/tokyo-night-vscode-theme
\definecolor{myPurple}{RGB}{90, 74, 120}
\definecolor{myBlue}{RGB}{15, 75, 110}
\definecolor{myRed}{RGB}{191,97,106}
\definecolor{myDarkGray}{RGB}{216, 222, 233}
\definecolor{myLightGray}{RGB}{236, 239, 244}

\definecolor{c1}{RGB}{129, 162, 193}%  {15, 75, 110} % myBlue
\definecolor{c2}{RGB}{216, 222, 233} % myDarkGray
\definecolor{c3}{RGB}{236, 239, 244} % myLightGray
\definecolor{c4}{RGB}{59, 66, 82}
\definecolor{c5}{RGB}{76, 86, 106}

% This color is a framed requirement
\definecolor{shadecolor}{RGB}{236, 239, 244}

%%%%%%%%%%%%%%%%%%%%%%%%
% THEOREM
%%%%%%%%%%%%%%%%%%%%%%%% 

\newtheorem{theorem}{Theorem}
\newtheorem{corollary}{Corollary}
\newtheorem{proposition}{Proposition}
\newtheorem{conjecture}{Conjecture}
\newtheorem{lemma}{Lemma}
\newtheorem{example}{Example}
\newtheorem{remark}{Remark}

%%%%%%%%%%%%%%%%%%%%%%%%
% MACROS
%%%%%%%%%%%%%%%%%%%%%%%% 

\DeclarePairedDelimiter{\bra}{\langle}{\rvert}
\DeclarePairedDelimiter{\ket}{\lvert}{\rangle}
\DeclarePairedDelimiter{\bbra}{\langle\!\langle}{\rvert}
\DeclarePairedDelimiter{\kket}{\lvert}{\rangle\!\rangle}
\DeclarePairedDelimiterX{\bracket}[2]{\langle}{\rangle}{#1\vert#2}
\DeclarePairedDelimiterX{\bbracket}[2]{\langle\!\langle}{\rangle\!\rangle}{#1\vert#2}


\bibliography{bib-database.bib}

\usepackage{amsaddr} % affiliation in the first page

\begin{document}

%%%%%%%%%%%%%%%%%%%%%%%%%%%%%%%%%%%%%%%%%%%%%%%%%%
%%%%%%%%%%%%%%%%%%%%%%%%%%%%%%%%%%%%%%%%%%%%%%%%%%

\title{\(Q\)-Boson model and relations with integrable hierarchies}

\author{Thiago Araujo}

\address{\noindent 
Instituto de Física Teórica, UNESP-Universidade Estadual Paulista,
R. Dr. Bento T. Ferraz 271, Bl. II, Sao Paulo 01140-070, SP, Brazil\\
\&
Instituto de Física, Universidade de S\~ao Paulo,
Rua do Matão Travessa 1371, 05508-090 São Paulo, SP. Brazil
}

\email{\texttt{\href{tr.araujo@unesp.br}{tr.araujo@unesp.br}}}

\begin{abstract}
In this work, we study aspects of the \(Q\)-boson model, and 
its relations with integrable hierarchies. 
\end{abstract}

%\date{\today}
\keywords{Integrability, Bethe, Schur}
\subjclass[2020]{82B20, 82B23}
\maketitle

\setcounter{tocdepth}{2}
\tableofcontents

%*%*%*%*%*%*%*%*%*%*%*%*%*%*%*%*%*%*%*%*%*%*%*%*%*
\section{Introduction}
%*%*%*%*%*%*%*%*%*%*%*%*%*%*%*%*%*%*%*%*%*%*%*%*%*




%*%*%*%*%*%*%*%*%*%*%*%*%*%*%*%*%*%*%*%*%*%*%*%*%*
\section{Phase Model and q-Bosons}
%*%*%*%*%*%*%*%*%*%*%*%*%*%*%*%*%*%*%*%*%*%*%*%*%*

This section introduces the \emph{phase} and \emph{\(q\)-boson}
models~\cite{Bogoliubov:1992, Bogoliubov:1997soj, Bogoliubov2005,
  Tsilevich:2006}. It serves as a comprehensive review of existing
literature, offering insights into established concepts. While the
content may not be entirely original, its presentation adds value,
particularly from a pedagogical standpoint. Additionally, it provides
a well-curated collection of references for further exploration.

In structuring this discussion, we adopt certain conventions
from~\cite{Wheeler:2010vmq}, while aligning our presentation style
more closely with that of~\cite{Tsilevich:2006}.



%%%%%%%%%%%%%%%%%%%%%%%%%%%%%%%%%%%%%%%%%%%%%%%%%%
%%%%%%%%%%%%%%%%%%%%%%%%%%%%%%%%%%%%%%%%%%%%%%%%%%

\subsection{Phase model}
Consider the \((M+1)\) set of operators \(\{\phi_i,
\phi^\dagger,\mathcal{N}_i\}_{i=0}^M\) such that
\begin{equation}
 [\mathcal{N}_i, \phi_j] = - \phi_i \delta_{i,j} \quad
 [\mathcal{N}_i, \phi_j^\dagger] =  \phi_i^\dagger \delta_{i,j}  \quad 
 [\phi_i, \phi_j^\dagger] =  \pi_i \delta_{i,j}  
\end{equation}
where \(\pi_i =(\ket{0} \bra{0})_i\) is the vacuum projection operator.
These operators can be written as
\begin{equation}
\phi = \sum_{n\geq 0}\ket{n}\bra{n+1}\quad 
\phi^\dagger = \sum_{n\geq 0}\ket{n+1}\bra{n} \quad 
N = \sum_{n\geq 0}n\ket{n}\bra{n}\; ,
\end{equation}
and it is easy to see that \(\phi^\dagger \phi = \bm{1} - \ket{0}
\ket{0}\) and \(\phi\phi^\dagger = \bm{1}\).

The Hamiltonian is given by
\begin{equation}
  H = - \frac{1}{2} \sum_{n =0}^M \left(\phi_n^\dagger \phi_{n+1}
  + \phi_n \phi_{n+1}^\dagger \right) + \bar{\mathcal{N}}\; .
\end{equation}
where \(\bar{\mathcal{N}} = \sum_{i=0}^M \mathcal{N}_i\) is the total
number operator, and we also impose periodic boundary conditions
\(\phi_{M+1} = \phi_0\) and \(\phi_{M+1}^\dagger = \phi_0^\dagger\).

These operators appear in the context of quantum optics, and for this
reason, it is referred to as the \emph{phase model}. It corresponds to
the strongly correlated limit of the \(q\)-bosons
model~\cite{Bogoliubov:1997soj}, which we will define shortly.

%%%%%%%%%%%%%%%%%%%%%%%%%%%%%%%%%%%%%%%%%%%%%%%%%%
%%%%%%%%%%%%%%%%%%%%%%%%%%%%%%%%%%%%%%%%%%%%%%%%%%
%%%%%%%%%%%%%%%%%%%%%%%%%%%%%%%%%%%%%%%%%%%%%%%%%%

\subsubsection{Representation}
The representation of the phase model algebra, is constructed using
the vacuum state, defined by \(\ket{0}_i\) by \(\phi_i\ket{0}_i =
0\). In this context, the state with \(n_i\) bosons (oscillators) is
given by \(\ket{n_i}_i = \phi_i^\dagger \ket{0}_i\).

Given the vacuum \(\ket{\bm{0}} = \ket{0}_0\otimes \ket{0}_1
\otimes \cdots \otimes  \ket{0}_M\), the Fock space is defined as 
\begin{equation}
  \mathcal{F} = \bigotimes_{i=0}^M \mathcal{F}_i = 
  \left\{\ket{\vec{n}} = \ket{n_0}\otimes \ket{n_1} \otimes \cdots
  \otimes \ket{n_M} \ | \ n_i \in \mathbb{N} \right\}\; ,
\end{equation}
where the states \(\ket{\vec{n}}\) are defined as 
\begin{equation}
  \ket{\vec{n}} = \ket{n_0}\otimes \ket{n_1} \otimes \cdots \otimes \ket{n_M} 
 \equiv  (\phi_0^\dagger)^{n_0} (\phi_1^\dagger)^{n_1} \cdots  (\phi_M^\dagger)^{n_M} \ket{\bm{0}} \; ,
\end{equation}
with \(\phi_i^\dagger \equiv \bm{1} \otimes \cdots \otimes
\phi_i^\dagger \otimes \cdots \otimes \bm{1}\).  Moreover, it is easy
to see that these states are normalized, that is
\(\bracket{\vec{n}}{\vec{m}}=\delta_{\vec{n}, \vec{m}}\).  Finally,
the actions of the operators \(\mathcal{N}_i\) and \(\pi_i\) are
\begin{equation}
    \pi_i\ket{\vec{n}}  = \delta_{n_i, 0} \ket{\vec{n}} \qquad 
    \mathcal{N}_i\ket{\vec{n}} = n_i \ket{\vec{n}}\; .
\end{equation}
 
Given a state \(\ket{\vec{n}} = \ket{n_0, n_1, \dots, n_M}\), we
associate a partition \( \lambda = (0^{n_0} 1^{n_1} 2^{n_2} \cdots
M^{n_M})\).  It's worth noting that this correspondence is not unique,
as the partition \(\lambda\) does not account for the number of
particles \(n_0\).  If the total number of particles \(N\) is known,
we can determine \(n_0 = N - \ell(\lambda)\), where \(\ell(\lambda)\)
represents the number of rows in the Young diagram defined by
partition \(\lambda\). This determination is crucial, as in our
subsequent considerations, the \(N\) particle sector remains fixed due
to the integrability of the model. Consequently, the value of \(n_0\)
becomes known once we specify the partition \(\lambda\).

Finally, based on the correspondence we mentioned above,
Wheeler~\cite{Wheeler:2010vmq} defines a map \(\mathcal{M}_\psi:
\mathcal{F} \to \mathcal{F}^{(0)}_\psi\), where
\(\mathcal{F}^{(0)}_\psi\) denotes the Fock space of charged free
fermions constructed from the neutral (fermionic) vacuum.

%%%%%%%%%%%%%%%%%%%%%%%%%%%%%%%%%%%%%%%%%%%%%%%%%%
%%%%%%%%%%%%%%%%%%%%%%%%%%%%%%%%%%%%%%%%%%%%%%%%%%
%%%%%%%%%%%%%%%%%%%%%%%%%%%%%%%%%%%%%%%%%%%%%%%%%%

\subsubsection{Bethe Ansatz}
The \(L\)-matrix is given by 
\begin{equation}
  L_{an} = 
\begin{pmatrix}
x^{ - 1/2} & \phi_n^\dagger \\ \phi_n & x^{1/2}
\end{pmatrix}_a\; , 
\end{equation}
where \(x \in \mathbb{S}^1\). We also have the monodromy matrix
\begin{equation}
  T_a(x) = L_{aM}(x) \cdots L_{a0}(x) = 
\begin{pmatrix}
A(x) & B(x) \\ C(x) & D(x)
\end{pmatrix}_a\; .
\end{equation}

With these expressions, one can finally build the Bethe states
\begin{equation}
  \ket{\Psi(y_1, \dots, y_N)} = \prod_{j=1}^N \mathbb{B}(y_j) \ket{\bm{0}} \qquad 
  \bra{\Psi(y_1, \dots, y_N)} = \bra{\bm{0}}\prod_{j=1}^N \mathbb{C}(y_j) 
\end{equation}
where \(\mathbb{B}(y) = y^{M/2} B(y)\) and \(\mathbb{C}(y) = y^{M/2} C(1/y)\).

When the coordinates \(\{ y_j \ | \ j =1, \dots , N\}\) satisfy the Bethe equations
\begin{equation}
\label{eq:bethe_eq}
  y^{N + M}_i = (-1)^{N-1} \prod_{\substack{j = 1 \\ j \neq i}}^N y_j\; , \qquad i = 1, \dots, N\; , 
\end{equation}
we say that the Bethe states are \emph{on-shell}; otherwise, we have
\emph{off-shell} states. In what follows, we will only consider
\emph{off-shell} states.

%%%%%%%%%%%%%%%%%%%%%%%%%%%%%%%%%%%%%%%%%%%%%%%%%%
%%%%%%%%%%%%%%%%%%%%%%%%%%%%%%%%%%%%%%%%%%%%%%%%%%

\subsection{q-Bosons}
The set of operators \(\{b_i, b^\dagger,\mathcal{N}_i\}_{i=0}^M\) constitute
\(M+1\) independent q-boson algebras defined by
\begin{equation}
[\mathcal{N}_i, b_j^\dagger]=\delta_{i,j} b_i^\dagger\; , \quad 
[\mathcal{N}_i, b_j]=-\delta_{i,j}b_i\; , \quad
[b_i, b_j^\dagger]= \delta_{i,j} q^{-2\mathcal{N}_i}  \equiv \delta_{i,j} Q^{\mathcal{N}_i}\; , 
\end{equation}
where we denote the deformation parameter as \(Q = q^{-2}\).

The q-boson model is characterized by its Hamiltonian
\begin{equation}
  \mathcal{H} = -\frac{1}{2} \sum_{n=0}^M
  \left(b_n^\dagger b_{n+1} + b_n b_{n+1}^\dagger \right) + \bar{\mathcal{N}}\; ,
\end{equation}
where \(\bar{\mathcal{N}} = \sum_{i=0}^M \mathcal{N}_i\), and we also
impose periodic boundary conditions \(b_{M+1} = b_0\) and
\(b_{M+1}^\dagger = b_0^\dagger\).

In the limit \(Q \to 1\), the q-bosons behave as ordinary bosons,
while the limit \(Q \to 0\) (\(q \to \infty\)) corresponds to the
phase model discussed earlier.

%%%%%%%%%%%%%%%%%%%%%%%%%%%%%%%%%%%%%%%%%%%%%%%%%%
%%%%%%%%%%%%%%%%%%%%%%%%%%%%%%%%%%%%%%%%%%%%%%%%%%
%%%%%%%%%%%%%%%%%%%%%%%%%%%%%%%%%%%%%%%%%%%%%%%%%%

\subsubsection{Representation}
We construct the representation of the q-boson algebra with
modifications of the phase model discussed earlier. Let us define the
\(i\)-th vacuum \(\ket{0}_i\) such that \(b_i\ket{0}_i = 0\). The
representation of this Hilbert space, denoted by \(\mathcal{F}_i^Q\),
is given by \(\ket{n_i}_i \propto b_i^\dagger \ket{0}_i\).

Given the vacuum \(\ket{\bm{0}} = \ket{0}_0\otimes \ket{0}_1
\otimes \cdots \otimes  \ket{0}_M\),
the Fock space is defined as 
\begin{equation}
  \mathcal{F}_{Q} = \bigotimes_{i=0}^M \mathcal{F}_i^Q = 
  \left\{\ket{\vec{n}} = \ket{n_0}\otimes \ket{n_1} \otimes \cdots
  \otimes \ket{n_M} \ | \ n_i \in \mathbb{N} \right\}\; ,
\end{equation}
where the actions of the operators \(\{b_i, b_i^\dagger\}\) are given
by the following relations
\begin{subequations}
\begin{alignat}{3}
    & b_0\ket{n_0} =(1 - \delta_{0, n_0}) \frac{(1 - Q^{n_0})}{(1 - Q)^{1/2}}\ket{n_0 - 1}
    &\hspace{0.5cm}& b_0^\dagger \ket{n_0} =  \frac{1}{(1 - Q)^{1/2}}\ket{n_0 + 1}  \\
    & b_i\ket{n_i} = \frac{(1 - \delta_{0, n_i})}{(1 - Q)^{1/2}}\ket{n_i - 1}
    && b_i^\dagger \ket{n_0} =  \frac{(1 - Q^{n_i+1})}{(1 - Q)^{1/2}}\ket{n_0 + 1}\quad i\neq 0\; .
\end{alignat}
\end{subequations}

Consequently, the states \(\ket{\vec{n}} \in \mathcal{F}_Q\) are
\begin{equation}
  \ket{\vec{n}} = \ket{n_0}\otimes \ket{n_1} \otimes \cdots \otimes \ket{n_M} 
 : =  (b_0^\dagger)^{n_0} (b_1^\dagger)^{n_1} \cdots  (b_M^\dagger)^{n_M} \ket{\bm{0}} \; ,
\end{equation}
where we write \(b_i^\dagger \equiv \bm{1} \otimes  \cdots \otimes
b_i^\dagger \otimes \cdots \otimes \bm{1}\).
Moreover, this space has an inner product that satisfies
\begin{equation}
  \bracket{\vec{n}}{\vec{m}} = \frac{[m_0]!}{\prod_{i=1}^M[m_i]!} \delta_{\vec{n}\vec{m}}\qquad 
    [n]! =
    \left\{
    \begin{array}{ll}
    \prod_{i=1}^n(1 - Q^i) & \textrm{if}\ \ n\neq 0 \\
    1 & \textrm{otherwise}
  \end{array}\right.\; .
\end{equation}

Similar to the phase model, we can associate to a given states
\(\ket{\vec{n}} = \ket{n_0, n_1, \dots, n_M}\), a Young diagram
\(\lambda = (1^{n_1} 2^{n_2} \cdots M^{n_M})\). It is useful to define
a proportionaly factor relating these two objects
\begin{equation}
  \ket{\lambda}_Q  = b_\lambda(Q) \ket{\vec{n}}  \ \; , \qquad b_\lambda(Q) = \prod_i [p_i(\lambda)]! 
\end{equation}
where \(p_i(\lambda)\) denotes the number of parts of size \(i\) in
the partition. These partition states satisfy
\(\bracket{\lambda}{\mu}_Q = b_\lambda(Q) \delta_{\lambda,\mu}\). Once
again, we see that the correspondence is not unique, as the number of
oscillators at site \(i=0\) is completely ignored in the partition
notation. Finally, if the number of particles is fixed, then \(n_0 = N
- \ell(\lambda)\).

%%%%%%%%%%%%%%%%%%%%%%%%%%%%%%%%%%%%%%%%%%%%%%%%%%
%%%%%%%%%%%%%%%%%%%%%%%%%%%%%%%%%%%%%%%%%%%%%%%%%%
%%%%%%%%%%%%%%%%%%%%%%%%%%%%%%%%%%%%%%%%%%%%%%%%%%

\subsubsection{Bethe Ansatz}
The \(L\)-operator for the q-boson is given by
\begin{equation}
  L_{an}(x, Q) =
  \begin{pmatrix}
    x^{-1/2} & (1 - Q)^{\frac{1}{2}} b_n^\dagger \\ (1 - Q)^{\frac{1}{2}} b_n & x^{1/2}
  \end{pmatrix}_a\; ,
\end{equation}
and the monodromy matrix is 
\begin{equation}
  T_a(x,Q) = L_{am}(x, Q)  \dots  L_{a0}(x, Q) = 
  \begin{pmatrix}
    A(x, Q) & B(x, Q) \\ C(x, Q) & D(x, Q)
  \end{pmatrix}_a\; .
\end{equation}

As before, the eigenstates of the Hamiltonian have the form
\begin{equation}
  \ket{\Psi(y_1, \dots, y_N; Q)} = \prod_{j=1}^N \mathbb{B}(y_j, Q) \ket{\bm{0}}\qquad 
  \bra{\Psi(y_1, \dots, y_N; Q)} = \bra{\bm{0}}\prod_{j=1}^N \mathbb{C}(y_j, Q) \; .
\end{equation}
where \(\mathbb{B}(y, Q) = y^{M/2} B(y, Q)\) and \(\mathbb{C}(y, Q) =
y^{M/2} C(1/y, Q)\). When the parameters \(\{ y_j \ | \ j =1, \dots , N\}\)
satisfy the Bethe equations given by
\begin{equation}
  y^{N + M}_i =\prod_{\substack{j = 1 \\ j \neq i}}^N\frac{Q y_i - y_j}{y_i - Q y_j}\; , \qquad i = 1, \dots, N\; , 
\end{equation}
we have \emph{on-shell} Bethe states; otherwise, they are \emph{off-shell} states.



%*%*%*%*%*%*%*%*%*%*%*%*%*%*%*%*%*%*%*%*%*%*%*%*%*
\section{Tau functions in the phase model}
%*%*%*%*%*%*%*%*%*%*%*%*%*%*%*%*%*%*%*%*%*%*%*%*%*

This section explores the presence of integrable hierarchies in the
phase model. We demonstrate its correspondence to a Toda hierarchy tau
function and discuss some implications, particularly its connection to
a matrix model. Additionally, we highlight the existence of
correlation functions in this model that satisfy the KP hierarchy
equations. Finally, we argue that these results imply that these
scalar products are also KP hierarchy tau functions, aligning with the
findings of Wheeler~\cite{Wheeler:2010vmq}.

Bogoliubov~\cite{Bogoliubov2005} has demonstrated that the scalar
product of two vectors in the \(N\)-particle sector of a chain with
length \(M+1\) is
\begin{equation}
\begin{split}
\label{eq:scalar}
  \mathcal{I}(N, M|\bm{x}, \bm{y}) & =
  \bra{\bm{0}} \prod_{i=1}^N \mathbb{C}(x_i) \prod_{j=1}^N \mathbb{B}(y_j) \ket{\bm{0}} \\ 
 & = \frac{ \det H(\bm{x},\bm{y})}{ \prod_{i<j}(x_i - x_j)(y_i - y_j)} \; ,
\end{split}
\end{equation}
where \(H\equiv [H_{ij}]_{i,j=1}^N\) is an \(N\times N\) matrix with components
\begin{equation}
\label{eq:h-matrix}
  H_{ij} = H(x_i, y_j) 
  =\frac{1 - (x_i y_j)^{ M + N}}{1 - x_i y_j }\; .
\end{equation}

%%%%%%%%%%%%%%%%%%%%%%%%%%%%%%%%%%%%%%%%%%%%%%%%%%
%%%%%%%%%%%%%%%%%%%%%%%%%%%%%%%%%%%%%%%%%%%%%%%%%%

\subsection{Toda tau functions}
We now argue that the scalar product defined above is a tau function
of the Toda hierarchy. Let us first write the function \(H(z,w)\) as
the geometric sum
\begin{equation}
\label{eq:h-exp}
  H(z,w) = \frac{1 - (zw)^{M+N}}{1 - zw} = \sum_{k=1}^{M+N} (zw)^{k-1} \; .
\end{equation}
Hence, we express the determinant of the \(H\) matrix as
\begin{equation}
  \det_{i,j} \left(H(x_i, y_j)\right) = \det_{i,j} \left( \sum_{k=1}^{M+N} x_i^{k-1} y_j^{k-1}\right)
\end{equation}
Furthermore, we can interpret this expression as the result of
multiplying an \(N\times (N+M)\) matrix \(\mathcal{X}\) by another
\((M + N)\times N\) matrix \(\mathcal{Y}\), which are given by
\begin{equation}
  \mathcal{X} = 
  \begin{pmatrix}
  x_1^0 & x_1^1 & \dots & x_1^{M+N-1} \\  
  x_2^0 & x_2^1 & \dots & x_2^{M+N-1} \\  
  \vdots \\
  x_N^0 & x_N^1 & \dots & x_N^{M+N-1} 
  \end{pmatrix}\quad \textrm{and} \quad 
  \mathcal{Y} = 
  \begin{pmatrix}
  y_1^0 & y_2^0 & \dots & y_N^0 \\  
  y_1^1 & y_2^1 & \dots & y_N^1 \\  
  \vdots & \vdots & & \vdots \\
  y_1^{N+M-1} & y_2^{M+N-1} & \dots & y_N^{M+N-1}
  \end{pmatrix}\; ,
\end{equation}
therefore 
\begin{equation}
  \det_{i,j} \left(H(x_i, y_j)\right)
  \equiv  \det_{i,j} \left( \mathcal{X}\mathcal{Y}\right)
  = \sum_{0 \leq \ell_{N} \leq \dots \leq \ell_1\leq N+M } \det_{ik}(x_i^{\ell_k}) \det_{ik}(y_j^{\ell_k})\; .
\end{equation}
If we now define \(\ell_j = \lambda_k - k + N\), and
using~(\ref{eq:scalar}), we have
\begin{equation}
\label{eq:scalar_exp}
\mathcal{I}(N, M|\bm{x}, \bm{y}) = \sum_{\lambda\subseteq [N,M]}
\frac{\det_{ik}(x_i^{\lambda_k - k + N})}{\Delta(\bm{x})} \frac{ \det_{ik}(y_j^{\lambda_k - k + N})}{\Delta(\bm{y})}
= \sum_{\lambda\subseteq [N,M]} s_\lambda(\bm{x}) s_\lambda(\bm{y}) \; .
\end{equation}
where \(\Delta(\bm{x})\) and \(\Delta(\bm{x})\) are Vandermonde
determinants. This formula agrees with the Schur expansion defined
in~\cite{Bogoliubov2005}.

Define two sets of Miwa coordinates \(\bm{t} = (t_1, t_2, \dots)\) and
\(\bm{t}' = (t'_{-1}, t'_{-2}, \dots)\) by
\begin{equation}
  t_p = \frac{1}{p}\sum_{j=1}^N x_j^p\qquad 
  t'_{-p} = \frac{1}{p}\sum_{j=1}^N y_j^p\; .
\end{equation}
One can write the inner product in terms of these coordinates, that is
\begin{equation}
  \mathcal{I}(N, M|\bm{t}, \bm{t}') = \sum_{\lambda\subseteq [N,M]} s_\lambda(\bm{t}) s_\lambda(\bm{t}') \; .
\end{equation}
This expression is known to be a tau function for \(M, N\to \infty\).
As such, utilizing the free fermions representation~\cite{Alexandrov:2012tr}, it can be written as
\begin{equation}
  \lim_{M,N\to \infty}\mathcal{I}(N, M|\bm{t}, \bm{t}')
 = \bra{\bm{0}} e^{\bm{J}_+(\bm{t})} e^{-\bm{J}_-(\bm{t}')} \ket{\bm{0}}\; .
\end{equation}
It is a tau function of the Toda hierarchy with trivial element \(\mathbb{1} \in GL(\infty)\).
It is nothing but the Cauchy's identity
\begin{equation}
  \sum_{\lambda } s_{\lambda}(\bm{t}) s_{\lambda}(\bm{t}')
    = \exp \left( \sum_{m\geq1} m t_m t'_{-m} \right) \; .
\end{equation}

Bringing all these facts together, the truncation for finite \(M\) and
\(N\) also yields tau functions of the Toda hierarchy. More
specifically, according to~\cite{Alexandrov:2012tr, Kharchev:1991gd,
  Zabrodin:2010ii}, the truncation of the tau function corresponds to
the inclusion of a projection operator \(\mathrm{P}^+\) in the
expectation value of the tau function written in the fermionic
representation.

As a final remark, we can also write
\begin{equation}
  H(z,w) = \sum_{k=0}^{M+N} h_k(zw)^{k-1} \; , 
\end{equation}
with \(h_k=1\) if \(k\in [0, M+N-1]\) and \(0\) otherwise. In this
case, one can define a diagonal \((N+M)\times (N+M)\) matrix
\(\mathcal{H} = \textrm{diag}(h_0, \dots, h_{M+N-1})\). Consequently,
if we repeat the arguments above, we find
\begin{equation}
\mathcal{I}(N, M|\bm{t}, \bm{t}') = \sum_{\lambda} h_\lambda s_\lambda(\bm{t}) s_\lambda(\bm{t}') \; .
\end{equation}
Here, \(h_\lambda\) is equal to \(1\) if \(\lambda \in [N,M]\), and it
is zero otherwise.

In this case, we find that this tau function is a trivial example of
the tau functions considered in~\cite{orlov:2001}. We anticipate that
in the analysis of more general correlation functions, the diagonal
terms \(h_\lambda\) will be more interesting. We will revisit this
discussion soon.

%%%%%%%%%%%%%%%%%%%%%%%%%%%%%%%%%%%%%%%%%%%%%%%%%%
%%%%%%%%%%%%%%%%%%%%%%%%%%%%%%%%%%%%%%%%%%%%%%%%%%
%%%%%%%%%%%%%%%%%%%%%%%%%%%%%%%%%%%%%%%%%%%%%%%%%%

\subsubsection{Matrix Models}
It is also interesting to note the particular case when \(M\to
\infty\), but with a finite number of particles \(N\). In this case,
the partitions \(\lambda \in [N, \infty]\) satisfy the condition
\(\ell(\lambda) \leq N\). Therefore, the scalar
product~(\ref{eq:scalar_exp}) becomes
\begin{equation}
  \mathcal{I}_N(\bm{t}, \bm{t}')\equiv 
  \lim_{M\to \infty}\mathcal{I}(N, M|\bm{t}, \bm{t}')
  = \sum_{\substack{\lambda \\ \ell(\lambda) \leq N}} s_\lambda(\bm{t})  s_\lambda(\bm{t}') \; .
\end{equation}
From~\cite{Zabrodin:2010ii}, we know that this expression can be
written as the following integral
\begin{equation}
  \mathcal{I}_N(\bm{t}, \bm{t}') =
  \frac{1}{N!} \prod_{\ell=1}^N \oint_{\Gamma_\ell} \frac{dz_\ell}{2 \pi i z_\ell}
  e^{\xi(\bm{t}, z_\ell) - \xi(\bm{t}', z_{\ell}^{-1})} \Delta(z)\Delta(z^{-1})\; ,
\end{equation}
where
\begin{equation}
\xi(\bm{t}, z) = \sum_{k\geq 0} t_k z^k \qquad
\xi(\bm{t}', 1/z) = \sum_{k\geq 0} t'_{-k} z^{-k} \; . 
\end{equation}
See also~\cite{Kharchev:1991gd} for a detailed proof of this relation,
and~\cite{Orlov:2005} for other details. 

From the results of~\cite{Zabrodin:2010ii}, see citations therein, we
have an interesting consequence of this representation. Impose the
Bethe equations~(\ref{eq:bethe_eq}) to one set of variables, say \(x_j
= e^{-ip_j} \in \mathbb{S}^1\). Additionally, let us set \(\bm{t}' = -
\bm{t}^\star\). In this particular case, we have
\begin{equation}
\xi(\bm{t}, z) - \xi(\bm{t}', 1/z)  = 2 \ \textrm{Re}\left(\sum_k t_k z^k\right)\; .
\end{equation}
Then, the phase model is equivalent to an ensemble of \(N\) 2D Coulomb
particles on a circle. In this case, we find that the quantities
\(\lambda_{\ell}\) are eigenvalues of a matrix \(U\).

Furthermore, according to Zabrodin~\cite{Zabrodin:2010ii}, see
citations therein, we also know that under the rescaling \(t_k \to
T_k/ \hbar\), \(t'_k \to T_{-k}/ \hbar\) and \(N = T_0/ \hbar\), we
obtain the dispersion limit tau function
\begin{equation}
  F_0(\bm{T}) = \log S_{T_0}(\bm{T}, \bm{T}') + \mathcal{O}(\hbar)\; ,
\end{equation}
that is a free energy, from the viewpoit of the matrix integral
partition function.

It remains unclear how one can use this fact to determine properties
of the integrable model, but it might be possible to study the
analytic structure of the free energy \(F_0\) to gain some
understanding of the Bethe roots \(\bm{x}\). This problem is currently
under further investigation, and we hope to report new results
elsewhere.

%%%%%%%%%%%%%%%%%%%%%%%%%%%%%%%%%%%%%%%%%%%%%%%%%%
%%%%%%%%%%%%%%%%%%%%%%%%%%%%%%%%%%%%%%%%%%%%%%%%%%
%%%%%%%%%%%%%%%%%%%%%%%%%%%%%%%%%%%%%%%%%%%%%%%%%%

\subsubsection{KP tau function}
Lastly, one may also observe that if we fix one set of coordinates,
say \(\bm{y}\), then we can write
\begin{equation}
 \mathcal{I}(N,M|\bm{t}) 
 = \sum_{\lambda} s_\lambda(\bm{y})  s_\lambda(\bm{t}) 
 \equiv \sum_{\lambda} c_\lambda(\bm{y})  s_\lambda(\bm{t}) \; ,
\end{equation}
with coefficients \(c_\lambda(\bm{y}) = \det (h_{\lambda_i-i
  +j}(\bm{y}))\). But now, it is trivial to notice that these are
Plücker coordinates in the Jacobi-Trudi form, as seen
in~\cite{Miwa2000, Alexandrov:2012tr}.

Hence, the following expression
\begin{equation}
 \mathcal{I}(N,M|\bm{t}) = \sum_{\lambda \subseteq [N,M]} s_\lambda(\bm{y})  s_\lambda(\bm{t}) \; ,
\end{equation}
is also a KP tau function, a fact that we already know
from~\cite{Wheeler:2010vmq}, where the author proved this statement
using the free fermions formalism.

%%%%%%%%%%%%%%%%%%%%%%%%%%%%%%%%%%%%%%%%%%%%%%%%%%
%%%%%%%%%%%%%%%%%%%%%%%%%%%%%%%%%%%%%%%%%%%%%%%%%%

\subsection{Correlation functions}
Bogoliubov has also shown in~\cite{Bogoliubov2005}, that the
correlation functions
\begin{subequations}
\begin{equation}
\begin{split}
\label{eq:correlation}
  A_m(N, M|\bm{x}, \bm{y}\setminus \{y_N\})
  & = \bra{0} \prod_{j=1}^N \mathbb{C}(x_j)
  \prod_{k=1}^{N-1} \mathbb{B}(y_k) \phi_m^\dagger \ket{0}\\
  & =  \prod_{j=1}^N x_{j}^{M/2} \prod_{k=1}^{N-1} y_{j}^{M/2}
  \bra{0} \prod_{j=1}^N C(1/x_j) \prod_{k=1}^{N-1} B(y_k) \phi_m^\dagger \ket{0}
\end{split}
\end{equation}
can be written as
\begin{equation}
  A_m(N, M|\bm{x}, \bm{y}\setminus \{y_N\}) = 
  \frac{(-1)^{N-1}}{y_N^{(N-1)/2}} \prod_{j=1}^N x_{j}^{M/2}
  \prod_{k=1}^{N-1} y_{j}^{M/2}
  \left( \prod_{t<N} \frac{y_N - y_t}{y_t} \right)
  \frac{\det Q}{\det H} \mathcal{I}(N,M|\bm{x}, \bm{y})
\end{equation}
\end{subequations}
where \(Q\) is an \(N\times N\) matrix with components 
\begin{equation}
 Q_{jN} = x_j^{(M + N - 1- 2m)/2} \quad  \textrm{and} \quad 
 Q_{jk} = H_{jk} \; , 
\end{equation}
and \(H_{jk} = H(x_j, y_k)\) are the components of the matrix \(H\)
in~(\ref{eq:h-matrix}).  The components \(Q_{jN}\) are independent of
the coordinates \(y\); therefore, we cannot express the above
expression as a Toda hierarchy tau function.

We already know from~(\ref{eq:scalar}) that 
\begin{equation}
    \frac{\mathcal{I}(N,M|\bm{x}, \bm{y})}{\det H}=  \frac{1}{\Delta(x)\Delta(y)} \; ,
\end{equation}
then
\begin{equation}
  A_m(N, M|\bm{x}, \bm{y}\setminus \{y_N\})  \propto
  \frac{\det Q}{\Delta(x)} \equiv \mathcal{A}_m(N,M|\bm{x}, \bm{y})\; ,
\end{equation}
and we treat the coordinates \(\{ \bm{y} \}\) as a set of \(N\) fixed
parameters.

Furthermore, we define vector field \(\bm{F}(z) = (F_1, \dots, F_N)\),
where the components are given by
\begin{equation}
    F_j (z) = H(z, y_j) \quad \textrm{if} \ j \neq N\; , \qquad \textrm{and}\qquad 
    F_N (z)  = z^{(M + N - 1 - 2m)/2} \; .
\end{equation}
As before, we expand \(F_j(z)\) as the geometric sum
\begin{equation}
  F_j(z) = \frac{1 - (y_j z)^{M + N}}{1 - y_j z} = \sum_{n=0}^{N + M - 1} (y_j z)^n
  \equiv \sum_{n=0}^{M + N -1} f_{j, n} z^n\; .
\end{equation}

With these definitions, we conclude that
\begin{equation}
\mathcal{A}_m(N,M|\bm{x}, \bm{y}) =\frac{\det_{jk} F_j(x_k)}{\Delta(x)}\; .
\end{equation}
From the expansion 
\begin{equation}
  \det_{jk} F_j(x_k) = \det_{jk} \left(  \sum_{n=0}^{M + N -1} f_{j, n} x_k^n \right) \; .
\end{equation}
By utilizing the Cauchy-Binet formula, we can write
\begin{equation}
    \mathcal{A}_m(N,M|\bm{x}, \bm{y})
  = \sum_{0\leq \ell_N\leq \dots \leq \ell_1 \leq N+M}
  \frac{\det_{jk}(f_{j, \ell_k}) \det_{jk}(x_k^{\ell_j})}{\Delta(x)}\; . 
\end{equation}

We now employ the definition of the Schur polynomials and utilize the
Jacobi-Trudi expression of Plücker coordinates, as detailed
in~\cite{Alexandrov:2012tr}. Ultimately, this leads to the following
expression:
\begin{equation}
\mathcal{A}_m(N,M|\bm{x}, \bm{y}) =
\sum_{\lambda} c_\lambda(\bm{y}) s_\lambda(\bm{x}) \; ,
\end{equation}
where \(c_\lambda \equiv \det_{jk}(y_j^{\ell_k})\),  \(\ell_k = \lambda_k - k +N\).
Therefore, we have that this expression is a KP tau function. 

%%%%%%%%%%%%%%%%%%%%%%%%%%%%%%%%%%%%%%%%%%%%%%%%%%
%%%%%%%%%%%%%%%%%%%%%%%%%%%%%%%%%%%%%%%%%%%%%%%%%%
%%%%%%%%%%%%%%%%%%%%%%%%%%%%%%%%%%%%%%%%%%%%%%%%%%

\subsubsection{Skew Schur polynomials expansion}
From the coordinate expansion of the Bethe
vectors~\cite{Bogoliubov2005, Tsilevich:2006}, it is easy to show that
\begin{equation}
\prod_{j=1}^N \mathbb{B}(x_j)\ket{\lambda}  = \sum_{\mu \supset \lambda } s_{\mu/\lambda}(\bm{x})\ket{\mu}\qquad 
 \bra{\lambda} \prod_{j=1}^N \mathbb{C}(x_j) = \sum_{\mu \supset \lambda } s_{\mu/\lambda}(\bm{x})\bra{\mu}
\end{equation}
where \(s_{\mu/\lambda}\) are skew Schur polynomials~\cite{Macdonald:1998}.
Therefore, we can write the correlation function~(\ref{eq:correlation}) as 
\begin{equation}
\begin{split}
  A_m(N, M|\bm{x}, \bm{y}\setminus \{y_N\})
  & = \sum_\lambda \sum_{\mu \supset (m)} s_{\lambda}(\bm{x}) s_{\mu/(m)}(\bm{y}\setminus\{y_N\})\\
  & = \sum_\mu s_{\mu/(m)}(\bm{y}\setminus\{y_N\}) s_{\mu}(\bm{x})\; .
\end{split}
\end{equation}
where we have used that \(\phi^\dagger \ket{\bm{0}} = \ket{(m)}\), and in the second line
we sum over all partitions, since \(s_{\mu/(m)} = 0\), \(\forall \) \(\mu \not \supset (m)\).
This expression also shows that the expression \(A_m\) can also be written as a tau function. 

More generally, let us consider the correlation functions 
\begin{equation}
\begin{split}
  A_{\lambda_1 \lambda_2}(N', N, M|\bm{x}, \bm{y}) & =
  \bra{\lambda_1} \prod_{j=1}^N \mathbb{C}(x_j)
  \prod_{k=1}^{N'} \mathbb{B}(y_k)\ket{\lambda_2} \\
  & = \sum_{\mu} s_{\mu/\lambda_1}(\bm{x}) s_{\mu/\lambda_2}(\bm{y}) \; .
\end{split}
\end{equation}
where \(N' + \ell(\lambda_2) + n_0 = N' + \ell(\lambda_2) + n'_0\) and
we have also used that the skew Schur polynomial for any Young diagram
\(\mu\) that does not contain \(\lambda_1\) and \(\lambda_2\)
vanishes. Using the elementary property of skew Schur
polynomials~\cite{Macdonald:1998}
\begin{equation}
  \sum_{\mu} s_{\mu/\lambda_1}(\bm{x}) s_{\mu/\lambda_2}(\bm{y}) = \prod_{i,j}\frac{1}{1 - x_i y_j}
 \sum_\nu s_{\lambda_1/\nu}(\bm{x}) s_{\lambda_2/\nu}(\bm{y})\; ,
\end{equation}
and the Cauchy's identity, we have 
\begin{equation}
\begin{split}
  A_{\lambda_1 \lambda_2}(N', N, M|\bm{x}, \bm{y}) & =
 \sum_\mu s_{\mu}(\bm{x}) s_{\mu}(\bm{y})
 \sum_\nu s_{\lambda_1/\nu}(\bm{x}) s_{\lambda_2/\nu}(\bm{y})\; .
\end{split}
\end{equation}
Therefore, this correlation function is nothing but the off-shell
norm~(\ref{eq:scalar}) times a finite sum over skew Schur
functions. This expression illustrates that the existence of these tau
functions is not a trivial aspect of this model.

%%%%%%%%%%%%%%%%%%%%%%%%%%%%%%%%%%%%%%%%%%%%%%%%%%
%%%%%%%%%%%%%%%%%%%%%%%%%%%%%%%%%%%%%%%%%%%%%%%%%%
%%%%%%%%%%%%%%%%%%%%%%%%%%%%%%%%%%%%%%%%%%%%%%%%%%

\subsubsection{Schur polynomials expansion of some correlation functions}
Indeed, it is worth noting that other quantities, which are not tau
functions, may have interesting Schur polynomial expansions. Consider
the state calculated in~\cite{Bogoliubov2005}.
\begin{equation}
  \ket{\mathcal{Y}} =
 \sum_{\vec{n}} \prod_{j=0}^M \ket{n_j} = \sum_\mu \ket{\mu} \qquad \sum_j n_j = N\; .
\end{equation}
Then
\begin{equation}
 \bra{\mathcal{Y}} \prod_{j=1}^N B(\bm{x}) \ket{\nu} = 
  \sum_{\lambda\subseteq [N,M]} s_{\lambda/\nu}(\bm{x})\; . 
\end{equation}
While it may not be a tau function, it has an interesting expansion as
a sum of skew Schur polynomials.

%%%%%%%%%%%%%%%%%%%%%%%%%%%%%%%%%%%%%%%%%%%%%%%%%%
%%%%%%%%%%%%%%%%%%%%%%%%%%%%%%%%%%%%%%%%%%%%%%%%%%

\subsection{General tau functions}
Based on the discussion we have had so far, and on the general mapping
between the Phase model and free fermions, one can grasp the general
form of tau functions in the context of this phase model.

Let us consider the vertex operator construction~\cite{Okounkov2001},
as also discussed in~\cite{Alexandrov:2012tr, Wheeler:2010vmq}. We
consider the vacuum state \(\ket{\bm{0}}\), often referred to as a
``Dirac sea'', defined by the conditions \(\psi_m\ket{\bm{0}} =
\psi_n^\star\ket{\bm{0}} =0 \) for \(m<0\) and \(n \geq 0\), where
\(\psi_n\) are components of a holomorphic free fermionic field. In
this formalism, the partition states are given by
\begin{equation}
  \ket{\mu} = \textrm{sign}(\sigma) \prod_{j=1}^d \psi_{a_j} \psi^\star_{-b_j}\ket{\bm{0}}\; ,
\end{equation}
where \(\textrm{sign}(\sigma) = \pm 1\) are defined in a such a way
that the Shur coefficients of the vertex operators (defined below)
have positive coefficients.

The pairs \(\{(a_j|b_j)\}_{j=1}^d\) define the Frobenius notation of
the partition \(\mu = (\mu_1, \mu_2, \dots, \mu_\ell)\). In this
notation, \(a_j\) is given by \(\mu_j - j\) and \(b_j\) is given by
\(\mu'_j - j\), where \(d\) represents the number of boxes in the
diagonal of the Young diagram, and \(\mu'\) is its conjugate, or
transpose, diagram. From these definitions, we have the equivalence
\begin{equation}
  \prod_{k\geq 1} (\phi_k^\dagger)^{n_k} \mapsto \textrm{sign}(\sigma) \prod_{j=1}^d
  \psi_{a_j} \psi^\star_{-b_j} \qquad |\mu| = \sum_k k n_k = \sum_j(a_j + b_j) + d\; .
\end{equation}

Finally, the \(\mathfrak{gl}(\infty)\) algebra has generators given by the
bilinears \(X = \sum_{j, j \in \mathbb{Z}} x_{ij} \bm{\colon} \psi_i \psi_j^\star\bm{\colon}
+ c\), where \(c\in \mathbb{C}\), \(x_{ij} =
0\) for large \(|j -i|\), say \(\geq M\), and the colons denote the normal ordering
\begin{equation}
  \bm{\colon} \psi_i \psi_j^\ast \bm{\colon} =  \psi_i \psi_j^\ast
  - \bra{0} \psi_i \psi_j^\ast \ket{0} \; .
\end{equation}
The group \(GL(\infty)\) is defined through the \(\exp:
\mathfrak{gl}(\infty) \to GL(\infty)\) as usual.

Note that these elements have finitely many non-zero entries,
including diagonal terms, upper triangular terms, and lower triangular
terms. These are equivalent to the number operators \(\mathcal{N}\),
annihilation operators \(\phi\), and creation operators
\(\phi^\dagger\), respectively. The central charge is equivalent to
the vacuum projection \(\pi = \ket{\bm{0}}\bra{\bm{0}}\). Therefore,
\begin{equation}
   GL(\infty) \ni G \mapsto
   \mathcal{G} = \exp \left[\sum_{i=1}^M \left( \sum_{a=1}^3
   c_{i, a} T_i^{a}  + c \pi_i \right)\right]\; ,
\end{equation}
where \(T^{a}_i \in \{ \mathcal{N}_i, \phi_i, \phi_i^\dagger \}_{i=1}^M\). 

The operators \(\mathbb{B}(x)\) and \(\mathbb{C}(x)\), 
for a large enough chain \(M\to \infty\),
are related to the vertex operators
\(\Gamma_-(x)\) and \(\Gamma_+(x)\), respectivelly, as
\begin{equation}
  \begin{split}
    \mathbb{B}(x) \mapsto \Gamma_-(x) & = \exp \left( \sum_{n\geq 1} \frac{1}{n}x^n J_{-n}\right) \\
    \mathbb{C}(x) \mapsto \Gamma_+(x) & = \exp \left( \sum_{n\geq 1} \frac{1}{n}x^n J_{n}\right) \; ,
  \end{split}
\end{equation}
where \(J_n \) is written in terms of free fermions as \(J_n =
\sum_{j\in \mathbb{Z}} \bm{\colon} \psi_j \psi_{j+n}^\ast
\bm{\colon}\). These set of operators generate a Heisenberg subalgebra
\(\widehat{\mathfrak{gl}}(1) \subset \mathfrak{gl}(\infty)\)
\begin{equation}
  [J_m, J_n] = m \delta_{n+m,0}\; .
\end{equation}

Putting all these facts together, we have that the tau functions of the Toda 
hierarchies, given by
\begin{equation}
  \tau_s(\bm{x}, \bm{y}) = \bra{s} \prod_{i} \Gamma_+(x_i) G \prod_j \Gamma_-(x_j) \ket{s}\; ,
\end{equation}
are mapped into objects of the form 
\begin{equation}
\label{eq:tau-corr}
\begin{split}
  \tau(\bm{x}, \bm{y}) & = \bra{\Psi(\bm{x})} \mathcal{G} \ket{\Psi(\bm{y})} \\
  & = \bra{\bm{0}} \prod_{i} \mathbb{C}(x_i)
  \mathcal{G} \prod_j \mathbb{B}(y_j) \ket{\bm{0}}\; ,
\end{split}
\end{equation}
where we necessarily have \(s=0\) in the phase model

%%%%%%%%%%%%%%%%%%%%%%%%%%%%%%%%%%%%%%%%%%%%%%%%%%
%%%%%%%%%%%%%%%%%%%%%%%%%%%%%%%%%%%%%%%%%%%%%%%%%%
%%%%%%%%%%%%%%%%%%%%%%%%%%%%%%%%%%%%%%%%%%%%%%%%%%

\subsubsection{Hypergeometric tau functions}
From these expressions, we can conclude that if we consider a diagonal
group element \(\mathcal{G} = \exp \left( \sum_{i\geq 0} c_i
\mathcal{N}_i\right)\), we have that~(\ref{eq:tau-corr}) becomes
\begin{equation}
\begin{split}
  \tau(\bm{x}, \bm{y}) 
  & = \bra{\bm{0}} \prod_{i} \mathbb{C}(x_i)
  e^{\sum_i c_i \mathcal{N}_i} \prod_j \mathbb{B}(y_j) \ket{\bm{0}}\\
  & = \sum_{\mu, \nu \subseteq [N,M]} c_{\mu\nu} s_\mu(\bm{x}) s_\nu(\bm{x})
\end{split}
\end{equation}
where
\begin{equation}
  c_{\mu \nu} = \bra{\mu} e^{\sum_i c_i \mathcal{N}_i}\ket{\nu} \; .
\end{equation}
From the representation of the phase model, we have that for the partition
\(\mu = (1^{n_1} 2^{n_2} \dots M^{n_M})\), we have
\begin{equation}
  \mathcal{N}_i \ket{\mu} = n_i \ket{\mu} \qquad n_0 = N - \ell(\mu)\; .
\end{equation}

Consequently, we have
\begin{equation}
  c_{\mu \nu} \equiv \delta_{\mu\nu} c_{\mu}  =  \delta_{\mu\nu} \prod_i e^{c_i n_i}\; , 
\end{equation}
and we conclude that 
\begin{equation}
  \tau(\bm{x}, \bm{y}) 
  = \sum_{\mu, \nu \subseteq [N,M]} c_{\mu} s_\mu(\bm{x}) s_\mu(\bm{x})\; ,
\end{equation}
We observe that this tau function has diagonal coordinates
\(c_\lambda\). These tau functions belong to the hypergeometric type
considered in~\cite{Orlov:2000, orlov:2001, Orlov:2001b, Orlov:2005}.








%*%*%*%*%*%*%*%*%*%*%*%*%*%*%*%*%*%*%*%*%*%*%*%*%*
\section{Tau functions in the Q-bosons model}
%*%*%*%*%*%*%*%*%*%*%*%*%*%*%*%*%*%*%*%*%*%*%*%*%*

We now shift our focus to the case of q-bosons. The analysis parallels
what we did before, but the specific details are markedly
different. We begin by examining the norm of two off-shell Bethe
states.

It has been shown~\cite{Tsilevich:2006} (see
also~\cite{Sulkowski:2008mx, Wheeler:2010vmq}) that the Bethe states
\(\ket{\Psi(\bm{x})}\) have coordinate expansions
\begin{equation}
  \ket{\Psi(\bm{x})} = \sum_{\mu \subseteq [N,M]} P_\mu(\bm{x}, Q) \ket{\mu}_Q  \qquad 
  \bra{\Psi(\bm{x})} = \sum_{\mu \subseteq [N,M]} P_\mu(\bm{x}, Q) \prescript{}{Q}{\bra{\mu}}\; ,
\end{equation}
where \(P_\lambda\) denote Hall-Littlewood polynomials. In this form,
the scalar product of two off-shell Bethe states in the q-boson model
can be easily calculated to be
\begin{equation}
\label{eq:scalar-hl}
\mathcal{I}_Q(N,M | \bm{x}, \bm{y}) = \bra{\bm{0}} \prod_{j=1}^N \mathbb{C}(x_j, Q)
\prod_{k=1}^N \mathbb{B}(y_k, Q) \ket{\bm{0}}
= \sum_{\lambda \subseteq [N,M]} b_\lambda(Q) P_{\lambda}(\bm{x}, Q) P_{\lambda}(\bm{y}, Q)
\end{equation}
where we use that \(\bracket{\lambda}{\mu} = b_\lambda(Q)
\delta_{\lambda, \mu}\) and the completeness relation
\begin{equation}
  \sum_{\lambda} \frac{1}{b_\lambda(Q)} \ket{\lambda}_Q {}_Q\bra{\lambda}  = \mathbb{1}\; .
\end{equation}

It turns out that this expansion is the Cauchy identity for
Hall-Littlewood polynomials~\cite{Macdonald:1998}
\begin{equation}
\label{eq:cauchy_hl}
\sum_{\lambda} b_\lambda(Q) P_{\lambda}(\bm{x}, Q) P_{\lambda}(\bm{y}, Q)
= \prod_{j, k=1}^\infty \frac{1-Q x_j y_k}{1 - x_j y_k}\; ,
\end{equation}

Our goal is to explore some properties of Hall-Littlewood polynomials
to gain insight into aspects of this expansion.

%%%%%%%%%%%%%%%%%%%%%%%%%%%%%%%%%%%%%%%%%%%%%%%%%%
%%%%%%%%%%%%%%%%%%%%%%%%%%%%%%%%%%%%%%%%%%%%%%%%%%

\subsection{Scalar product: determinant formula}
We aim to refine the expressions above. First, we consider a
determinant expression for the scalar product~(\ref{eq:scalar-hl}). We
proceed with the scenario where \(M\) and \(N\) are very large but
finite. In this case, we express this scalar product as
\begin{equation}
\label{eq:q-inner}
  \mathcal{I}_Q(N,M | \bm{x}, \bm{y})  
= \prod_{j_1, j_2} (1-Q x_{j_1} y_{j_2}) \prod_{k_2, k_2}(1 - x_{k_1} y_{k_2})^{-1}\; .
\end{equation}

Using the Cauchy's identity for Schur polynomials, that is 
\begin{equation}
  \sum_\lambda s_\lambda(\bm{x}) s_\lambda(\bm{y}) = \prod_{i,j} \frac{1}{1 - x_i y_j}\; ,
\end{equation}
and from the results derived in the phase model, we find that
\begin{equation}
  \prod_{i,j}\frac{1}{1 - x_i y_j}  \equiv \mathcal{I}(N,M|\bm{x}, \bm{y}) = 
  \frac{\det_{j,k}H(\bm{x},\bm{y})}{\Delta(\bm{x}) \Delta(\bm{y})}\; .
\end{equation}
where \(H\) is the matrix~(\ref{eq:h-matrix}). Consequently, the scalar
product~(\ref{eq:q-inner}) becomes
\begin{equation}
  \mathcal{I}_Q(N,M | \bm{x}, \bm{y})  
= \frac{\mathcal{I}(N,M|\bm{x}, \bm{y})}{\mathcal{I}(N,M|\bm{x}, Q\bm{y})}\; , 
\end{equation}
and we see that it is the quotient of scalar products of the phase
model. Hence, it is the quotient of two Toda tau functions.
Observe that in the case \(Q = 0\), we have \(\mathcal{I}(N, M|
\bm{x}, \bm{0}) = 1\), then \(\mathcal{I}_0(N,M | \bm{x}, \bm{y})
= \mathcal{I}(N,M | \bm{x}, \bm{y})\), as expected.

Additionally, we write
\begin{equation}
  \mathcal{I}_Q(N,M | \bm{x}, \bm{y})  
  = \frac{\Delta(Q\bm{y})}{\Delta(\bm{y})}
  \frac{\det H(\bm{x}, \bm{y})}{\det H(\bm{x},Q \bm{y})} \; .
\end{equation}
Using \(\Delta(Q\bm{y})/ \Delta(\bm{y}) = \prod_{j=1}^{N-1} Q^j\), we have 
\begin{equation}
  \mathcal{I}_Q(N,M | \bm{x}, \bm{y}) =  Q^{N(N-1)/2}\;
  \frac{\det H(\bm{x}, \bm{y})}{\det H(\bm{x}, Q \bm{y})} \; .
\end{equation}
Let us denote \(H(\bm{x}, Q\bm{y}) = H_{Q}\), therefore
\begin{equation}
  \mathcal{I}_Q(N,M | \bm{x}, \bm{y}) = Q^{N(N-1)/2} \det H_{Q}^{-1} \det H
= Q^{N(N-1)/2} \det \mathcal{H}(\bm{x}, \bm{y}) \; ,
\end{equation}
where \(\mathcal{H} = H_{Q}^{-1} H\) is a matrix with components 
\(\mathcal{H}_{i,j} \equiv \mathcal{H}(x_i, y_j)\) where
\begin{equation}
  \mathcal{H}(z, w) = \sum_{k} \frac{(1 - z y_k)}{(1 - Q z y_k)} 
\frac{(1 - Q x_k w)}{(1 - x_k w)}\; .
\end{equation}
Note that from this expression, we cannot decompose this function as
in~(\ref{eq:h-exp}) since the coefficients in this expansion depend on
\(\bm{x}\) and \(\bm{y}\).

%%%%%%%%%%%%%%%%%%%%%%%%%%%%%%%%%%%%%%%%%%%%%%%%%%
%%%%%%%%%%%%%%%%%%%%%%%%%%%%%%%%%%%%%%%%%%%%%%%%%%

\subsection{Scalar product: big Schur functions}

Let us now recover the result originally derived
in~\cite{Foda:2008hn} that shows how the scalar
product~(\ref{eq:scalar-hl}) is a tau function of the KP hierarchy.
From~\cite[see chap.3, sec. 4, eq. (4.7)]{Macdonald:1998}, we can
expand the Cauchy identity~(\ref{eq:cauchy_hl}) as
\begin{equation}
 \prod_{j, k=1}^\infty \frac{1-Q x_j y_k}{1 - x_j y_k} = 
\sum_{\lambda} \mathcal{S}_{\lambda}(\bm{x}, Q) s_{\lambda}(\bm{y}) =
\sum_{\lambda} \mathcal{S}_{\lambda}(\bm{y}, Q) s_{\lambda}(\bm{x}) \; ,
\end{equation}
where the polynomials \(\mathcal{S}_\lambda\), the big-Schur
functions, are given by a Jacobi-Trudi formula
\begin{equation}
  \mathcal{S}_{\lambda} (\bm{y}, Q) = \det(q_{\lambda_i -i + j}(\bm{y}, Q))\; , 
\end{equation}
and the coefficients \(q_m\) can be obtained from the following expression
\begin{equation}
 \sum_{m} q_m(\bm{y}, Q) z^m =
 \prod_{j} \frac{1-Q y_j z}{1 - y_j z}
\end{equation}
where \(z\) is a formal variable. It has been
argued~\cite{Foda:2008hn} in that if we interpret the big-Schur
functions as Plücker coordinates, the expression
\begin{equation}
  \mathcal{I}_Q(N,M | \bm{x}, \bm{y})
  = \sum_{\lambda \subseteq [N,M]} \mathcal{S}_{\lambda}(\bm{x}, Q) s_{\lambda}(\bm{y})
  = \sum_{\lambda \subseteq [N,M]} \mathcal{S}_{\lambda}(\bm{y}, Q) s_{\lambda}(\bm{x})\; ,
\end{equation}
is a restricted KP tau function with respect to both set of coordinates,
that is \(\bm{x}\) and \(\bm{y}\).

%%%%%%%%%%%%%%%%%%%%%%%%%%%%%%%%%%%%%%%%%%%%%%%%%%
%%%%%%%%%%%%%%%%%%%%%%%%%%%%%%%%%%%%%%%%%%%%%%%%%%

\subsection{Kostka-Foulkes expansion}
Let us expand these polynomials in a Schur polynomials basis, that is
\begin{equation}
P_\lambda(\bm{x}, Q) = \sum_{\mu} K^{-1}_{\lambda \mu}(Q) s_\lambda(\bm{x})\; , 
\end{equation}
where \(K^{-1}_{\mu\nu}(Q) \in \mathbb{Z}[Q]\) are inverse
Kostka-Foulkes polynomials~\cite{Macdonald:1998, Wheeler:2018}.

Expanding now the big-Schur functions as
\begin{equation}
\label{eq:kf-exp}
  \mathcal{S}_\lambda(\bm{x}, Q) = \sum_{\mu} c_{\lambda\mu}(Q) s_\mu(\bm{x})\; ,
\end{equation}
we can fix the coefficients \(c_{\lambda\mu}\) in terms of Kostka-Foulkes
polynomials using the orthogonality relations of these polynomials~\cite{Macdonald:1998}. 
In particular, there is an inner product in the ring of symmetric functions such that
\begin{equation}
  \frac{1}{b_\mu}\langle P_\mu(\bm{x}, Q), P_\mu(\bm{x}, Q)\rangle = \langle
  \mathcal{S}_\mu(\bm{x}, Q), s_\mu(\bm{x})\rangle = \delta_{\mu\nu}\; .
\end{equation}
Therefore,
\begin{equation}
  \begin{split}
\delta_{\mu\nu} & = \sum_\lambda c_{\mu\lambda}  \langle s_\lambda(\bm{x}), s_\mu(\bm{x})\rangle\\
& = \sum_\lambda \sum_{\pi_1, \pi_2} c_{\mu\lambda}  K_{\lambda \pi_1} K_{\mu \pi_2}
\langle P_{\pi_1}(\bm{x}, Q), P_{\pi_2}(\bm{x}, Q)\rangle \\ 
& = \sum_\lambda \sum_{\pi} c_{\mu\lambda} b_{\mu}^{-1}  K_{\lambda \pi} K_{\mu \pi}\; ,
  \end{split}
\end{equation}
and we conclude that 
\begin{equation}
\label{eq:indices}
c_{\mu \nu} = \sum_\pi K_{\mu\pi}^{-1} b_\mu (K_{\nu\pi}^{-1})^T\; .
\end{equation}
\marginpar[left]{\tiny Fix these \\ indices later.}

Hence, we conclude that 
\begin{equation}
\sum_{\mu} S_{\mu}(\bm{x},Q) s_{\nu}(\bm{y}) = 
\sum_{\mu , \nu} c_{\mu\nu} s_{\mu}(\bm{x}) s_{\nu}(\bm{y}) 
\end{equation}
and everything boils down to the analysis of the
indices~(\ref{eq:indices}). In has been shown
in~\cite{Necoechea:2019wbg} that the big Schur functions themselves
are KP-tau functions\footnote{ We give a simple argument for this fact
in the remark 2 below.} therefore, the coefficients \((c_\lambda)_\mu
\equiv c_{\lambda\mu}\) must satisfy Plücker relations for the KP
hierarchy.

\begin{remark}
 This expression show somethign interesting. We can formulate this
problem in terms of partitions defined in the phase
model. Equivalently, we can use the usual vertex operator
construction, and not the q-deformed version of Jing~\emph{\cite{Jing1991,
  Jing1995}}, that is much harder to deal with.
\end{remark}

If these coefficients satisfy the Toda hierarchy Plücker relations, then 
one can find a \(Q\)-dependent group element \(G_Q\in GL(\infty)\) such that 
\begin{equation}
  c_{\mu\nu} = \bra{\mu} G_Q \ket{\nu}\; ,
\end{equation}
where \(\ket{\mu}\) and \(\ket{\nu}\) are phase model
states. Unfortunatelly, we could not find any good reason to explore
this expansion in a consistent way. In what follows, we try to explore
alternative expansions that give a double Schur expansion.

%%%%%%%%%%%%%%%%%%%%%%%%%%%%%%%%%%%%%%%%%%%%%%%%%%
%%%%%%%%%%%%%%%%%%%%%%%%%%%%%%%%%%%%%%%%%%%%%%%%%%

\subsection{Scalar product: Supersymmetric Schur polynomials}

With this result, we now argue that~(\ref{eq:scalar-hl}) is also a tau
function of the Toda hierarchy in disguise. Let us decompose
\begin{equation}
 \sum_{m} q_m(\bm{y}, Q) z^m =
 \prod_j (1 + y_j (-Q z)) \prod_k (1 - y_k z)^{-1} = 
 \sum_{j, k} e_j(\bm{y}) h_k(\bm{y}) (- Q)^j z^{j+k}\; .
\end{equation}
And reorganizing this sum, we conclude that 
\begin{equation}
  q_m(\bm{y}, Q)  = \sum_{j=0}^m e_j(\bm{y}) h_{m-j}(\bm{y}) (- Q)^j =
 \sum_{j=0}^m e_j(-Q\bm{y}) h_{m-j}(\bm{y}) \; ,
\end{equation}
where in the last equality we have used the homogeneity of the elementary symmetric polynomials,
and \( -Q\bm{y} = (-Qy_1, -Qy_2, \dots)\). 

It is easier to consider the following expansions 
\begin{subequations}
\begin{equation}
  \begin{split}
    \prod_{j=1}^N (1 - (Q y_i)z) & = \prod_{j=1}^N e^{ \ln  (1 - (Q y_i)z)} = 
    \prod_{j=1}^N \exp \left( - \sum_{n>0} \frac{Q^n y^n_j}{n} z^n \right) \\ 
    & = \exp \left( - \sum_{n>0} \sum_{j=1}^N \frac{Q^n y^n_j}{n} z^n \right)  =
    \exp \left( - \sum_{n>0} t^{(Q)}_n z^n \right)  
  \end{split}
\end{equation}
where \(t^{(Q)}_n = \frac{Q^n}{n} \sum_j y_j^n = Q^n t'_n\), and 
\begin{equation}
    \prod_{j=1}^N \frac{1}{(1 - y_iz)} = \prod_{j=1}^N e^{- \ln  (1 - y_iz)} = 
    \prod_{j=1}^N \exp \left( \sum_{n>0} \frac{y_j}{n} z^n \right)=
    \exp \left(\sum_{n>0} t_n z^n \right) \; . 
\end{equation}
\end{subequations}

Therefore 
\begin{equation}
 \sum_{m} q_m(\bm{y}, Q) z^m =
 \prod_j \frac{1 + y_j (-Q z)}{1 - y_k z}
 = \exp \left( \sum_{n\geq 1} (t'_n - t_n^{(Q)})z^n \right)
 \equiv \exp \left( \sum_{n\geq 1} T_n z^n \right)\; ,
\end{equation}
with \(T_n = (1 - Q^n) t'_n\). Then, we conclude that \(q_m(\bm{y}, Q)\)
are homogeneous polynomials with respect the Miwa coordinates
\(\bm{T} = (T_1, T_2, \dots)\). Then
\begin{equation}
  \mathcal{S}_\lambda(\bm{t}', Q) = \det \left(h_{\lambda_i - i +j}(\bm{T})\right) = s_\lambda(\bm{T})\; .
\end{equation}
All in all, we conclude that the big Schur functions
\(S_\lambda(\bm{t}', Q)\) are ordinary Schur functions with respect
the coordinates \(\bm{T}\).

In fact, we can do even better. Supersymmetric (or Hook) Schur
functions~\cite{Berele:1983}, see also see~\cite{Macdonald:1998, Moens:2003},
\(s_\lambda(\bm{\alpha}/\bm{\beta})\) are defined as ordinary schur
functions evaluated at Miwa coodinates of the form
\begin{equation}
  T_n = \frac{1}{n} \left( \sum_{i=1}^{\dim(\alpha)} \alpha_i^n
       - \sum_{i=1}^{\dim(\beta)} (-\beta_i)^n\right)\; ,
\end{equation}
and comparing with the results above, we can see that the big Schur functions
\(S_\lambda(\bm{t}', Q)\)\; , is the supersymmetric Schur functions for \(\bm{\alpha} = \bm{y}\) 
and \(\bm{\beta} = - Q \bm{y}\), that is 
\begin{equation}
  S_\lambda(\bm{t}', Q) = s_\lambda[\bm{y}/(- Q\bm{y})]\; .
\end{equation}

Putting these facts together, we immediately conclude that since 
\begin{equation}
  \sum_\lambda s_\lambda(\bm{T}) s_\lambda (\bm{t}) \; ,
\end{equation}
is a Toda hierarchy tau function, we obviously have that 
\begin{equation}
  S_Q(N,M| \bm{x}, \bm{y}) \equiv S_Q(N,M| \bm{t}, \bm{T})
  = \sum_{\lambda \subseteq [N,N,M]} s_\lambda(\bm{T}) s_\lambda(\bm{t})\; , 
\end{equation}
is also a restricted tau function of the Toda hierarchy with respect
to \(\bm{t}\) and \(\bm{T}\).

Now we can consider an expansion that differs from
from~(\ref{eq:kf-exp}). Let us consider the supersymmetric Schur
polynomials in terms of ordinary Schur functions as~\cite[Sec. I.5,
  exerc. 23]{Macdonald:1998}
\begin{equation}
  s_{\lambda}(\bm{x}/\bm{y}) = \sum_\mu s_{(\lambda/\mu)'}(\bm{y}) s_\mu(\bm{x})\; ,
\end{equation}
where the prime denotes the conjugate diagram. Then
\begin{equation}
  \label{eq:tau-hl}
  \sum_{\lambda } s_\lambda(\bm{y}/ (- Q\bm{y})) s_\lambda(\bm{x}) =
  \sum_{\lambda,\mu } s_{(\lambda/\mu)'}(- Q\bm{y}) s_\lambda(\bm{x}) s_\mu(\bm{y})\; .
\end{equation}

From this expression we conclude (and speculate) the following. 

\begin{remark}
Since the skew Schur polynomials have determinant expressions,
therefore we have that the \(\bm{y}\)-dependent coefficients \(c_{\mu
  \lambda} = s_{(\lambda/\mu)'}(-Q\bm{y})\) have Jacobi-Trudi
expressions. It is tempting to think of these objects as
\(\bm{y}\)-dependent Pl\"ucker coordinates. In this sense, we would have a curve in 
the indinite Grassmannian, instead a point.
\end{remark}

\begin{remark}
Remeber that one of the easiest nontrivial solutions of the KP
hierarchies are the Schur functions
themselves~\emph{\cite{Zabrodin2018}}. We have just seen that the
big-Schur polynomials can be written as supersymmetric Shur
polynomials, and these are nothing but ordinary Schur polynomials in a
particular choice of Miwa coordiantes. Then, we also see that 
the big Schur polynomials are also KP tau functions. This direct conclusion 
is an alternative proof of this fact, see~\emph{\cite{Necoechea:2019wbg}}.
\end{remark}

Putting these facts together, we have that the left-hand side
of~(\ref{eq:tau-hl}) is a tau function with respect the coordinates
\(\bm{T}\) and \(\bm{t}\). Furthermore, from the Vertex operator
formalism, we write the Schur polynomials as
\begin{equation}
  s_{\lambda}(\bm{x}) = \bra{\lambda} e^{J_-}\ket{\bm{0}}\qquad
  s_{\mu}(\bm{y}) = \bra{\bm{0}} e^{J_+}\ket{\mu}\; ,
\end{equation}
we have
\begin{equation}
  \sum_{\lambda,\mu } s_\mu(\bm{y})c_{\mu\lambda}(Q,\bm{y}) s_\lambda(\bm{x}) = 
  \sum_{\lambda,\mu } \bra{\bm{0}} e^{J_+}\ket{\mu} c_{\mu\lambda}(Q,\bm{y})
  \bra{\lambda} e^{J_-}\ket{\bm{0}} \; ,
\end{equation}
therefore 
\begin{equation}
  c_{\mu\lambda}(Q, \bm{y}) = \bra{\mu} G_Q(\bm{y}) \ket{\lambda}\; ,
\end{equation}
where \(G_Q(\bm{y}) \in GL(\infty)\). We can find the expression for
this group element as follows. 

Remember that \(T_m = t'_m - Q^m t'_m\), therefore we write
\begin{subequations}
\begin{equation}
  \bra{\bm{0}} e^{J_+(\bm{t})} e^{J_-(\bm{T})} \ket{\bm{0}} = 
\bra{\bm{0}} e^{J_+(\bm{t})} e^{J_-(\bm{t}')  - J_-(\bm{t}^{(Q)})} \ket{\bm{0}} \; .
\end{equation}
From the Heisenberg algebra, we conclude that \([J_-(\bm{t}),
  J_-(\bm{t}')] = 0\), therefore
\begin{equation}
  \bra{\bm{0}} e^{J_+(\bm{t})} e^{J_-(\bm{T})} \ket{\bm{0}} = 
\bra{\bm{0}} e^{J_+(\bm{t})} e^{-J_-(\bm{t}^{(Q)})} e^{J_-(\bm{t}')}  \ket{\bm{0}} \; ,
\end{equation}
\end{subequations}
from where we finally conclude that 
\begin{equation}
  G_{Q}(\bm{y}) = \exp \left(-J_-(\bm{t}^{(Q)})\right) \qquad
  t^{(Q)}_n = \frac{Q^n}{n}\sum_j y_j^n\; . 
\end{equation}

In general, the expectation values with coordinate dependent group
elements \(G_{Q}(\bm{y})\) are not Toda hierarchy tau functions. But
in the above example, the coordinates combine among themselves and
generate a tau function with respect the coordiantes \(\bm{T}\) and
\(\bm{t}\).
 
%%%%%%%%%%%%%%%%%%%%%%%%%%%%%%%%%%%%%%%%%%%%%%%%%%
%%%%%%%%%%%%%%%%%%%%%%%%%%%%%%%%%%%%%%%%%%%%%%%%%%

%*%*%*%*%*%*%*%*%*%*%*%*%*%*%*%*%*%*%*%*%*%*%*%*%*
\section{Conclusions and Perspectives}
%*%*%*%*%*%*%*%*%*%*%*%*%*%*%*%*%*%*%*%*%*%*%*%*%*

Let us now consider the fermionic construction of this tau function. We write 
the limit \(N,M\to \infty\) of the scalar product
\begin{subequations}
\begin{equation}
  S_Q(\bm{T}, \bm{t}') = \sum_{\mu} s_\mu(\bm{T}) s_\mu(\bm{t}')
\end{equation}
that is quite quite trivial from the viewpoint of the free fermion construction, 
\begin{equation}
  S_Q(\bm{T}, \bm{t}') = \bra{\bm{0}} e^{\bm{J}_+(\bm{T})} e^{\bm{J}_-(\bm{t}')} \ket{\bm{0}}\; .
\end{equation}

Moreover, we also know that
\begin{equation}
  S_Q(\bm{t}, \bm{t}') = \sum_{\mu\nu} c_{\mu\nu} s_\mu(\bm{t}) s_\mu(\bm{t}')
\end{equation}
where \(n t = \sum_j x^n_j\) and \(n t' = \sum_j y^n_j\) are Miwa coordinates, then  
\begin{equation}
  S_Q(\bm{t}, \bm{t}') = \bra{\bm{0}} e^{J_+(\bm{t})} G_Q  e^{J_-(\bm{t}')} \ket{\bm{0}}
\end{equation}
and \(G_Q \in GL(\infty)\) that depends on \(Q\). We could also think of \(G_Q\)
as a twist
\(G_Q = e^{Q H} G e^{-Q H}\), with the action of \(H\in \mathfrak{gl}(\infty)\)
on ordinary fermions generating the Q-fermions. 
\marginpar[left]{\tiny Investigate this\\ twists}
\end{subequations}

It seems that we have some different representations for this tau function, 
and it is worth investigating what is happening here. That is what we want 
to do now. 

\printbibliography

\end{document}
