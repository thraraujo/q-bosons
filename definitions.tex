
% DEFINITIONS

%%%%%%%%%%%%%%%%%%%%%%%%
% ADDITIONAL PACKAGES
%%%%%%%%%%%%%%%%%%%%%%%% 
\usepackage{parskip}
\usepackage{tikz} % Drawing package
\usepackage[backend=biber, style=alphabetic]{biblatex}
\usepackage{listings}
\usepackage{ytableau}

\usetikzlibrary{intersections,decorations.text} % this is to make the cover
\usetikzlibrary{matrix, arrows.meta} % this is to improve diagrams

\setlength\parindent{13pt}

%\setdefaultlanguage[variant=brazilian]{portuguese}
%\setdefaultlanguage{english}

%%%%%%%%%%%%%%%%%%%%%%%%
% METADATA
%%%%%%%%%%%%%%%%%%%%%%%% 

%\usepackage{framed} 


% Official manual defines the options
\hypersetup{ 
	pdftitle={integrable systems, programming and so on},
	pdfsubject={High Energy Physics and so on}, 
	pdfauthor={author},
	pdfkeywords={gauge; susy; strings; fields; cft; python},
	colorlinks=true, % false: color frames ; true: color links
    linkcolor=myPurple, % Color of internal links (sections, pages and so on)
    citecolor=myPurple, % color for bibliographical citations
    urlcolor =myPurple, % color for linked URL
    linktocpage=true % link page to table of contents
}


% Options for listings package I saw the sniipet below here: 
% https://stackoverflow.com/questions/3175105/inserting-code-in-this-latex-document-with-indentation
\lstset{frame=tb,
  language=Python,
  aboveskip=3mm,
  belowskip=3mm,
  showstringspaces=false,
  columns=flexible,
  basicstyle={\small\ttfamily},
  numbers=none,
  numberstyle=\tiny\color{myPurple},
  keywordstyle=\color{myRed},
  commentstyle=\color{myBlue},
  stringstyle=\color{myPurple!80},
  breaklines=true,
  breakatwhitespace=true,
  tabsize=3
}

%%%%%%%%%%%%%%%%%%%%%%%% 
% COLORS
%%%%%%%%%%%%%%%%%%%%%%%% 
% see palette here: https://github.com/enkia/tokyo-night-vscode-theme
\definecolor{myPurple}{RGB}{90, 74, 120}
\definecolor{myBlue}{RGB}{15, 75, 110}
\definecolor{myRed}{RGB}{191,97,106}
\definecolor{myDarkGray}{RGB}{216, 222, 233}
\definecolor{myLightGray}{RGB}{236, 239, 244}

\definecolor{c1}{RGB}{129, 162, 193}%  {15, 75, 110} % myBlue
\definecolor{c2}{RGB}{216, 222, 233} % myDarkGray
\definecolor{c3}{RGB}{236, 239, 244} % myLightGray
\definecolor{c4}{RGB}{59, 66, 82}
\definecolor{c5}{RGB}{76, 86, 106}

% This color is a framed requirement
\definecolor{shadecolor}{RGB}{236, 239, 244}

%%%%%%%%%%%%%%%%%%%%%%%%
% THEOREM
%%%%%%%%%%%%%%%%%%%%%%%% 

\newtheorem{theorem}{Theorem}
\newtheorem{corollary}{Corollary}
\newtheorem{proposition}{Proposition}
\newtheorem{conjecture}{Conjecture}
\newtheorem{lemma}{Lemma}
\newtheorem{example}{Example}
\newtheorem{remark}{Remark}

%%%%%%%%%%%%%%%%%%%%%%%%
% MACROS
%%%%%%%%%%%%%%%%%%%%%%%% 

\DeclarePairedDelimiter{\bra}{\langle}{\rvert}
\DeclarePairedDelimiter{\ket}{\lvert}{\rangle}
\DeclarePairedDelimiter{\bbra}{\langle\!\langle}{\rvert}
\DeclarePairedDelimiter{\kket}{\lvert}{\rangle\!\rangle}
\DeclarePairedDelimiterX{\bracket}[2]{\langle}{\rangle}{#1\vert#2}
\DeclarePairedDelimiterX{\bbracket}[2]{\langle\!\langle}{\rangle\!\rangle}{#1\vert#2}
