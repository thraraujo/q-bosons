\section{Deformed fermions and Hall-Littlewood polynomials}

The deformed fermions and their relations with the Hall-Littlewood
polynomials were defined in~\cite{Jing1991, Jing1995}.
See also~\cite{Foda:2008hn, Wheeler:2010vmq, Sulkowski:2008mx}.

We start with the deformed anticommutation relations.
\begin{equation}
\begin{split} 
\{\psi_m, \psi_n\} &= Q\left( \psi_{m + 1} \psi_{n-1} + \psi_{n+1} \psi_{m-1} \right)\\
\{\psi_m^\star, \psi_n^\star\} &= Q\left( \psi_{m-1}^\star \psi_{n+1}^\star
+ \psi_{n-1}^\star \psi_{m+1}^\star \right)\\
\{\psi_m, \psi_n^\star\} &= Q\left( \psi_{m-1} \psi_{n-1}^\star
+ \psi_{n+1}^\star \psi_{m+1} \right) + (1-Q)^2\delta_{m,n}
\end{split}	
\end{equation}
where \(Q\in \mathbb{C}\). Observe that the indices \(m, n\) are now
integers. The limit \(Q=0\) gives the usual charged free fermions we
have just considered.

The Fock space generated by these operators is similar to the
undeformed one. In particular, we have the vacuum \(\ket{\bm{0}}\)
which is annihilated as
\begin{equation}
\psi_m \ket{\bm{0}} = \psi_{n}^\star \ket{\bm{0}} = 0 \qquad  m < 0 \quad n \geq 0\; .
\end{equation}
Similarly, one can define the charged vacua \(\ket{-\ell}\), with \(\ell > 0\), as
\begin{equation}
\begin{split}
\ket{-\ell} = \psi^\ast_{-\ell}\cdots \psi^\ast_{-1}\ket{\bm{0}}\; .
\end{split}
\end{equation}
These states satisfy the relations
\begin{equation}
  \psi_m\ket{-\ell} = \left\{
\begin{array}{ll}
 0 & \quad m < -\ell\\
 \ket{-\ell + 1} & \quad m = -\ell\\
\end{array}
\right.
\end{equation}
The partition states \(\ket{\lambda}\) with \(\lambda = (\lambda_1,
\dots, \lambda_\ell)\) are defined via
\begin{equation}
\ket{\lambda} = \psi_{\lambda_1 - 1} \cdots \psi_{\lambda_\ell - \ell}\ket{-\ell}\; .
\end{equation}

%%%%%%%%%%%%%%%%%%%%%%%%%%%%%%%%%%%%%%%%%%%%%%%%%%
%%%%%%%%%%%%%%%%%%%%%%%%%%%%%%%%%%%%%%%%%%%%%%%%%%

\subsection{Deformed Heisenberg Algebra}

We also have the deformed Heisenberg algebra
\begin{equation}
[H_m, H_n] = \frac{m}{1-Q^{|m|}}\delta_{m+n}
\end{equation}
where
\begin{equation}
\begin{split}
  H_m = & \frac{1}{1-Q}\sum_{j\in \mathbb{Z}} :\psi_j\psi^\star_{j+m}:\qquad \forall\ m>0\\
  H_m = & \frac{1}{(1-Q)(1-Q^{|m|})}\sum_{j\in \mathbb{Z}} :\psi_j\psi^\star_{j+m}:\qquad \forall\ m<0\; .
\end{split}
\end{equation}
And in the limit \(Q=0\) we recover the components of the current
\(J(z)\).  As in the undeformed case, the normal ordering is not very
important in the case \(m\neq n\). It can also be shown that
\begin{equation}
\label{eq:hpsi}
[\psi_n, H_m] = - \psi_{n -m}\qquad 
[\psi^\star_n, H_m] = \psi^\star_{n + m}\; .
\end{equation}

From these anticommutation relations, it can be easily shown that
\begin{equation}
  \label{eq:modulus}
   \bra{\bm{0}} \prod_{j\geq 1} H_{j}^{k_j} H_{-j}^{k_j}\ket{\bm{0}}
    = \prod_{j\geq 1} \frac{ k_j! j^{k_j}}{(1 - Q^j)^{k_j}}
\end{equation}
therefore
\begin{equation}
    \bracket{\vec{k}}{\vec{k}'}
    = \prod_{j\geq 1} \frac{ k_j! j^{k_j}}{(1 - Q^j)^{k_j}} \delta_{\vec{k}\vec{k}'}
    = z_{\vec{k}} \delta_{\vec{k}\vec{k}'}\; .
\end{equation}
where \(z_{\lambda} = z_{\vec{k}} = \prod_j k_j! j^{k_j}/(1 -
Q^j)^{k_j}\) and the partition \(\lambda = (\lambda_1, \lambda_2,
\dots, \lambda_m)\) can be represented as \(\lambda = (1^{k_1}
2^{k_2} 3^{k_3}\dots)\). Finally, we also have
\begin{equation}
\ket{\vec{k}} = \prod_{j\geq 1} H_{-j}^{k_j}\ket{\bm{0}}\; . 
\end{equation}
In other words, the expression~(\ref{eq:modulus}) describes an inner product. 

%%%%%%%%%%%%%%%%%%%%%%%%%%%%%%%%%%%%%%%%%%%%%%%%%%
%%%%%%%%%%%%%%%%%%%%%%%%%%%%%%%%%%%%%%%%%%%%%%%%%%

\subsection{Vertex Operators}

And from these operators, the vertex operators (or transfer matrices)~\cite{Jing1991, Jing1995}
are \(\Gamma_\pm(w,Q)=\exp (\phi_\pm(w,Q))\), where
\begin{equation}
\phi_\pm(w,Q) = \sum_{m>0} \frac{1-Q^m}{m} w^{\mp m} H_{\pm n}\; .
\end{equation}
We can also write
\begin{equation}
    \Gamma_-(w,Q)  = \exp\left( \sum_{m>0} \frac{1-Q^m}{m} w^{\mp m} H_{\pm n} \right)
    = \sum_{n\geq 0} \Gamma_n w^n  \; ,
\end{equation}
where
\begin{equation}
  \Gamma_n = \sum_{k_1 + 2k_2 + \dots = n} \frac{1}{z_{\vec{k}}} H_{-1}^{k_1} H_{-2}^{k_2}\cdots
    \equiv \sum_{\lambda \vdash n} \frac{1}{z_{\lambda}} \bm{H}_{-\lambda} \; , \qquad 
 \bm{H}_{\lambda} = H_{-\lambda_1} H_{-\lambda_2}\cdots H_{-\lambda_m}\; , 
\end{equation}
and the sum is taken over all partitions of \(n\).

As in the underformed case, we have
\begin{equation}
 \begin{split}
    \prod_{j=1}^N \Gamma_- (x_j, Q) &
    = \exp \left( \sum_{m>0} \sum_{j=1}^N \frac{(1 - Q^m)}{m} z_j^m H_{-m}
    \right) \\ 
    & = \exp \left( \sum_{m>0} (1 - Q^m) t_m H_{-m} \right) \equiv e^{\bm{J}_-(\bm{t}, Q)}\; .
 \end{split}
\end{equation}
where \(t_m = \frac{1}{m} \sum_{j=1}^N x_j\) are the usual Miwa
coordinates.


%%%%%%%%%%%%%%%%%%%%%%%%%%%%%%%%%%%%%%%%%%%%%%%%%%
%%%%%%%%%%%%%%%%%%%%%%%%%%%%%%%%%%%%%%%%%%%%%%%%%%

\subsection{Hall-Littlewood functions}

Now that we have seen that the construction of the deformed fermion is
completely analogous to the undeformed case, we can define the The
Hall-Littlewood polynomials as
\begin{equation}
e^{\bm{J}_-(\bm{t}, Q)} \ket{\bm{0}} = 
  \prod_{j=1}^N \Gamma_-(x_j, Q)  \ket{\bm{0}}
  = \sum_{\lambda}P_{\lambda}(x_1, x_2, \dots, x_N;Q) \ket{\lambda}\; .
\end{equation}
where the \(P_\lambda\) are known as the Hall-Littlewood polynomials.
One can define \(T_m = (1- Q^m) t_m H_m\), and using the definition of
the complete homogeneous polynomials, we have
\begin{equation}
  \sum_{\lambda}P_{\lambda}(x_1, x_2, \dots, x_N;Q) \ket{\lambda} =
  \sum_{n\geq 0}\sum_{k_1 + 2k_2 + \dots =n} \left( \prod_{i\geq 1}
  \frac{(1-Q^i)^{k_i} t_i^{k_i}}{k_i!} H_{-i}^{k_i}\right)  \ket{\bm{0}}\; .
\end{equation}
As in the undeformed case, the Fock space is decomposed in terms of
its levels, or number of boxes in the Young diagram.

It may be interesting to calculate the first terms of this expansion
to see that everything works accordingly. Let us write the the first
few terms.
\begin{equation}
  \begin{split}
    e^{\bm{J}_-(\bm{t}, Q)} \ket{\bm{0}} & =\left( 1 + (1 -
    Q)t_{1}H_{-1} + (1 - Q^2)t_{2}H_{-2} + \frac{(1 -
      Q)^2}{2}t_{1}^2H_{-1}^2 + \cdots \right) \ket{\bm{0}} \\ & =
    P_{(0)}(\bm{t}, Q) \ket{\bm{0}} +
    \ytableausetup{centertableaux, smalltableaux} P_{(1)}(\bm{t}, Q) |\ydiagram{1} \rangle +
    P_{(2)}(\bm{t}, Q) |\ydiagram{2} \rangle+ P_{(1,1)}(\bm{t}, Q)
    \left|\ydiagram{1,1} \right\rangle+ \cdots
    \end{split}
\end{equation}

\emph{n=0:} We obviously have \(P_{(0)} =1\). 

\emph{n=1:} Now we have 
\begin{equation}
    P_{(1)}(\bm{t}, Q) |\ydiagram{1} \rangle = (1 - Q)t_{1}H_{-1} \ket{\bm{0}} \; .
\end{equation}
Using now that 
\begin{equation}
  \begin{split}
  H_{-1} \ket{\bm{0}} & = \frac{1}{(1-Q)^2} \left(\sum_{j < 1} +
  \sum_{j \geq 1} \right) \psi_j\psi^\star_{j-1} \ket{\bm{0}} =
  \frac{1}{(1-Q)^2} \sum_{j < 1} \psi_j\psi^\star_{j-1} \ket{\bm{0}}
  \\ & = \frac{1}{(1-Q)^2} \left ( \sum_{j < -1}
  \psi_j\psi^\star_{j-1} + \psi_0\psi^\star_{-1} +
  \psi_{-1}\psi^\star_{-2}\right)\ket{\bm{0}}
  \end{split}
\end{equation}
where we have used that \(\psi_m^\star \ket{\bm{0}} = 0 \) for \(m\geq
0\). Moreover, let us identify the one box state as \(|\ydiagram{1}
\rangle = \psi_0 \psi_{-1}^\ast \ket{\bm{0}}\).  From the
anticommutation relations, it is easy to see that \(\psi_j
\psi_{j-1}^\ast \ket{\bm{0}} = 0\) for \(j < -1\), as a result of
\(\psi_j\ket{\bm{0}} = 0\). Therefore
\begin{equation}
  H_{-1} \ket{\bm{0}} = \frac{1}{(1-Q)^2} \left(|\ydiagram{1}\rangle
  + \psi_{-1}\psi^\star_{-2}\ket{\bm{0}} \right)\; .
\end{equation}
From the anticommutation relations, we have
\begin{equation}
  \psi_{-1}\psi^\star_{-2}\ket{\bm{0}} = Q\psi_{-1}^\star\psi_{0}\ket{\bm{0}}\; ,
\end{equation}
then
\begin{equation}
  H_{-1} \ket{\bm{0}} = \frac{1}{(1-Q)^2} \left(|\ydiagram{1}\rangle
  + Q \psi_{-1}^\star\psi_{0}\ket{\bm{0}} \right)\; .
\end{equation}
From the anticommutation relations, we have
\begin{equation}
  \begin{split}
    \psi_{-1}^\star\psi_{0}\ket{\bm{0}} & = \left( - \psi_0
    \psi_{-1}^\star + Q \psi_{-1} \psi_{-2}^\star + Q \psi_0^\star
    \psi_1 \right) \ket{\bm{0}} \\ & = - |\ydiagram{1}\rangle +
    Q^2\psi_{-1}^\star\psi_{0}\ket{\bm{0}} + Q \psi_0^\star \psi_1
    \ket{\bm{0}} \; .
  \end{split}
\end{equation}
that is
\begin{equation}
 (1 - Q^2)\psi_{-1}^\star\psi_{0}\ket{\bm{0}} = - |\ydiagram{1}\rangle
  + Q \psi_0^\star \psi_1 \ket{\bm{0}}
\end{equation}

Now, we have 
\begin{equation}
  \psi_0^\star \psi_1 \ket{\bm{0}} = Q \psi_0 \psi_{-1}^\star \ket{\bm{0}} +
  Q \psi^\star_1 \psi_2 \ket{\bm{0}}
\end{equation}
The second term vanishes since its commutation generates terms with
positive indices \(\psi_n^\star\) acting on the vacuum. We can see
this fact using the identity (4.2.7) of~\cite{Wheeler:2010vmq},
\begin{equation}
  \psi_m^\star \psi_n = Q \psi_{n-1} \psi_{m-1} + (Q^2 - 1)
  \sum_{i\geq 0} Q^i \psi_{n+i} \psi_{m+i}^\star + (1-Q)\delta_{m,n}\;
  .
\end{equation}
Therefore
\begin{equation}
  \psi_0^\star \psi_1 \ket{\bm{0}} = Q |\ydiagram{1}\rangle\; , 
\end{equation}
and we conclude that
\begin{equation}
  \begin{split}
    (1 - Q^2)\psi_{-1}^\star\psi_{0}\ket{\bm{0}} & = -
    |\ydiagram{1}\rangle + Q \psi_0^\star \psi_1 \ket{\bm{0}} \\ & = -
    (1-Q^2) |\ydiagram{1}\rangle\quad \Rightarrow \quad
    \psi_{-1}^\star\psi_{0}\ket{\bm{0}} = - |\ydiagram{1}\rangle\; .
  \end{split}
\end{equation}
And we finally conclude that 
\begin{equation}
H_{-1} \ket{\bm{0}} = \frac{1}{(1- Q)}|\ydiagram{1}\rangle \qquad
\Rightarrow \qquad P_{(1)} = t_1\;,
\end{equation}
that is the expected result. 

\emph{n=2:} The calculation of higher state partitions can be
daunting. So, it might be usefult to consider some identities
originally proved in~\cite{Foda:2008hn}. Moreover, it is easier
to calculate the elements
\begin{align}
  \ytableausetup{centertableaux, smalltableaux} 
  P_{(2)}(\bm{t}, Q) & = \frac{1}{\langle \ydiagram{2}  | \ydiagram{2} \rangle} \langle \ydiagram{2}|
  \left((1 - Q^2) t_2 H_{-2} + \frac{(1-Q^2)}{2} t_1^2 H_{-1}^2\right) \ket{\bm{0}} \\
  P_{(1,1)}(\bm{t}, Q) & = \frac{1}{\left\langle \ydiagram{1,1}  | \ydiagram{1,1} \right\rangle}
  \left\langle \ydiagram{1,1}\right|
  \left((1 - Q^2) t_2 H_{-2} + \frac{(1-Q^2)}{2} t_1^2 H_{-1}^2\right) \ket{\bm{0}} \; .
\end{align}

First of all, it has been shown that given two partition states \(\ket{\lambda}\) and \(\ket{\mu}\),
they satisfy
\begin{equation}
  \bracket{\lambda}{\mu} = b_{\lambda} \delta_{\mu\nu}\qquad
  b_\lambda = \prod_{i=1}^\infty \prod_{j=1}^{p_i(\lambda)} (1 - Q^j)\; ,
\end{equation}
where \(p_i(\lambda)\) is the number of rows of size \(i\). Therefore, the normalization factors
are
\begin{equation}
  \langle \ydiagram{2}  | \ydiagram{2} \rangle = 1 - Q\qquad
    \left\langle \ydiagram{1,1}  | \ydiagram{1,1} \right\rangle= (1 - Q)(1- Q^2)\; .
\end{equation}
We now have four different terms to calculate.

\paragraph{1.} We first have
\begin{equation}
  \begin{split}
\langle \ydiagram{2}| H_{-2} \ket{\bm{0}} & = \bra{-1} \psi_1^\ast H_{-2} \ket{\bm{0}} = 
\bra{-1} [\psi_1^\ast, H_{-2}] \ket{\bm{0}} \\
& = \bra{-1} \psi_{-1}^\ast\ket{\bm{0}} = \bra{0} \psi_{-1} \psi_{-1}^\ast \ket{\bm{0}} = 1\; , 
  \end{split}
\end{equation}
where we have used that
\begin{equation}
 \bra{-\ell} H_{-m} = 0 \quad \forall m \geq 1\; ,
\end{equation}
and the commutation relations~(\ref{eq:hpsi}). The proof that
\(\bra{0} \psi_{-1} \psi_{-1}^\ast \ket{\bm{0}} = 1\)  is not very difficult, but a proof can be found in
equations~(4.2.17) and (4.2.18) of~\cite{Wheeler:2010vmq}.

\paragraph{2.} Now
\begin{equation}
  \begin{split}
    \langle \ydiagram{2}| H_{-1}^2 \ket{\bm{0}} & = \bra{-1} \psi_1^\ast H_{-1}^2 \ket{\bm{0}} = 
    \bra{-1} [\psi_1^\ast, H_{-1}^2] + H_{-1}^2  \psi_1^\ast\ket{\bm{0}} \\
    & = \bra{-1} H_{-1}\underbrace{[\psi_1^\ast, H_{-1}]}_{\psi^\ast_0} +
    \underbrace{[\psi_1^\ast, H_{-1}]}_{\psi_0^\ast} H_{-1}  \ket{\bm{0}}
     = \bra{-1} [\psi_0^\ast, H_{-1}]  \ket{\bm{0}} =  \\
     &= \bra{-1} \psi_{-1}^\ast  \ket{\bm{0}} = \bra{\bm{0}} \psi_{-1} \psi_{-1}^\ast  \ket{\bm{0}} =  1\; .
  \end{split}
\end{equation}


\paragraph{3.} Moving to the next diagram
\begin{equation}
  \begin{split}
    \left\langle \ydiagram{1,1}\right| H_{-2} \ket{\bm{0}}
    & = \bra{-2} \psi_{-1}^\ast \psi_0^\ast H_{-2} \ket{\bm{0}} =
    \bra{-2} \psi_{-1}^\ast [\psi_0^\ast, H_{-2}] \ket{\bm{0}} =
    \bra{-2} \psi_{-1}^\ast \psi_{-2}^\ast\ket{\bm{0}}\\
    & = \bra{-2} \left[-\psi_{-2}^\ast \psi_{-1}^\ast
      + Q\left( \psi_{-3}^\ast \psi_0^\ast +
      \psi_{-2}^\ast \psi_{-1}^\ast \right)  \right]\ket{\bm{0}}
    = - (1 - Q) \bra{-2} \psi_{-2}^\ast \psi_{-1}^\ast \ket{\bm{0}}\\
  \end{split}
\end{equation}
where we have used the anticommutation relations. Finally, using the relation
\(\bra{-\ell} \psi_{-\ell}^\ast = \bra{-\ell + 1}\), we find
\begin{equation}
    \left\langle \ydiagram{1,1}\right| H_{-2} \ket{\bm{0}}
    = - (1 - Q) \bra{-1} \psi_{-1}^\ast \ket{\bm{0}} = - (1 - Q)\; .
\end{equation}


\paragraph{4.} Finally
\begin{equation}
\begin{split}
\left\langle \ydiagram{1,1}\right| H_{-1}^2 \ket{\bm{0}}
& = \bra{-2} \psi_{-1}^\ast \psi_0^\ast H_{-1}^2 \ket{\bm{0}} =
\bra{-2} \psi_{-1}^\ast [\psi_0^\ast, H_{-1}^2] \ket{\bm{0}} \\
& = \bra{-2} \psi_{-1}^\ast \left( H_{-1}[\psi_0^\ast, H_{-1}] +
    [\psi_0^\ast, H_{-1}] H_{-1}\right) \ket{\bm{0}} \\
& = \bra{-2} \psi_{-1}^\ast \left( H_{-1}\psi_{-1}^\ast + \psi_{-1}^\ast H_{-1}\right) \ket{\bm{0}} \\
& = \bra{-2} \left([\psi_{-1}^\ast, H_{-1}]\psi_{-1}^\ast + \psi_{-1}^\ast \psi_{-1}^\ast H_{-1}\right) \ket{\bm{0}} 
 = \bra{-2} \left(\psi_{-2}^\ast\psi_{-1}^\ast + Q \psi_{-2}^\ast \psi_{0}^\ast H_{-1}\right) \ket{\bm{0}} \\
& = \bra{-2} \left(\psi_{-2}^\ast \psi_{-1}^\ast + Q \psi_{-2}^\ast [\psi_{0}^\ast, H_{-1}]
\right) \ket{\bm{0}}
= \bra{-2} \left(\psi_{-2}^\ast \psi_{-1}^\ast + Q \psi_{-2}^\ast \psi_{-1}^\ast
\right) \ket{\bm{0}} = 1 + Q
\\
\end{split}
\end{equation}

Putting all these facts together, we find that
\begin{subequations}
\begin{equation}
\begin{split}
P_{(2)} & = \frac{1}{(1 - Q)}\left( (1 - Q^2) t_2 + \frac{(1 - Q)^2}{2}t_1^2 \right)\\
 & =(1 + Q) t_2 + \frac{(1 - Q)}{2}t_1^2 
  \end{split}
\end{equation}
and 
\begin{equation}
\begin{split}
P_{(1,1)} & = \frac{1}{(1 - Q)(1 - Q^2)}\left( - (1 - Q)(1 - Q^2) t_2 + \frac{(1 + Q)(1 - Q)^2}{2}t_1^2 \right)\\
 & = \frac{1}{2}t_1^2  - t_2 = e_2\; .
  \end{split}
\end{equation}
\end{subequations}

