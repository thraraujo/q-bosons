\section{Transfer matrices}

This section is a review of the transfer matrix
formalism~\cite{Alexandrov:2012tr, Okounkov:2003sp, Justin2008}.  Let
us now define the following operators
\begin{subequations}
\begin{equation}
\Gamma_\pm(z):=\exp \phi_\pm(z)\; ,
\end{equation}		
where 
\begin{equation}
\phi_\pm(z):= \sum_{\pm n\geq 1}\frac{z^{n}}{n} J_{n}\; .
\end{equation}
These are the creation and annihilation components of the bosonic
vertex operator \(\mathcal(z) = \exp \phi(z)\).  By analogy with
statistical mechanics systems, we may consider that the time evolves
in a series of small steps, and for this reason we say that
\(\Gamma_\pm(x_i)\) is a transfer operator.
\end{subequations}

Moreover, observe that 
\begin{subequations}
\begin{equation}
\begin{split}
\prod_{a=1}^m \Gamma_\pm(x_a) & = \exp \sum_{a = 1}^m \phi_\pm(x_a) \\
	& = \exp \sum_{a = 1}^m  \sum_{\pm n\geq 1}\frac{x_a^{n}}{n} J_{n}\\
	& = \exp  \sum_{n\geq 1} \left(\sum_{a = 1}^m  \frac{x_a^{n}}{n} \right) J_{n}
\end{split}
\end{equation}
and using the Miwa coordinates (\ref{sympol}), we finally obtain
\begin{equation}
\label{oper}
\boxed{\prod_{i\geq 1} \Gamma_\pm(x_a) = e^{J_\pm (\mathbf{t})}}\; .
\end{equation}	
\end{subequations}

Using the expansion of skew Schur polynomials~(\ref{eq:skschur}), we have 
\begin{subequations}
\begin{equation}
   e^{\bm{J}_-(\bm{t}}\ket{\mu} \equiv 
   \prod_{a=1}^m \Gamma_-(x_a) \ket{\mu}
   = \sum_\lambda s_{\lambda/\mu}(\mathbf{t})\ket{\lambda}\; .
\end{equation}

One now can use the special case 1 we discussed above where \(\vec{x}
= (z,0,0, \dots) \), that as we alrealdy seen, corresponds to the
choice \(t_n = \tfrac{x^n}{n}\), \(h_k = z^k\) and \(e_l = 0\) for \(l > 1\).
Moreover, using~\ref{eq:schur:case1}, we can see that
\begin{equation}
  \begin{split}
    e^{\bm{J}_-(\{z^k/k\})} \ket{\mu} & \equiv \Gamma_-(z) \ket{\mu}
    = \sum_\lambda s_{\lambda/\mu}(\{z^k/k\})\ket{\lambda}\\
    & = \sum_{\lambda \succ \mu}z^{|\lambda| - |\mu|}\ket{\lambda}\; .
  \end{split}
\end{equation}
Therefore, the particular case where \(z=1\) gives the beautiful expression
\begin{equation}
    \Gamma_-(1) \ket{\mu} = \sum_{\lambda \succ \mu}\ket{\lambda}\; .
\end{equation}
It is also easy to show that
\begin{equation}
    \Gamma_+(1) \ket{\mu} = \sum_{\lambda \prec \mu}\ket{\lambda}\; .
\end{equation}
\end{subequations}
For example, the case \(\mu = \bm{\emptyset}\) gives an expansion of
one row diagrams. An in fact, \(\Gamma_+(1)^n \ket{\bm{0}}\) gives an
expansion of interlaced Young diagrams with a maximum of \(n\) rows. 
For more information on this topic, see~\cite{Okounkov:2003sp, Okounkov2001}

%%%%%%%%%%%%%%%%%%%%
%%%%%%%%%%%%%%%%%%%%


\subsection{Sum over partitions}

The tau-function is a function defined on the orbit of the vacuum
\(\ket{\bm{0}}\) under the action of the group element \(g\in
\widehat{GL}(\infty)\). It is defined as the Vacuum Expectation Value (VEV)
\begin{equation}
  \tau(\mathbf{t},g):= \bra{\bm{0}} e^{\mathbf{J}_+(\mathbf{t}) } g
  \ket{\bm{0}}\; , \qquad g\in \widehat{GL}(\infty)\; .
\end{equation}
This object is intrumental in the study of several integrable systems.
Additionally, we have seen that the coherent state
\(e^{\mathbf{J}_-(\mathbf{t})} \ket{\bm{0}}\) can be written in terms
of Schur polynomials. Therefore, we want to study some aspects of
these expectation values.

Let us first define the following fermionic state
\begin{equation}
\begin{split}
   \ket{\widetilde{\Psi}_0} & = \lim_{d\to \infty} \left[ 
   \sum_{\bm{a}^{<}_d;\bm{b}^{<}_d}
   q^{\frac{1}{2}\sum_{i=1}^d (a_i+b_i)} \prod_{i=1}^d \psi^\star_{a_i}\psi_{-b_i}
   \right] |\mathtt{half}\rangle \\ 
   & = q^{L_0/2} \left( 1 + \sum_{a_1; b_1}\psi^\star_{a_1} \psi_{-b_1} + 
   \sum_{\bm{a}^{<}_2 ; \bm{b}^{<}_2} \psi^\star_{a_2} \psi_{-b_2} \psi^\star_{a_1}\psi_{-b_1} + \cdots 
   \right) |\mathtt{half}\rangle 
\end{split}
\end{equation}
where \(\bm{a}^{<}_d = \{ (a_1, \dots, a_d)\ | \ 0<a_1<\cdots < a_d\}\)
and \(\bm{b}^{<}_d = \{ (b_1, \dots, b_d)\ | \ 0<b_1<\cdots < b_d\}\).
We can immediately write this state as
\begin{equation}
\ket{\Psi_0}_f = \sum_{\lambda} q^{\frac{1}{2}|\lambda|} \ket{\lambda} \; ,
\quad \textrm{where}\quad 
\ket{\lambda} = \prod_{j=1}^{d}\psi^\star_{a_j} \psi_{-b_j} \ket{\bm{0}}\; .
\end{equation}
Diagrammatically, this state is the sum
\begin{equation}
\ket{\widetilde{\Psi}_0} =  \ket{\bm{0}} + 
\ytableausetup{centertableaux, smalltableaux}
q^{1/2} |\ydiagram{1} \rangle +
q  \left|\ydiagram{1,1} \right\rangle +
q |\ydiagram{2} \rangle+ 
q^{3/2} \left| \ydiagram{3} \right\rangle+ q^{3/2} \left| \ydiagram{1,2} \right\rangle+
q^{3/2} \left| \ydiagram{1,1,1} \right\rangle+\cdots
\end{equation}
From the fermion-boson correspondence we have discussed earlier, one
can write this state in terms of bosonic states \(\ket{\vec{k}}\) as
\begin{equation}
   \ket{\widetilde{\Psi}_0}  = q^{L_0/2} \sum_{\lambda} \ket{\lambda} = 
q^{L_0/2} \sum_{\vec{k}} \sum_{\lambda} \frac{\chi_\lambda[C(\vec{k})]}{z_{\vec{k}}} \ket{\vec{k}}
\end{equation}
where \(|\lambda| = \sum_j j k_j\).

Using that under the fermion-boson correspondence, the bosonic states might also be
written in terms of conjugacy classes of the symmetric group
\(\mathfrak{S}_n\) (or the partitions of \(n\)), let us write 
\begin{equation}
  \label{eq:bos:states}
  \begin{split}
    \ket{\vec{k}}  & = \ket{(k_1, k_2, \dots, k_m)}  
    = \ket{(1^{k_1} 2^{k_2} 3^{k_3}\dots m^{k_m})} \\
    & \equiv | (\underbrace{m, \dots, m}_{k_m},m-1, \dots ,3, \underbrace{2, \dots, 2}_{k_2},
  \underbrace{1, \dots, 1}_{k_1}, 0\dots)\rangle \equiv
  \kket{\lambda}\; ,
  \end{split}
\end{equation}
where we use the double notation \(\kket{\lambda}\) to denote the bosonic partition. 
These states are not normalized, and in fact, they satisfy
\begin{equation}
\bracket{\vec{k}'}{\vec{k}} = z_{\vec{k}} \delta_{\vec{k}', \vec{k}}\; .
\end{equation}
Therefore, one can define the state
\begin{equation}
\ket{\Psi_0} = q^{L_0/2} \sum_{k\geq 0} \sum_{\vec{k}} \frac{1}{\sqrt{z_{\vec{k}}}} \ket{\vec{k}}\; . 
\end{equation}
Using the bosonization formula, we can write this state as
\begin{equation}
  \ket{\Psi_0} = q^{L_0/2} \sum_{\lambda} \sum_{\vec{k}}
  \frac{1}{\sqrt{z_{\vec{k}}}} \chi_\mu[C(\vec{k})] \ket{\lambda}\; ,
\end{equation}
and the number of boxes in the partition is given by \(|\lambda| = \sum_j j k_j\).
Using the explicit expression for the characters, we have that the
first terms in this expansion are
\begin{equation}
  \begin{split}
\ket{\Psi_0} & = \ket{\bm{0}} + 
\ytableausetup{centertableaux, smalltableaux}
q^{1/2} |\ydiagram{1} \rangle +
q \sqrt{2} |\ydiagram{2} \rangle+ 
q^{3/2} \left[ \frac{1}{\sqrt{6}} (1 + \sqrt{2} + \sqrt{3})\left| \ydiagram{3} \right\rangle + \right. \\
& + \left. \frac{1}{\sqrt{6}} (2 - \sqrt{2}) \left| \ydiagram{1,2} \right\rangle+
 \frac{1}{\sqrt{6}} (1 + \sqrt{2} - \sqrt{3})\left| \ydiagram{1,1,1} \right\rangle\right] + \\
& + q^2 \left[ \frac{1}{12}\left( \sqrt{6} + 3 \sqrt{2} + 4 \sqrt{3}
  + 12\right) \left| \ydiagram{4} \right\rangle 
  + \frac{1}{4}\left( \sqrt{6} - \sqrt{2} \right)\left| \ydiagram{1,3} \right\rangle \right.\\
  & + \left. \frac{1}{2}\left( \sqrt{2} - \frac{2\sqrt{3}}{3} + \frac{\sqrt{6}}{3} \right)
  \left| \ydiagram{2,2} \right\rangle + 
  \frac{1}{4}\left( \sqrt{6} - \sqrt{2} \right) \left| \ydiagram{1,1,2} \right\rangle + 
  \frac{1}{12}\left( \sqrt{6} + 3 \sqrt{2} + 4 \sqrt{3}
  - 12\right) \left| \ydiagram{1,1,1,1} \right\rangle + \dots
  \right]
  \end{split}
\end{equation}

\subsection{Vacuum Expectation Values and Euler Formula}

From the orthonormality of the fermionic plane partition states, we have
\begin{equation}
\begin{split}
  Z & = \bracket{\widetilde{\Psi}_0}{\widetilde{\Psi}_0}\\
  & = 1 + q + 2 q^2 + 3 q^3 + 5 q^4 + \dots = \sum_{n\geq 0} p(n) q^n \\
  & = \prod_{n=1}^\infty \frac{1}{1-q^n}\; .
\end{split}
\end{equation}
that is the Euler formula for integer partitions. It is also easy to see that 
\begin{equation}
\begin{split}
  Z & = \bracket{\Psi_0}{\Psi_0}\\
  & = 1 + q + 2 q^2 + 3 q^3 + 5 q^4 + \dots = \sum_{n\geq 0} p(n) q^n \\
  & = \prod_{n=1}^\infty \frac{1}{1-q^n}\; .
\end{split}
\end{equation}

%\(\clubsuit\) It would be very interesting to see what is the subgroup of
%\(\widehat{\mathfrak{gl}}(\infty)\) that keeps this ground state invariant. 
% See Macmahon function here~\cite{Okounkov:2003sp, Okounkov2001, Dijkgraaf:2008ua}.

